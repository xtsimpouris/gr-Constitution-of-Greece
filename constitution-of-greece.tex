\documentclass[twoside, a4paper, 10pt]{article}

\usepackage{OfficialGazette}

\InitOfficialGazette{ΠΡΩΤΟ}{9}{6}{1975}{111}{623}{0}
\begin{document}        % Enarksi eggrafou

\begin{multicols}{2}

% Gia na einai mesa sto distilo
% \TOCprint

\Ekdotis{Ο ΠΡΟΕΔΡΟΣ ΤΗΣ ΕΛΛΗΝΙΚΗΣ ΔΗΜΟΚΡΑΤΙΑΣ}
Έχοντες υπ' όψει το από 7ης Ιουνίου 1975 ΙΒ' Ψήφισμα της Ε' Αναθεωρητικής Βουλής των Ελλήνων “περί ψηφίσεως και θέσεως εις ισχύν του νέου Συντάγματος της Χώρας”, αποφασίζομεν:

\begin{enumerate}
\item[Α.] Να δημοσιευθή δια της Εφημερίδος της Κυβερνήσεως το δια του Ψηφίσματος τούτου τιθέμενον εν ισχύει οριστικόν Σύνταγμα της Ελλάδος, έχον ως έπεται:

\begin{BigQuote}
\TOCSection{ΣΥΝΤΑΓΜΑ  ΤΗΣ  ΕΛΛΑΔΟΣ}{}
\BigTitle{ΣΥΝΤΑΓΜΑ  ΤΗΣ  ΕΛΛΑΔΟΣ}
\BigTitle{ΕΙΣ ΤΟ ΟΝΟΜΑ ΤΗΣ ΑΓΙΑΣ ΚΑΙ ΟΜΟΟΥΣΙΟΥ ΚΑΙ ΑΔΙΑΙΡΕΤΟΥ ΤΡΙΑΔΟΣ}
\BigTitle{Η Ε' ΑΝΑΘΕΩΡΗΤΙΚΗ ΒΟΥΛΗ ΤΩΝ ΕΛΛΗΝΩΝ ΨΗΦΙΖΕΙ}

\Part{ΜΕΡΟΣ ΠΡΩΤΟΝ}{ΒΑΣΙΚΑΙ ΔΙΑΤΑΞΕΙΣ}

\Section{Α'}{ΜΟΡΦΗ ΤΟΥ ΠΟΛΙΤΕΥΜΑΤΟΣ}
\Article{1}{}
\TOCArticle{1}{}
\begin{enumerate}
  \item[1.] Το πολίτευμα της Ελλάδος είναι Προεδρευομένη Κοινοβουλευτική Δημοκρατία.
  \item[2.] Θεμέλιον του πολιτεύματος είναι η λαϊκή κυριαρχία.
  \item[3.] Άπασαι αι εξουσίαι πηγάζουν εκ του Λαού και υπάρχουν υπέρ αυτού και του Έθνους, ασκούνται δε καθ'ον τρόπον ορίζει το Σύνταγμα.
\end{enumerate}

\Article{2}{}
\TOCArticle{2}{}
\begin{enumerate}
  \item[1.] Ο σεβασμός και η προστασία της αξίας του ανθρώπου αποτελούν την πρωταρχικήν υποχρέωσιν της Πολιτείας.
  \item[2.] Η Ελλάς ακολουθούσα τους γενικής αναγνωρίσεως κανόνας του διεθνούς δικαίου, επιδιώκει την εμπέδωσιν της ειρήνης, της δικαιοσύνης, ως και την ανάπτυξιν των φιλικών σχέσεων μεταξύ των λαών και των κρατών.
\end{enumerate}

\Section{B'}{ΣΧΕΣΕΙΣ ΕΚΚΛΗΣΙΑΣ ΚΑΙ ΠΟΛΙΤΕΙΑΣ}
\Article{3}{}
\TOCArticle{3}{}
\begin{enumerate}
  \item[1.] Επικρατούσα θρησκεία εν Ελλάδι είναι η της Ανατολικής Ορθοδόξου του Χριστού Εκκλησίας. Η Ορθόδοξος Εκκλησία της Ελλάδος, κεφαλήν γνωρίζουσα τον Κύριον ημών Ιησούν Χριστόν, υπάρχει αναποσπάστως ηνωμένη δογματικώς μετά της εν Κωνσταντινουπόλει Μεγάλης και πάσης άλλης ομοδόξου του Χριστού Εκκλησίας, τηρούσα απαρασαλεύτως, ως εκείναι, τους ιερούς αποστολικούς και συνοδικούς κανόνας και τας ιεράς παραδόσεις. Είναι αυτοκέφαλος και διοικείται υπό της Ιεράς Συνόδου, συγκροτουμένης ως ο Καταστατικός Χάρτης της Εκκλησίας ορίζει, τηρουμένων των διατάξεων του Πατριαρχικού Τόμου της κθ'(29) Ιουνίου του έτους 1850 και της συνοδικής Πράξεως της 4ης Σεπτεμβρίου 1928.
  \item[2.] Το υφιστάμενον εις ωρισμένας περιοχάς του Κράτους εκκλησιαστικόν καθεστώς δεν αντίκειται εις τα ς διατάξεις της προηγουμένης παραγράφου.
  \item[3.] Το κείμενον της Αγίας Γραφής τηρείται αναλλοίωτον. Η εις άλλον γλωσσικόν τύπον επίσημος μετάφρασις τούτου, άνευ εγκρίσεως της Αυτοκεφάλου Εκκλησίας της Ελλάδος και της εν Κωνσταντινουπόλει Μεγάλης του Χριστού Εκκλησίας, απαγορεύεται.
\end{enumerate}

\Part{ΜΕΡΟΣ ΔΕΥΤΕΡΟΝ}{ΑΤΟΜΙΚΑ ΚΑΙ ΚΟΙΝΩΝΙΚΑ ΔΙΚΑΙΩΜΑΤΑ}
\Article{4}{}
\TOCArticle{4}{}
\begin{enumerate}
  \item[1.] Οι Έλληνες είναι ίσοι ενώπιον του νόμου.
  \item[2.] Έλληνες και Ελληνίδες έχουν ίσα δικαιώματα και υποχρεώσεις.
  \item[3.] Έλληνες πολίται είναι ίσοι κέκτηνται τα υπό του νόμου οριζόμενα προσόντα. Αφαίρεσις της ελληνικής ιθαγενείας επιτρέπεται μόνον εις περίπτωσιν εκουσίας αποκτήσεως ετέρας ή αναλήψεως αντιθέτου προς τα εθνικά συμφέροντα υπηρεσίας εις ξένην χώραν, κατά τας υπό του νόμου ειδικώτερον προβλεπομένας προϋποθέσεις και διαδικασίαν.
  \item[4.] Μόνον Έλληνες πολίται είναι δεκτοί εις πάσας τας δημοσίας λειτουργίας, πλην των δι' ειδικών νόμων εισαγομένων εξαιρέσεων.
  \item[5.] Οι Έλληνες πολίται συνεισφέρουν αδιακρίτως εις τα δημόσια βάρη αναλόγως των δυνάμεών των.
  \item[6.] Πας Έλλην, δυνάμενος να φέρη όπλα, υποχρεούται να συντελή εις την άμυναν της Πατρίδος, κατά τους ορισμούς των νόμων.
  \item[7.] Τίτλοι ευγενείας ή διακρίσεως ούτε απονέμονται ούτε αναγνωρίζονται εις Έλληνας πολίτας.
\end{enumerate}

\Article{5}{}
\TOCArticle{5}{}
\begin{enumerate}
  \item[1.] Έκαστος δικαιούται να αναπτύσση ελευθέρως την προσωπικότητά του και να συμμετέχη εις την κοινωνικήν, οικονομικήν και πολιτικήν ζωήν της Χώρας, εφ' όσον δεν προσβάλλει τα δικαιώματα των άλλων και δεν παραβιάζει το Σύνταγμα ή τα χρηστά ήθη.
  \item[2.] Πάντες οι ευρισκόμενοι εντός της Ελληνικής Επικρατείας απολαύουν απολύτου προστασίας της ζωής, της τιμής και της ελευθερίας των, αδιακρίτως εθνικότητος, φυλής ή γλώσσης και θρησκευτικών ή πολιτικών πεποιθήσεων. Εξαιρέσεις επιτρέπονται εις τας περιπτώσεις τας προβλεπομένας υπό του διεθνούς δικαίου.
Απαγορεύεται η έκδοσις αλλοδαπού, δικαιωμένου δια την υπέρ της ελευθερίας δράσιν του.
  \item[3.] Η προσωπική ελευθερία είναι απαραβίαστος. Ουδείς καταδιώκεται, συλλαμβάνεται, φυλακίζεται ή άλλως περιορίζεται, ει μη όταν και όπως ο νόμος ορίζη.
  \item[4.] Ατομικά διοικητικά μέτρα, περιοριστικά της ελευθέρας κινήσεως ή εγκαταστάσεως εν τη Χώρα, ως και της ελευθέρας εξόδου και εισόδου εις αυτήν παντός Έλληνος, απαγορεύονται. Εις εξαιρετικάς περιπτώσεις ανάγκης και μόνον προς πρόληψιν αξιοποίνων πράξεων δύναται να επιβληθούν τοιαύτα μέτρα, μετ' απόφασιν ποινικού δικαστηρίου, ως νόμος ορίζει. Εις περίπτωσιν κατεπείγοντος, η απόφασις δύναται να εκδοθή και μετά την λήψιν του διοικητικού μέτρου, το βραδύτερον δε εντός τριών ημερών, άλλως αίρεται τούτο αυτοδικαίως.

\end{enumerate}

\textbf{Ερμηνευτική δήλωσις:}

Δεν περιλαμβάνεται εις την εν παραγράφω 4 απαγόρευσιν ή συνεπεία ποινικής διώξεως απαγόρευσις της εξόδου δια πράξεως του εισαγγελέως, ή η λήψις μέτρων επιβαλλομένων προς προστασίαν της δημοσίας υγείας ή της υγείας νοσούντων ατόμων, ως νόμος ορίζει.

\Article{6}{}
\TOCArticle{6}{}
\begin{enumerate}
  \item[1.] Ουδείς συλλαμβάνεται, ουδέ φυλακίζεται άνευ ητιολογημένου δικαστικού εντάλματος, το οποίον πρέπει να επιδοθή κατά την στιγμήν της συλλήψεως ή προφυλακίσεως. Εξαιρούνται τα επ' αυτοφόρω εγκλήματα.
  \item[2.] Ο επ' αυτοφόρω ή δι' εντάλματος συλληφθείς προσάγεται εις τον αρμόδιον ανακριτήν το βραδύτερον εντός είκοσι τεσσάρων ωρών από της συλλήψεως, εάν δε η σύλληψις εγένετο εκτός της έδρας του ανακριτού, εντός του απολύτως αναγκαίου προς μεταγωγήν χρόνου. Ο ανακριτής οφείλει, εντός τριών ημερών από της προσαγωγής, είτε να απολύση τον συλληφθέντα είτε να εκδώση κατ' αυτού ένταλμα φυλακίσεως. Αιτήσει του προσαχθέντος, ή εν περιπτώσει ανωτέρας βίας βεβαιουμένης αμέσως δι' αποφάσεως του αρμοδίου δικαστικού συμβουλίου, η προθεσμία αύτη παρατείνεται επί δύο ημέρας.
  \item[3.] Παρελθούσης απράκτου εκατέρας των προθεσμιών τούτων, πας δεσμοφύλαξ ή άλλος επιτετραμμένος την κράτησιν του συλληφθέντος, είτε πολιτικός υπάλληλος είτε στρατιωτικός, οφείλει ν' απολύση αυτόν παραχρήμα. Οι παραβάται τιμωρούνται επί παρανόμω κατακρατήσει, υποχρεούνται δε εις ανόρθωσιν πάσης ζημίας προσγενομένης εις τον παθόντα και εις ικανοποίησιν αυτού, λόγω ηθικής βλάβης, δια χρηματικού ποσού, ως νόμος ορίζει.
  \item[4.] Νόμος ορίζει το ανώτερον όριον διαρκείας της προφυλακίσεως, το οποίον δεν δύναται να υπερβή το έτος επί κακουργημάτων και τους εξ μήνας επί πλημμελημάτων. Επί όλως εξαιρετικών περιπτώσεων, δύναται τα ανώτατα όρια να παραταθούν κατά εξ και τρεις μήνας αντιστοίχως, δι' αποφάσεως του αρμοδίου δικαστικού συμβουλίου.
\end{enumerate}

\Article{7}{}
\TOCArticle{7}{}
\begin{enumerate}
  \item[1.] Έγκλημα δεν υπάρχει, ουδέ ποινή επιβάλλεται άνευ νόμου, ισχύοντος προ της τελέσεως της πράξεως και ορίζοντος τα στοιχεία ταύτης. Ουδέποτε επιβάλλεται βαρυτέρα ποινή της προβλεπομένης κατά την τέλεσιν πράξεως.
  \item[2.] Αι βάσανοι, οιαδήποτε σωματική κάκωσις, βλάβη υγείας ή άσκησις ψυχολογικής βίας, ως και πάσα ετέρα προσβολή της ανθρωπίνης αξιοπρεπείας απαγορεύονται και τιμωρούνται, ως νόμος ορίζει.
  \item[3.] Γενική δήμευσις απαγορεύεται. Θανατική ποινή επί πολιτικών εγκλημάτων, εκτός των συνθέτων, δεν επιβάλλεται.
  \item[4.] Νόμος ορίζει τους όρους υπό τους οποίους παρέχεται υπό του Κράτους, κατόπιν δικαστικής αποφάσεως, αποζημίωσις εις αδίκως ή παρανόμως καταδικασθέντας, προφυλακισθέντας ή άλλως πως στερηθέντας της προσωπικής των ελευθερίας.
\end{enumerate}

\Article{8}{}
\TOCArticle{8}{}
	Ουδείς αφαιρείται άκων του παρά του νόμου ωρισμένου εις αυτόν δικαστού.

	Δικαστικαί επιτροπαί και έκτακτα δικαστήρια, υφ'οιονδήποτε όνομα, δεν επιτρέπεται να συσταθούν.

\Article{9}{}
\TOCArticle{9}{}
\begin{enumerate}
  \item[1.] Η κατοικία εκάστου είναι άσυλον. Η ιδιωτική και οικογενειακή ζωή του ατόμου είναι απαραβίαστος. Ουδεμία κατ' οίκον έρευνα ενεργείται, ει μη όταν και όπως ο νόμος ορίζη, πάντοτε δε παρουσία εκπροσώπων της δικαστικής εξουσίας.
  \item[2.] Οι παραβάται της προηγουμένης διατάξεως τιμωρούνται επί παραβιάσει του οικιακού ασύλου και καταχρήσει εξουσίας, υποχρεούνται δε εις πλήρη αποζημίωσιν του παθόντος, ως νόμος ορίζει.
\end{enumerate}

\Article{10}{}
\TOCArticle{10}{}
\begin{enumerate}
  \item[1.] Έκαστος ή πολλοί ομού έχουν το δικαίωμα, τηρούντες τους νόμους του Κράτους, όπως αναφέρωνται εγγράφως προς τα αρχάς, υποχρεουμένας εις ταχείαν ενέργειαν επί τη βάσει των κειμένων διατάξεων και εις έγγραφον ητιολογημένην απάντησιν προς τον αναφερόμενον, κατά τας διατάξεις του νόμου.
  \item[2.] Μόνον μετά την κοινοποίησιν της τελικής αποφάσεως της αρχής, προς την οποίαν απευθύνεται η αναφορά και κατόπιν αδείας ταύτης, επιτρέπεται η δίωξις του υποβαλόντος την αναφοράν δια τυχόν εν αυτή υπαρχούσας παραβιάσεις.
  \item[3.] Αίτησις πληροφοριών υποχρεοί την αρμοδίαν αρχήν εις απάντησιν, εφ' όσον τούτο προβλέπεται υπό του νόμου.
\end{enumerate}

\Article{11}{}
\TOCArticle{11}{}
\begin{enumerate}
  \item[1.] Οι Έλληνες έχουν το δικαίωμα όπως συνέρχωνται ησύχως και αόπλως.
  \item[2.] Μόνον εις τα ς δημοσίας εν υπαίθρω συναθροίσεις δύναται να παρίσταται η αστυνομία. Αι εν υπαίθρω συναθροίσεις δύναται να απαγορευθούν δι' ητιολογημένης αποφάσεως της αστυνομικής αρχής, γενικώς μεν αν εκ τούτων επίκειται σοβαρός κίνδυνος εις την δημοσίαν ασφάλειαν, εις ωρισμένην δε περιοχήν αν απειλήται σοβαρά διαταραχή της κοινωνικοοικονομικής ζωής, ως νόμος ορίζει.
\end{enumerate}

\Article{12}{}
\TOCArticle{12}{}
\begin{enumerate}
  \item[1.] Οι Έλληνες έχουν το δικαίωμα όπως συνιστούν ενώσεις και σωματεία μη κερδοσκοπικού σκοπού, τηρούντες τους νόμους, οίτινες όμως ουδέποτε δύναται να υπαγάγουν την άσκησιν του δικαιώματος τούτου εις προηγουμένην άδειαν.
  \item[2.] Το σωματείον δεν δύναται να διαλυθή ένεκα παραβάσεως του νόμου ή ουσιώδους διατάξεως του καταστατικού του, ει μη μόνον δια δικαστικής αποφάσεως.
  \item[3.] Αι διατάξεις της προηγουμένης παραγράφου εφαρμόζονται αναλόγως και επί ενώσεων προσώπων μη συνιστωσών σωματείον.
  \item[4.] Δια νόμου δύναται να επιβληθούν περιορισμοί εις το δικαίωμα των δημοσίων υπαλλήλων όπως συνεταιρίζωνται. Περιορισμοί του δικαιώματος τούτου δύναται  να επιβληθούν και εις τους υπαλλήλους οργανισμών τοπικής αυτοδιοικήσεως  ή άλλων νομικών προσώπων δημοσίου δικαίου ή δημοσίων επιχειρήσεων.
  \item[5.] Οι πάσης φύσεως γεωργικοί και αστικοί συνεταιρισμοί αυτοδιοικούνται κατά τους όρους του νόμου και του καταστατικού των, τελούντες υπό την προστασίαν και εποπτείαν του Κράτους, υποχρεωμένου να μεριμνά δια την ανάπτυξιν αυτών.
  \item[6.] Επιτρέπεται η δια νόμου σύστασις αναγκαστικών συνεταιρισμών αποβλεπόντων εις εκπλήρωσιν σκοπών κοινής ωφελείας ή δημοσίου ενδιαφέροντος ή κοινής εκμεταλλεύσεως γεωργικών εκτάσεων ή άλλης πλουτοπαραγωγικής πηγής, εξασφαλιζομένης πάντως της ίσης μεταχειρίσεως των συμμετεχόντων.
\end{enumerate}

\Article{13}{}
\TOCArticle{13}{}
\begin{enumerate}
  \item[1.] Η ελευθερία της θρησκευτικής συνειδήσεως είναι απαραβίαστος. Η απόλαυσις των ατομικών και πολιτικών δικαιωμάτων δεν εξαρτάται εκ των θρησκευτικών εκάστου πεποιθήσεων.
  \item[2.] Πάσα γνωστή θρησκεία είναι ελευθέρα και τα της λατρείας αυτής τελούνται ακωλύτως υπό την προστασίαν των νόμων. Η άσκησις της λατρείας δεν επιτρέπεται να προσβάλη την δημοσίαν τάξιν ή τα χρηστά ήθη. Ο προσηλυτισμός απαγορεύεται.
  \item[3.] Οι λειτουργοί όλων των γνωστών θρησκειών υπόκεινται εις την αυτήν εποπτείαν της Πολιτείας και εις τας αυτάς έναντι ταύτης υποχρεώσεις, ως και οι της επικρατούσης θρησκείας.
  \item[4.] Ουδείς δύναται ένεκα των θρησκευτικών αυτού πεποιθήσεων να απαλλαγή της εκπληρώσεως των προς το Κράτος υποχρεώσεων ή να αρνηθή την συμμόρφωσίν του προς τους νόμους.
  \item[5.] Ουδείς όρκος επιβάλλεται άνευ νόμου ορίζοντος και τον τύπον αυτού.
\end{enumerate}

\Article{14}{}
\TOCArticle{14}{}
\begin{enumerate}
  \item[1.] Έκαστος δύναται να εκφράζη και να διαδίδη προφορικώς, εγγράφως και δια του τύπου τους στοχασμούς του, τηρών τους νόμους του Κράτους.
  \item[2.] Ο τύπος είναι ελεύθερος. Η λογοκρισία και παν άλλο προληπτικόν μέτρον απαγορεύεται.
  \item[3.] Η κατάσχεσις εφημερίδων και άλλων εντύπων, είτε προ της κυκλοφορίας  είτε μετ' αυτήν, απαγορεύεται. Κατ' εξαίρεσιν επιτρέπεται η κατάσχεσις, παραγγελία του εισαγγελέως, μετά την κυκλοφορίαν:
	\begin{enumerate}
		\item[α)] Ένεκα προσβολής της χριστιανικής και πάσης άλλης γνωστής θρησκείας.
		\item[β)] Ένεκα προσβολής του προσώπου του Προέδρου της Δημοκρατίας.
		\item[γ)] Ένεκα δημοσιεύματος, το οποίον αποκαλύπτει πληροφορίας περί την σύνθεσιν, εξοπλισμόν και διάταξιν των ενόπλων δυνάμεων ή την οχύρωσιν της Χώρας, ή το οποίον σκοπεί εις την βιαίαν ανατροπήν του πολιτεύματος ή στρέφεται κατά της εδαφικής ακεραιότητος του Κράτους.
		\item[δ)] Ένεκα ασέμνων δημοσιευμάτων προσβαλλόντων καταφανώς την δημοσίαν αιδώ, κατά τας υπό του νόμου οριζομένας περιπτώσεις.
	\end{enumerate}
  
  \item[4.] Εις πάσας τας περιπτώσεις της προηγουμένης παραγράφου ο εισαγγελεύς, εντός είκοσι τεσσάρων ωρών από της κατασχέσεων, οφείλει να υποβάλη την υπόθεσιν εις το δικαστικόν συμβούλιον και τούτο, εντός ετέρων είκοσι τεσσάρων ωρών, ν' αποφανθή περί διατηρήσεως ή άρσεως της κατασχέσεως, άλλως η κατάσχεσις αίρεται αυτοδικαίως. Τα ένδικα μέσα της εφέσεως και της αναιρέσεως επιτρέπονται εις τον εκδότην της κατασχεθείσης εφημερίδος ή άλλου εντύπου και εις τον εισαγγελέα.
  \item[5.] Νόμος ορίζει τον τρόπον της δια του τύπου πλήρους επανορθώσεως ανακριβών δημοσιευμάτων.
  \item[6.] Το δικαστήριον, μετά τρεις τουλάχιστον καταδίκας εντός πενταετίας δια παράβασιν των εν παραγράφω 3 προβλεπομένων εγκλημάτων, διατάσσει την οριστικήν ή προσωρινήν παύσιν της εκδόσεως του εντύπου και εις βαρείας περιπτώσεις την απαγόρευσιν της ασκήσεως του δημοσιογραφικού επαγγέλματος υπό του καταδικασθέντος, ως νόμος ορίζει. Η παύσις ή η απαγόρευσις άρχονται αφ' ης η καταδικαστική απόφασις καταστή αμετάκλητος.
  \item[7.] Τα  αδικήματα του τύπου είναι αυτόφωρα και εκδικάζονται ως νόμος ορίζει.
  \item[8.] Νόμος ορίζει τας προϋποθέσεις και τα προσόντα ασκήσεως του δημοσιογραφικού επαγγέλματος.
  \item[7.] Νόμος δύναται να προσδιορίση ότι τα μέσα χρηματοδοτήσεως εφημερίδων και περιοδικών δέν να καθίστανται γνωστά.
\end{enumerate}

\Article{15}{}
\TOCArticle{15}{}
\begin{enumerate}
  \item[1.] Αι προστατευτικαί δια τον τύπον διατάξεις του προηγουμένου άρθρου δεν εφαρμόζονται επί του κινηματογράφου, της φωνογραφίας, της ραδιοφωνίας, της τηλεοράσεως και παντός είδους άλλου παρεμφερούς μέσου μεταδόσεως λόγου ή παραστάσεως.
  \item[2.] Η ραδιοφωνία και η τηλεόρασις τελούν υπό τον άμεσον έλεγχον του Κράτους, σκοπούν δε εις την αντικειμενικήν και επί ίσοις όροις μετάδοσιν πληροφοριών και ειδήσεων, ως και προϊόντων του λόγου και της τέχνης, διασφαλιζομένης πάντως της εκ της κοινωνικής αποστολής αυτών και εκ της πολιτιστικής αναπτύξεως της Χώρας επιβαλλομένης ποιοτικής στάθμης των εκπομπών.
\end{enumerate}

\Article{16}{}
\TOCArticle{16}{}
\begin{enumerate}
  \item[1.] Η τέχνη και η επιστήμη, η έρευνα και η διδασκαλία είναι ελεύθεραι, η δε ανάπτυξις και προαγωγή αυτών αποτελεί υποχρέωσιν του Κράτους. Η ακαδημαϊκή ελευθερία και η ελευθερία της διδασκαλίας δεν απαλλάσσουν από του καθήκοντος της υπακοής εις το Σύνταγμα.
  \item[2.] Η παιδεία αποτελεί βασικήν αποστολήν του Κράτους, έχει δε ως σκοπόν την ηθικήν, πνευματικήν, επαγγελματικήν και φυσικήν αγωγήν των Ελλήνων, την ανάπτυξιν της εθνικής και θρησκευτικής συνειδήσεως και την διάπλασιν αυτών ως ελευθέρων και υπευθύνων πολιτών.
  \item[3.] Τα έτη υποχρεωτικής φοιτήσεως δεν δύναται να είναι ολιγώτερα των εννέα.
  \item[4.] Πάντες οι Έλληνες έχουν δικαίωμα δωρεάν παιδείας, καθ' όλας τας βαθμίδας αυτής, εις τα κρατικά εκπαιδευτήρια. Το Κράτος ενισχύει τους  διακρινομένους, ως και τους δεομένους αρωγής ή ειδικής προστασίας σπουδαστάς, αναλόγως προς τας ικανότητας αυτών.
  \item[5.] Η ανωτάτη εκπαίδευσις παρέχεται αποκλειστικώς υπό ιδρυμάτων αποτελούντων νομικά πρόσωπα δημοσίου δικαίου, πλήρως αυτοδιοικουμένων. Τα ιδρύματα ταύτα τελούν υπό την εποπτείαν του Κράτους και δικαιούνται της οικονομικής ενισχύσεως αυτού, λειτουργούν δε επί τη βάσει των περί οργανισμών αυτών νόμων. Συγχώνευσις ή κατάτμησις ανωτάτων εκπαιδευτικών ιδρυμάτων δύναται να χωρίση και κατά παρέκκλισιν πάση αντιθέτου διατάξεως, ως νόμος ορίζει.

Ειδικός νόμος ορίζει τα αφορώντα εις τους φοιτητικούς συλλόγους και την εις τούτους συμμετοχήν των σπουδαστών.
  \item[6.] Οι καθηγηταί των ανωτάτων εκπαιδευτικών ιδρυμάτων είναι δημόσιοι λειτουργοί. Το λοιπόν διδακτικόν προσωπικόν αυτών αποτελεί ωσαύτως δημόσιον λειτούργημα, υφ' ας προϋποθέσεις νόμος ορίζει. Τα της καταστάσεως πάντων τούτων καθορίζονται υπό των οργανισμών των οικείων ιδρυμάτων.
  
Οι καθηγηταί των ανωτάτων εκπαιδευτικών ιδρυμάτων δεν δύνανται να παυθούν προ της κατά νόμον λήξεως του χρόνου υπηρεσίας των ει μη μόνον υπό τας εν άρθρω 88 παράγραφος 4 ουσιαστικάς προϋποθέσεις και κατόπιν αποφάσεως συμβουλίου αποτελουμένου κατά πλειοψηφίαν εξ ανωτάτων δικαστικών λειτουργών ως νόμος ορίζει.
Νόμος ορίζει το όριον της ηλικίας των καθηγητών των ανωτάτων εκπαιδευτικών ιδρυμάτων, μέχρι δε της εκδόσεως τούτου, οι υπηρετούντες καθηγηταί αποχωρούν αυτοδικαίως επί τη λήξει του ακαδημαϊκού έτους, κατά το οποίον συμπληρούν το εξηκοστόν έβδομον έτος της ηλικίας των.
  \item[7.] Η επαγγελματική και πάσα άλλη ειδική εκπαίδευσις παρέχεται υπό του Κράτους και δια σχολών ανωτέρας βαθμίδος επί χρονικόν διάστημα ουχί μείζον της τριετίας, ως ειδικώτερον προβλέπεται υπό του νόμου, ορίζοντος και τα επαγγελματικά δικαιώματα των εκ των σχολών τούτων αποφοιτούντων.
  \item[8.] Νόμος ορίζει τας προϋποθέσεις και τους όρους χορηγήσεως αδείας προς ίδρυσιν και λειτουργίαν εκπαιδευτηρίων μη ανηκόντων εις το Κράτος, τα της υπ' αυτών ασκουμένης εποπτείας, ως και την υπηρεσιακήν κατάστασιν του διδακτικού προσωπικού αυτών.
Η σύστασις ανωτάτων σχολών υπό ιδιωτών απαγορεύεται.
  \item[9.] Ο αθλητισμός τελεί υπό την προστασίαν και την ανωτάτην εποπτείαν του Κράτους.
  
Το Κράτος επιχορηγεί και ελέγχει τας πάσης φύσεως ενώσεις των αθλητικών σωματείων, ως  νόμος ορίζει. Νόμος ορίζει επίσης την σύμφωνον  προς τον προορισμόν των επιχορηγουμένων ενώσεων διάθεσιν των παρεχομένων εκάστοτε ενισχύσεων.
\end{enumerate}

\Article{17}{}
\TOCArticle{17}{}
\begin{enumerate}
  \item[1.] Η ιδιοκτησία τελεί υπό την προστασίαν του Κράτους, τα εξ αυτής όμως δικαιώματα δεν δύναται να ασκώνται εις βάρος του γενικού συμφέροντος.
  \item[2.] Ουδείς στερείται της ιδιοκτησίας αυτού, ει μη δια δημοσίαν ωφέλειαν προσηκόντως αποδεδειγμένην, όταν και όπως ο νόμος ορίζη, πάντοτε δε προηγουμένης πλήρους αποζημιώσεως, ανταποκρινομένης προς την αξίαν του απαλλοτριουμένου κατά τον χρόνον της ενώπιον του δικαστηρίου συζητήσεως περί προσωρινού προσδιορισμού της αποζημιώσεως. Επί απ' ευθείας αιτήσεως περί οριστικού προσδιορισμού της αποζημιώσεως λαμβάνεται υπ' όψιν η αξία κατά τον χρόνον της περί τούτου συζητήσεως ενώπιον του δικαστηρίου.
  \item[3.] Η μετά την δημοσίευσιν της πράξεως απαλλοτριώσεως και ένεκα μόνον ταύτης ενδεχόμενη μεταβολή της αξίας του απαλλοτριουμένου δεν λαμβάνεται υπ' όψιν.
  \item[4.] Η αποζημίωσις ορίζεται πάντοτε υπό των πολιτικών δικαστηρίων. Αύτη δύναται και προσωρινώς να ορισθή δικαστικώς, μετ' ακρόασιν ή πρόσκλησιν του δικαιούχου, όστις δύναται να υποχρεωθή, κατά την κρίσιν του δικαστηρίου, δια την είσπραξιν αυτής, εις την παροχήν αναλόγου εγγυήσεως, καθ' όν τρόπον νόμος ορίζει.
Προ της καταβολής της οριστικής ή προσωρινώς ορισθείσης αποζημιώσεως διατηρούνται ακέραια πάντα τα δικαιώματα του ιδιοκτήτου, μη επιτρεπομένης της καταλήψεως.
Η ορισθείσα αποζημίωσις καταβάλλεται υποχρεωτικώς το βραδύτερον εντός ενός και ημίσεος έτους από της δημοσιεύσεως της αποφάσεως περί προσωρινού προσδιορισμού της αποζημιώσεως, επί απ' ευθείας δε αιτήσεως περί οριστικού προσδιορισμού της αποζημιώσεως, από της δημοσιεύσεως της περί τούτου αποφάσεως του δικαστηρίου, άλλως η απαλλοτρίωσις αίρεται αυτοδικαίως.
Η αποζημίωσις, ως τοιαύτη, εις ουδένα φόρον, κράτησιν ή τέλος υπόκειται.
  \item[5.] Νόμος ορίζει τας περιπτώσεις υποχρεωτικής ικανοποιήσεως των δικαιούχων δια την μέχρι του χρόνου καταβολής της αποζημιώσεως απολεσθείσαν πρόσοδον εκ του απαλλοτριωθέντος ακινήτου.
  \item[6.] Προκειμένης της εκτελέσεως έργων κοινής ωφελείας ή γενικωτέρας σημασίας δια την οικονομίαν της Χώρας, νόμος δύναται να επιτρέψη την υπέρ του δημοσίου απαλλοτρίωσιν ευρυτέρων ζωνών, πέραν των εκτάσεων των αναγκαιουσών δια την κατασκευήν των έργων. Ο αυτός νόμος καθορίζει τας προϋποθέσεις και τους όρους της τοιαύτης απαλλοτριώσεως, ως και τα της δια δημοσίους ή κοινωφελείς εν γένει σκοπούς διαθέσεως ή χρησιμοποιήσεως των, επί πλέον των αναγκαιουσών δια το υπό εκτέλεσιν έργον, απαλλοτριουμένων εκτάσεων.
  \item[7.] Νόμος δύναται να ορίση ότι δι' εκτέλεσιν έργων προφανούς κοινής ωφελείας υπέρ του δημοσίου, νομικών προσώπων δημοσίου δικαίου, οργανισμών τοπικής αυτοδιοικήσεως, οργανισμών κοινής ωφελείας και δημοσίων επιχειρήσεων, επιτρέπεται η εις το επιβαλλόμενον βάθος διάνοιξις υπογείων σηράγγων, άνευ αποζημιώσεως, υπό τον όρον ότι δεν θα παραβλάπτεται η συνήθης εκμετάλλευσις του υπερκειμένου ακινήτου.
\end{enumerate}

\Article{18}{}
\TOCArticle{18}{}
\begin{enumerate}
  \item[1.] Ειδικοί νόμοι ρυθμίζουν τα της ιδιοκτησίας και διαθέσεως μεταλλείων, ορυχείων, σπηλαίων, αρχαιολογικών χώρων και θησαυρών, ιαματικών, ρεόντων και υπογείων υδάτων και του υπογείου εν γένει πλούτου.
  \item[2.] Δια νόμου ρυθμίζονται τα της ιδιοκτησίας, εκμεταλλεύσεως και διαχειρίσεως των λιμνοθαλασσών και μεγάλων λιμνών, ως και τα της εν γένει διαθέσεως των εξ αποξηράνσεως τούτων προκυπτουσών εκτάσεων.
  \item[3.] Ειδικοί νόμοι ρυθμίζουν τα των επιτάξεων δια τας ανάγκας των ενόπλων δυνάμεων εις περίπτωσιν πολέμου ή επιστρατεύσεως, ή προς θεραπείαν αμέσου κοινωνικής ανάγκης, δυναμένης να θέση εις κίνδυνον την δημοσίαν τάξιν ή υγείαν.
  \item[4.] Επιτρέπεται, κατά την υπό ειδικού νόμου καθοριζομένην διαδικασίαν, ο αναδασμός αγροτικών εκτάσεων προς επωφελεστέραν  εκμετάλλευσιν του εδάφους, ως και η λήψις μέτρων προς αποφυγήν της υπερμέτρου κατατμήσεως ή προς διευκόλυνσιν της ανασυγκροτήσεως της κατατετμημένης μικράς αγροτικής ιδιοκτησίας.
  \item[5.] Εκτός των κατά τας προηγουμένας παραγράφους περιπτώσεων, δύναται δια νόμου να προβλεφθή και πάσα άλλη εξ ιδιαιτέρων περιστάσεων απαιτουμένη στέρησις της ελευθέρας χρήσεως και καρπώσεως της ιδιοκτησίας. Νόμος ορίζει τον υπόχρεον και την διαδικασίαν καταβολής εις τον δικαιούχον του ανταλλάγματος της χρήσεως ή καρπώσεως, το οποίον πρέπει να ανταποκρίνεται εις τας υφισταμένας εκάστοτε συνθήκας.
Επιβληθέντα μέτρα κατ' εφαρμογήν της παρούσης παραγράφου αίρονται ευθύς άμα εκλείψουν οι προκαλέσαντες ταύτα ιδιαίτεροι λόγοι. Επί αδικαιολογήτου παρατάσεως των μέτρων αποφασίζει περί άρσεως αυτών, κατά κατηγορίας περιπτώσεων, το Συμβούλιον της Επικρατείας, τη αιτήσει παντός έχοντος έννομον συμφέρον.
  \item[6.] Δια νόμου δύναται να ρυθμίζωνται τα της διαθέσεως εγκαταλελειμμένων εκτάσεων προς αξιοποίησιν αυτών υπέρ της εθνικής οικονομίας και αποκατάστασιν ακτημόνων. Δια του αυτού νόμου ορίζονται και τα της μερικής ή ολικής αποζημιώσεως των ιδιοκτητών εν περιπτώσει επανεμφανίσεώς των εντός ευλόγου προθεσμίας.
  \item[7.] Δια νόμου δύναται να καθιερωθή η αναγκαστική συνιδιοκτησία συνεχομένων ιδιοκτησιών αστικών περιοχών, εφ' όσον η αυτοτελής ανοικοδόμησις τούτων ή τινών εξ αυτών δεν ανταποκρίνεται προς τους εν τη περιοχή ταύτη ισχύοντας ή μέλλοντας α ισχύσουν όρους δομήσεως.
  \item[8.] Δεν υπόκεινται εις απολλοτρίωσιν η αγροτική ιδιοκτησία των Σταυροπηγιακών Ιερών Μονών της Αγίας Αναστασίας της Φαρμακολυτρίας εν Χαλκιδική, των Βλατάδων εν Θεσσαλονίκη και του Ευαγγελιστού Ιωάννου του Θεολόγου εν Πάτμω, εξαιρουμένων των μετοχίων. Ομοίως δεν υπόκεινται εις απαλλοτρίωσιν η εν Ελλάδι περιουσία των Πατριαρχείων Αλεξανδρείας, Αντιοχείας και Ιεροσολύμων, ως και της Ιεράς Μονής του Σινά.
\end{enumerate}

\Article{19}{}
\TOCArticle{19}{}
Το απόρρητον των επιστολών και της καθ' οιονδήποτε άλλον τρόπον ελευθέρας ανταποκρίσεως ή επικοινωνίας είναι απολύτως απαραβίαστον. Νόμος ορίζει τας εγγυήσεις υπό τας οποίας η δικαστική αρχή δεν δεσμεύεται εκ του απορρήτου δια λόγους εθνικής ασφαλείας ή προς διακρίβωσιν ιδιαιτέρως σοβαρών εγκλημάτων.

\Article{20}{}
\TOCArticle{20}{}
\begin{enumerate}
  \item[1.] Έκαστος δικαιούται εις παροχήν εννόμου προστασίας  υπό των δικαστηρίων και δύναται  να αναπτύξη ενώπιον τούτων τας απόψεις του περί των δικαιωμάτων ή συμφερόντων του,  ως ο νόμος ορίζει.
  \item[2.] Το δικαίωμα της προηγουμένης ακροάσεως του ενδιαφερομένου ισχύει και δια πάσαν διοικητικήν ενέργειαν ή μέτρον λαμβανόμενον εις βάρος των δικαιωμάτων ή συμφερόντων αυτού.
\end{enumerate}

\Article{21}{}
\TOCArticle{21}{}
\begin{enumerate}
  \item[1.] Η οικογένεια, ως θεμέλιον της συντηρήσεως  και προαγωγής του Έθνους, ως και ο γάμος, η μητρότης και η παιδική ηλικία τελούν υπό την προστασίαν του Κράτους.
  \item[2.] Πολύτεκνοι οικογένειαι, ανάπηροι πολέμου και ειρηνικής περιόδου, θύματα πολέμου, χήραι και ορφανά των εν πολέμω πεσούντων, ως και πάσχοντες εξ ανιάτου σωματικής ή πνευματικής νόσου δικαιούνται της ειδικής φροντίδας του Κράτους.
  \item[3.] Το Κράτος μεριμνά δια την υγείαν των πολιτών και λαμβάνει ειδικά μέτρα δια την προστασίαν της νεότητος, του γήρατος, της αναπηρίας και δια την περίθαλψιν των απόρων.
  \item[4.] Η απόκτησις κατοικίας υπό των στερουμένων ταύτης ή ανεπαρκώς στεγαζομένων αποτελεί αντικείμενον ειδικής φροντίδος του Κράτους.
\end{enumerate}

\Article{22}{}
\TOCArticle{22}{}
\begin{enumerate}
  \item[1.] Η εργασία αποτελεί δικαίωμα και τελεί υπό την προστασίαν του Κράτους. Μεριμνώντος δια την δημιουργίαν συνθηκών απασχολήσεως πάντων των πολιτών και δια την ηθικήν και υλικήν εξύψωσιν του εργαζομένου αγροτικού και αστικού πληθυσμού. Πάντες οι εργαζόμενοι, ανεξαρτήτως φύλλου ή άλλης διακρίσεως, δικαιούνται ίσης αμοιβής  δι' ίσης αξίας παρεχομένην εργασίαν.
  \item[2.] Δια νόμου καθορίζονται οι γενικοί όροι εργασίας, συμπληρούμενοι υπό των δι' ελευθέρων διαπραγματεύσεων συναπτομένων συλλογικών συμβάσεων εργασίας και, εν αποτυχία τούτων, υπό των δια διαιτησίας τιθεμένων κανόνων.
  \item[3.] Οιαδήποτε μορφή αναγκαστικής εργασίας απαγορεύεται. Ειδικοί νόμοι ρυθμίζουν τα της επιτάξεως προσωπικών υπηρεσιών εις περίπτωσιν πολέμου ή επιστρατεύσεως ή προς αντιμετώπισιν αναγκών της αμύνης της Χώρας ή επειγούσης κοινωνικής ανάγκης εκ θεομηνίας ή δυναμένης να θέση εις κίνδυνον την δημοσίαν υγείαν, ως και τα της προσφοράς προσωπικής εργασίας εις τους οργανισμούς τοπικής αυτοδιοικήσεως προς ικανοποίησιν τοπικών αναγκών.
  \item[4.] Το Κράτος μεριμνά δια την κοινωνικήν ασφάλισιν των εργαζομένων, ως ο νόμος ορίζει.
\end{enumerate}
\textbf{Ερμηνευτική δήλωσις:}

Εις τους γενικούς όρους εργασίας περιλαμβάνεται και ο προσδιορισμός του τρόπου και υποχρέου εισπράξεως και αποδόσεως προς τας συνδικαλιστικάς οργανώσεις της υπό των οικείων καταστατικών προβλεπομένης συνδρομής των μελών αυτών.

\Article{23}{}
\TOCArticle{23}{}
\begin{enumerate}
  \item[1.] Το Κράτος λαμβάνει τα προσήκοντα μέτρα προς διασφάλισιν της συνδικαλιστικής ελευθερίας και την ακώλυτον άσκησιν των συναφών προς ταύτην δικαιωμάτων κατά πάσης προσβολής τούτων εντός των ορίων του νόμου.
  \item[2.] Η απεργία αποτελεί δικαίωμα, ασκείται δε υπό των νομίμως συνεστημένων συνδικαλιστικών οργανώσεων προς διαφύλαξιν και προαγωγήν των οικονομικών και εργασιακών εν γένει συμφερόντων των εργαζομένων.
Απαγορεύεται η υφ' οιανδήποτε μορφήν απεργία εις τους δικαστικούς λειτουργούς και τους υπηρετούντας εις τα σώματα ασφαλείας. Το δικαίωμα προσφυγής εις απεργίαν τελεί υπό τους συγκεκριμένους περιορισμούς του ρυθμίζοντος τούτο νόμου προκειμένου περί των δημοσίων υπαλλήλων και των υπαλλήλων της τοπικής αυτοδιοικήσεως και των νομικών προσώπων δημοσίου δικαίου ως και του προσωπικού των πάσης μορφής επιχειρήσεων δημοσίου χαρακτήρος ή κοινής ωφελείας, η  λειτουργία των οποίων  έχει ζωτικήν σημασίαν δια την εξυπηρέτησιν βασικών αναγκών του κοινωνικού συνόλου. Οι περιορισμοί ούτοι δεν δύνανται να εξικνούνται μέχρι της καταργήσεως  του δικαιώματος της απεργίας  ή της παρακωλύσεως της νομίμου ασκήσεως αυτού.
\end{enumerate}

\Article{24}{}
\TOCArticle{24}{}
\begin{enumerate}
  \item[1.] Η προστασία του φυσικού και πολιτιστικού περιβάλλοντος αποτελεί υποχρέωσιν του Κράτους. Το Κράτος  υποχρεούται να λαμβάνη ιδιαίτερα προληπτικά ή κατασταλτικά μέτρα προς διαφύλαξιν αυτού. Νόμος καθορίζει τα αφορώντα εις την προστασίαν των δασών και των δασικών εν γένει εκτάσεων. Απαγορεύεται  η μεταβολή του προορισμού των δημοσίων δασών και των δημοσίων δασικών εκτάσεων, πλην αν προέχη δια την Εθνικήν Οικονομίαν η αγροτική  εκμετάλλευσις τούτων ή άλλη χρήσις εκ δημοσίου συμφέροντος επιβαλλομένη.
  \item[2.] Η χωροταξική αναδιάρθρωσις της Χώρας , η διαμόρφωσις, η ανάπτυξις, η πολεοδόμησις και η επέκτασις των πόλεων και των οικιστικών εν γένει περιοχών, τελεί υπό την ρυθμιστικήν αρμοδιότητα και τον έλεγχον του Κράτους, επί των τέλει της εξυπηρετήσεως της  λειτουργικότητος και αναπτύξεως των οικισμών και της εξασφαλίσεως των καλλιτέρων δυνατών όρων διαβιώσεως.
  \item[3.] Προς αναγνώρισιν περιοχής ως οικιστικής, ως και δια την πολεοδομικήν ενεργοποίησιν αυτής, αι περιλαμβανόμεναι εν αυτή ιδιοκτησίαι συμμετέχουν υποχρεωτικώς εις την διάθεσιν, ανευ αποζημιώσεως υπό του οικείου φορέως, των απαραιτήτων εκτάσεων προς δημιουργίαν οδών, πλατειών και χώρων εν γένει κοινωφελών χρήσεων και σκοπών, ως και εις τας δαπάνας προς εκτέλεσιν των βασικών κοινοχρήστων πολεοδομικών έργων, ως νόμος ορίζει.
  \item[4.] Δια νόμου  δύναται να προβλέπεται συμμετοχή των ιδιοκτητών περιοχής χαρακτηριζομένης ως οικιστικής, εις την επί βάσει εγκεκριμένου σχεδίου αξιοποίησιν και γενικήν διαρρύθμισιν ταύτης, δι αντιπαροχής ίσης αξίας ακινήτων ή τμημάτων κατ' όροφον ιδιοκτησίας, εκ των τελικώς καθοριζομένων ως οικοδομησίμων χώρων ή κτιρίων της περιοχής ταύτης.
  \item[5.] Οι διατάξεις των προηγουμένων παραγράφων έχουν εφαρμογήν και επί αναμορφώσεως υφισταμένων ήδη οικιστικών περιοχών. Αι εκ της αναμορφώσεως ελεύθεραι εκτάσεις διατίθενται προς δημιουργίαν κοινοχρήστων χώρων ή εκποιούνται προς κάλυψιν των δαπανών της πολεοδομικής αναμορφώσεως, ως νόμος ορίζει.
  \item[6.] Τα μνημεία και αι παραδοσιακαί περιοχαί και στοιχεία τελούν υπό την προστασίαν του Κράτους. Νόμος θέλει ορίσει τα αναγκαία προς πραγματοποίησιν της προστασίας ταύτης περιοριστικά της ιδιοκτησίας μέτρα, ως και τον τρόπον και το είδος της αποζημιώσεως των ιδιοκτητών.
\end{enumerate}

\Article{25}{}
\TOCArticle{25}{}
\begin{enumerate}
  \item[1.] Τα δικαιώματα του ανθρώπου, ως ατόμου και ως μέλους του κοινωνικού συνόλου, τελούν υπό την εγγύησιν του Κράτους, πάντων των οργάνων αυτού υποχρεωμένων να διασφαλίζουν την ακώλυτον άσκησιν αυτών.
  \item[2.] Η αναγνώρισις και προστασία των θεμελιωδών και απαραγράπτων δικαιωμάτων του ανθρώπου υπό της Πολιτείας αποβλέπει εις την πραγματοποίησιν της κοινωνικής προόδου εν ελευθερία και δικαιοσύνη.
  \item[3.] Η καταχρηστική άσκησις δικαιώματος δεν επιτρέπεται.
  \item[4.] Το Κράτος δικαιούται να αξιώνη παρ' όλων των πολιτών την εκπλήρωσιν του χρέους της κοινωνικής και εθνικής αλληλεγγύης.
\end{enumerate}

\Part{ΜΕΡΟΣ ΤΡΙΤΟΝ}{ΟΡΓΑΝΩΣΙΣ ΚΑΙ ΛΕΙΤΟΥΡΓΙΑΙ ΤΗΣ ΠΟΛΙΤΕΙΑΣ}

\Section{Α'}{ΣΥΝΤΑΞΙΣ ΤΗΣ ΠΟΛΙΤΕΙΑΣ}
\Article{26}{}
\TOCArticle{26}{}
\begin{enumerate}
  \item[1.] Η νομοθετική λειτουργία ασκείται υπό της Βουλής και του Προέδρου της Δημοκρατίας.
  \item[2.] Η εκτελεστική λειτουργία ασκείται υπό του Προέδρου της Δημοκρατίας και της Κυβερνήσεως.
  \item[3.] Η δικαστική λειτουργία ασκείται υπό των δικαστηρίων, αι αποφάσεις δε αυτών εκτελούνται εν ονόματι του Ελληνικού Λαού.
\end{enumerate}

\Article{27}{}
\TOCArticle{27}{}
\begin{enumerate}
  \item[1.] Ουδεμία μεταβολή των ορίων της Επικρατείας δύναται να γίνη άνευ νόμου ψηφιζομένου δια της απολύτου πλειοψηφίας, του όλου αριθμού των βουλευτών.
  \item[2.] Άνευ νόμου, ψηφιζομένου δια της απολύτου πλειοψηφίας του όλου αριθμού των βουλευτών, δεν είναι δεκτή εις την Ελληνικήν Επικράτειαν ξένη στρατιωτική δύναμις, ουδέ δύναται να διαμένη εν αυτή ή να διέλθη δι' αυτής.
\end{enumerate}

\Article{28}{}
\TOCArticle{28}{}
\begin{enumerate}
  \item[1.] Οι γενικώς παραδεδεγμένοι κανόνες του διεθνούς δικαίου, ως και αι διεθνείς συμβάσεις από της επικυρώσεως αυτών δια νόμου και της κατά τους όρους εκάστης τούτων θέσεως αυτών εν ισχύϊ αποτελούν αναπόσπαστον μέρος του εσωτερικού ελληνικού δικαίου, υπερισχύουν δε πάσης αντιθέτου διατάξεως νόμου. Η εφαρμογή των κανόνων του διεθνούς δικαίου και των διεθνών συμβάσεων έναντι των αλλοδαπών τελεί πάντοτε υπό τον όρον της αμοιβαιότητος.
  \item[2.] Προς εξυπηρέτησιν σπουδαίου εθνικού συμφέροντος και προαγωγήν της συνεργασίας μετ' άλλων κρατών είναι δυνατή  η δια συνθήκης ή συμφωνίας αναγνώρισις αρμοδιοτήτων κατά το Σύνταγμα εις όργανα διεθνών οργανισμών. Προς ψήφισιν του κυρούντος την συνθήκην ή συμφωνίαν νόμου απαιτείται πλειοψηφία των τριών πέμπτων του όλου αριθμού των βουλευτών.
  \item[3.] Η Ελλάς προέρχεται ελευθέρως, δια νόμου ψηφισομένου υπό της απολύτου πλειοψηφίας του όλου αριθμού των βουλευτών, εις περιορισμούς εις την άσκησιν της εθνικής κυριαρχίας, εφ' όσον τούτο υπαγορεύεται εκ σπουδαίου εθνικού συμφέροντος, δεν θίγει τα δικαιώματα του ανθρώπου και τας βάσεις του δημοκρατικού πολιτεύματος, γίνεται δε βάσει των αρχών της ισότητος και υπό τον όρον της αμοιβαιότητος.
\end{enumerate}

\Article{29}{}
\TOCArticle{29}{}
\begin{enumerate}
  \item[1.] Οι Έλληνες πολίται, έχοντες το δικαίωμα του εκλέγειν, δύνανται να ιδρύουν ελευθέρως και να μετέχουν εις πολιτικά κόμματα, η οργάνωσις και η δράσις των οποίων οφείλει να υπηρετή την ελευθέραν λειτουργίαν του δημοκρατικού πολιτεύματος.
Πολίται, μη αποκτήσαντες έτι το δικαίωμα του εκλέγειν, δύνανται να μετέχουν εις τα νέων των κομμάτων.
  \item[2.] Νόμος δύναται να ορίζη την οικονομικήν υπό του Κράτους ενίσχυσιν των κομμάτων και την δημοσιότητα των εκλογικών δαπανών αυτών και των υποψηφίων βουλευτών.
  \item[3.] Απαγορεύονται απολύτως  αι οιασδήποτε μορφής εκδηλώσεις υπέρ πολιτικών κομμάτων των δικαστικών λειτουργών, των στρατιωτικών εν γένει  και των οργάνων των σωμάτων ασφαλείας και των δημοσίων υπαλλήλων, ως και η ενεργός υπέρ κόμματος δράσις των υπαλλήλων νομικών προσώπων δημοσίου δικαίου, των δημοσίων επιχειρήσεων, ως και των οργανισμών τοπικής αυτοδιοικήσεως.
\end{enumerate}

\Section{Α'}{ΠΡΟΕΔΡΟΣ ΤΗΣ ΔΗΜΟΚΡΑΤΙΑΣ}
\Chapter{ΠΡΩΤΟΝ}{Ανάδειξις του Προέδρου}

\Article{30}{}
\TOCArticle{30}{}
\begin{enumerate}
  \item[1.] Ο Πρόεδρος της Δημοκρατίας είναι ρυθμιστής του Πολιτεύματος. Εκλέγεται υπό της Βουλής δια περίοδον πέντε ετών, κατά τα εν άρθρω 32 και 33 οριζόμενα.
  \item[2.] Το αξίωμα του Προέδρου είναι ασυμβίβαστον προς οιονδήποτε άλλο αξίωμα, θέσιν ή έργον.
  \item[3.] Η προεδρική περίοδος άρχεται από της ορκωμοσίας του Προέδρου.
  \item[4.] Εις περίπτωσιν πολέμου η προεδρική θητεία παρατείνεται μέχρι λήξεως αυτού.
  \item[5.] Επανεκλογή του αυτού προσώπου άπαξ μόνο επιτρέπεται.
\end{enumerate}

\Article{31}{}
\TOCArticle{31}{}
Πρόεδρος της Δημοκρατίας δύναται να εκλεγή η από πενταετίας τουλάχιστον και εκ πατρός την καταγωγήν Έλλην πολίτης, συμπληρώσας το τεσσαρακοστόν έτος της ηλικίας του και έχων την νόμιμον ικανότητα του εκλέγειν.

\Article{32}{}
\TOCArticle{32}{}
\begin{enumerate}
  \item[1.] Η εκλογή του Προέδρου της Δημοκρατίας υπό της Βουλής ενεργείται δια μυστικής ψηφοφορίας και εις ειδικήν προς τούτο συνεδρίασιν, προσκαλουμένης υπό του Προέδρου της Βουλής ένα τουλάχιστον μήνα προ της λήξεως της θητείας του εν ενεργεία Προέδρου της Δημοκρατίας κατά τα υπό του Κανονισμού αυτής οριζόμενα.

Εις περίπτωσιν οριστικής αδυναμίας του Προέδρου της Δημοκρατίας προς εκπλήρωσιν των καθηκόντων αυτού, κατά τα εν παραγράφω 2 του άρθρου 34 οριζόμενα, ως και εις περίπτωσιν παραιτήσεως, θανάτου ή εκπτώσεως αυτού, κατά τας διατάξεις του Συντάγματος, η προς εκλογήν νέου Προέδρου της Δημοκρατίας συνεδρίασις συγκαλείται εντός δέκα το βραδύτερον ημερών από της προώρου λήξεως της θητείας του προηγουμένου Προέδρου.
  \item[2.] Η εκλογή του Προέδρου της Δημοκρατίας ενεργείται εις πάσαν περίπτωσιν δια πλήρη θητείαν.
  \item[3.] Πρόεδρος της Δημοκρατίας εκλέγεται ο συγκεντρώσας την πλειοψηφίαν των δύο τρίτων του συνολικού αριθμού βουλευτών.

Εις περίπτωσιν μη συγκεντρώσεως της πλειοψηφίας ταύτης η ψηφοφορία επαναλαμβάνεται μετά πενθήμερον.

Εάν δεν επιτευχθή ουδέ κατά την δευτέραν ψηφοφορίαν η καθοριζομένη πλειοψηφία, η ψηφοφορία επαναλαμβάνεται άπαξ έτι μετά πενθήμερον, εκλέγεται δε Πρόεδρος της Δημοκρατίας ο συγκεντρώσας την πλειοψηφίαν των τριών πέμπτων του όλου αριθμού των βουλευτών.
  \item[4.] Μη επιτευχθείσης ουδέ κατά την τρίτην ψηφοφορίαν της μνησθείσης  ηυξημένης πλειοψηφίας, διαλύεται η Βουλή εντός δεκαημέρου από ταύτης, προκηρυσσομένης εκλογής προς ανάδειξιν νέας Βουλής. Το σχετικόν διάταγμα υπογράφεται υπό μόνου του εν ενεργεία Προέδρου της Δημοκρατίας και τούτου μη υπάρχοντος υπό του αναπληρούντος αυτόν Προέδρου της Βουλής.

Η εκ των νέων εκλογών αναδεικνυομένη Βουλή ευθύς μετά την συγκρότησιν αυτής εις σώμα, προβαίνει δια μυστικής ψηφοφορίας εις την εκλογήν Προέδρου της Δημοκρατίας δια της πλειοψηφίας των τριών πέμπτων του όλου αριθμού των βουλευτών.

Μη επιτευχθείσης της μνησθείσης πλειοψηφίας η ψηφοφορία επαναλαμβάνεται εντός πενθημέρου, εκλέγεται δε Πρόεδρος της Δημοκρατίας ο συγκεντρώσας την απόλυτον πλειοψηφίαν του όλου αριθμού των βουλευτών. Μη επιτευχθείσης ουδέ της πλειοψηφίας ταύτης, η πλειοψηφία επαναλαμβάνεται άπαξ έτι, μετά πενθήμερον και μεταξύ των δύο πλειοψειφούντων προσώπων, θεωρείται δε εκλεγείς ως Πρόεδρος της Δημοκρατίας ο σχετικώς πλειοψηφήσας.
  \item[5.] Εάν η Βουλή είναι απούσα συγκαλείται αύτη εκτάκτως προς εκλογήν του Προέδρου της Δημοκρατίας, κατά τα εν παραγράφω 4 οριζόμενα.
Εάν η Βουλή έχει διαλυθή καθ' οιονδήποτε τρόπον, η εκλογή του Προέδρου αναβάλλεται μέχρι συγκροτήσεως εις σώμα της Νέας Βουλής και εντός είκοσι το βραδύτερον ημερών από ταύτης, κατά τα εν παραγράφω 3 και 4 οριζόμενα, τηρουμένων και των ορισμών της παραγράφου 1 του άρθρου 34.
  \item[6.] Εφ' όσον η κατά τας προηγουμένας παραγράφους οριζομένη διαδικασία εκλογής του νέου Προέδρου δεν ήθελε περατωθή εγκαίρως, ο διατελών Πρόεδρος της Δημοκρατίας συνεχίζει την άσκησιν των καθηκόντων του και μετά την λήξιν της θητείας του, μέχρι αναδείξεως νέου Προέδρου.
\end{enumerate}
\textbf{Ερμηνευτική δήλωσις:}

“Πρόεδρος της Δημοκρατίας, παραιτούμενος  προ της λήξεως της θητείας του, δεν δύναται να μετάσχη εις την επακολουθούσαν, συνεπεία της παραιτήσεώς του, εκλογή”.

\Article{33}{}
\TOCArticle{33}{}
\begin{enumerate}
  \item[1.] Ο εκλεγόμενος Πρόεδρος της Δημοκρατίας αναλαμβάνει την άσκησιν των καθηκόντων αυτού από της επομένης της λήξεως της θητείας του αποχωρούντος Προέδρου, εις πάσας δε τας λοιπάς περιπτώσεις από της επομένης της εκλογής αυτού.
  \item[2.] Ο Πρόεδρος της δημοκρατίας δίδει ενώπιον της Βουλής τον ακόλουθον όρκον πριν ή αναλάβη την άσκησιν των καθηκόντων του:
“Ομνύω εις το όνομα της Αγίας και Ομοουσίου και Αδιαιρέτου Τριάδος να φυλάττω το Σύνταγμα και τους νόμους, να μεριμνώ δια την πιστήν τήρησιν αυτών, να υπερασπίζω την εθνικήν ανεξαρτησίαν και την ακεραιότητα της Χώρας, να προστατεύω τα δικαιώματα και τας ελευθερίας των Ελλήνων και να υπηρετώ το γενικόν συμφέρον και την πρόοδον του Ελληνικού Λαού”.
  \item[3.] Νόμος ορίζει την προς τον Πρόεδρον της Δημοκρατίας καταβλητέαν χορηγίαν και την λειτουργίαν των δια την εκτέλεσιν των καθηκόντων του οργανουμένων υπηρεσιών.
\end{enumerate}

\Article{34}{}
\TOCArticle{34}{}
\begin{enumerate}
  \item[1.] Τον Πρόεδρον της Δημοκρατίας αποδημούντα υπέρ τας δέκα ημέρας, εκλείποντα, παραιτούμενον, εκπεσόντα ή κωλυόμενον εξ οιουδήποτε λόγου να ασκήση τα καθήκοντα αυτού, αναπληροί προσωρινώς ο Πρόεδρος της Βουλής, μη υπαρχούσης δε ταύτης, ο Πρόεδρος της τελευταίας Βουλής και τούτου αρνουμένου ή μ η υπάρχοντος η Κυβέρνησις συλλογικώς.
Κατά την περίοδον της αναπληρώσεως του Προέδρου δεν έχουν εφαρμογήν αι διατάξεις περί διαλύσεως της Βουλής, εξαιρέσει της περιπτώσεως του άρθρου 32 παρ. 4, ως και αι διατάξεις περί παύσεως της Κυβερνήσεως και προσφυγής εις δημοψήφισμα κατά τας διατάξεις του άρθρου 37 παράγραφος 4 και του άρθρου 44 παράγραφος 2.
  \item[2.] Παρατεινομένης της αδυναμίας του Προέδρου της δημοκρατίας προς άσκησιν των καθηκόντων του πέραν των τριάκοντα ημερών , συγκαλείται υποχρεωτικώς η Βουλή και αν αύτη έχει διαλυθή, όπως αποφανθή δια της πλειοψηφίας των τριών πέμπτων του συνόλου των μελών αυτής, αν συντρέχη περίπτωσις  εκλογής νέου Προέδρου. Εις ουδεμίαν περίπτωσιν πάντως η εκλογή νέου Προέδρου της Δημοκρατίας δύναται να καθυστερήση πέραν των έ συνολικώς μηνών από της ενάρξεως της λόγω αδυναμίας αυτού αναπληρώσεώς του.
\end{enumerate}

\Chapter{Δεύτερον}{Εξουσίαι και ευθύνη εκ των πράξεων του Προέδρου}

\Article{35}{}
\TOCArticle{35}{}
\begin{enumerate}
  \item[1.] Ουδεμία πράξις του Προέδρου της Δημοκρατίας ισχύει, ουδέ εκτελείται, άνευ της προσυπογραφής του αρμοδίου Υπουργού, όστις δια μόνης της υπογραφής του καθίσταται υπεύθυνος, ως και της δημοσιεύσεως αυτής εις την Εφημερίδα της Κυβερνήσεως.

Εν περιπτώσει παύσεως της Κυβερνήσεως αν ο Πρόεδρος αυτής δεν προσυπογράψη το οικείον διάταγμα, προσυπογράφεται τούτο υπό του νέου Πρωθυπουργού.
  \item[2.] Κατ' εξαίρεσιν δεν χρήζουν προσυπογραφής αι ακόλουθοι αποκλειστικώς πράξεις:
  \begin{enumerate}
  	\item[α)] Ο διορισμός του Πρωθυπουργού.
  	\item[β)] Η σύγκλησις του Υπουργικού Συμβουλίου υπό την Προεδρίαν αυτού, κατά τα εν άρθρω 38 παράγραφος 3 οριζόμενα.
  	\item[γ)] Η σύγκλησις του Συμβουλίου της Δημοκρατίας.
  	\item[δ)] Η αναπομπή ψηφισθέντος υπό της Βουλής νομοσχεδίου ή προτάσεως νόμου κατά το άρθρον 42 παράγραφος 3.
  	\item[ε)]  Αι εν άρθροις 32 παράγραφος 4, 37 παράγραφος 3, 41 παράγραφοι 1 και 4 και 44 παράγραφος 2 αναφερόμεναι αρμοδιότητες.
  	\item[στ)] Διαγγέλματα εκδιδόμενα εις όλως εκτάκτους περιστάσεις κατά την παράγραφον 3 του άρθρου 44.
   	\item[ζ)] Ο διορισμός του προσωπικού των υπηρεσιών της Προεδρίας της Δημοκρατίας.
  \end{enumerate}
\end{enumerate}

\Article{36}{}
\TOCArticle{36}{}
\begin{enumerate}
  \item[1.] Ο Πρόεδρος της Δημοκρατίας, τηρουμένων οπωσδήποτε των ορισμών του άρθρου 35 παράγραφος 1, εκπροσωπεί διεθνώς το Κράτος, κηρύττει πόλεμον, συνομολογεί συνθήκας ειρήνης, συμμαχίας, οικονομικής συνεργασίας και συμμετοχής εις διεθνείς οργανισμούς ή ενώσεις,ανακοινώνει δε αυτάς εις την Βουλήν μετά των αναγκαίων διασαφήσεων, άμα το συμφέρον του Κράτους το επιτρέπουν.
  \item[2.] Αι περί εμπειρίας, ως και αι περί φορολογίας, οικονομικής συνεργασίας και συμμετοχής εις διεθνείς οργανισμούς ή ενώσεις, συνθήκαι και όσαι άλλαι περιέχουν παραχωρήσεις περί των οποίων κατ' άλλας διατάξεις του Συντάγματος  δεν δύναται να ορισθή τι άνευ νόμου ή επιβαρύνουν ατομικώς τους Έλληνας δεν έχουν ισχύν άνευ τυπικού νόμου κυρούντος ταύτας.
  \item[3.] Μυστικά άρθρα συνθήκης ουδέποτε δύναται να ανατρέψουν τα φανερά.
  \item[4.] Η κύρωσις διεθνών συνθηκών δεν δύναται ν' αποτελέση αντικείμενον νομοθετικής εξουσιοδοτήσεως κατά το άρθρον 43 παράγραφοι 2 και 4.
\end{enumerate}

\Article{37}{}
\TOCArticle{37}{}
\begin{enumerate}
  \item[1.] Ο Πρόεδρος της Δημοκρατίας διορίζει τον Πρωθυπουργόν, επί τη προτάσει δε αυτού διορίζει και παύει τα λοιπά μέλη της Κυβερνήσεως  και τους Υφυπουργούς.
  \item[2.] Πρωθυπουργός διορίζεται ο αρχηγός του διαθέτοντος εν τη Βουλή την απόλυτον πλειοψηφίαν των εδρών κόμματος. Αν το κόμμα δεν έχει αρχηγόν, ή αν ο αρχηγός αυτού δεν εξελέγη βουλευτής ή δεν υπάρχη εκπρόσωπος, ο διορισμός ενεργείται μετά την υπό της κοινοβουλευτικής ομάδος του κόμματος ανάδειξιν αρχηγού αυτής, πραγματοποιουμένην το βραδύτερον εντός πενθημέρου από της υπό του Προέδρου της Βουλής ανακοινώσεως εις τον Πρόεδρον της Δημοκρατίας της εν τη Βουλή δυνάμεως των κομμάτων.
  \item[3.] Εάν ουδέν κόμμα διαθέτη την απόλυτον πλειοψηφίαν των εν τη Βουλή εδρών,ο Πρόεδρος της Δημοκρατίας παρέχει εντολήν διερευνητικήν εις τον αρχηγόν  του σχετικώς πλειοψηφούντος κόμματος προς διακρίβωσιν της δυνατότητος σχηματισμού Κυβερνήσεως απολαυούσης της εμπιστοσύνης της Βουλής, κατά τα εν τη προηγουμένη παραγράφω οριζόμενα.
  \item[4.] Μη επιτυγχανομένου τούτου, ο Πρόεδρος της Δημοκρατίας δύναται να αναθέση νέαν διερευνητικήν  εντολήν εις τον αρχηγόν του δεύτερου εις κοινοβουλευτικήν δύναμιν κόμματος  ή να διορίση Πρωθυπουργόν, μετά γνώμην του Συμβουλίου Δημοκρατίας, μέλος ή μη της Βουλής δυνάμενον κατά την κρίσιν του  να τύχη ψήφου εμπιστοσύνης της Βουλής.

Εις τον ούτω προκριθέντα Πρωθυπουργόν ο Πρόεδρος της Δημοκρατίας δύναται να παράσχη και το δικαίωμα διαλύσεως της Βουλής προς διεξαγωγήν εκλογών.
\end{enumerate}

\Article{38}{}
\TOCArticle{38}{}
\begin{enumerate}
  \item[1.] Ο Πρόεδρος της Δημοκρατίας απαλλάσσει των καθηκόντων του τον Πρωθυπουργόν παραιτούμενον, ως και αν η Κυβέρνησις αποδοκιμασθή υπό της Βουλής, κατά τα εν άρθρω 84 οριζόμενα.

Εις τας περιπτώσεις ταύτας η εντολή σχηματισμού Κυβερνήσεως ανατίθεται εις μέλος της Βουλής υποχρεούμενον όπως ζητήση ψήφον εμπιστοσύνης κατά το άρθρον 84, ή εις μέλος  ή μη της Βουλής προς άμεσον διάλυσιν αυτής και ενέργειαν εκλογών.
  \item[2.] Ο Πρόεδρος της Δημοκρατίας δύναται να παύση την Κυβέρνησιν, αφού ακούση την γνώμην του Συμβουλίου δημοκρατίας εφαρμοζομένου περαιτέρω του β' εδαφίου της προηγουμένης παραγράφου.
  \item[3.] Ο Πρόεδρος της Δημοκρατίας εις εκτάκτους περιστάσεις δύναται να συγκαλή παρ'αυτώ το Υπουργικόν Συμβούλιον υπό την προεδρίαν του.
\end{enumerate}

\Article{39}{}
\TOCArticle{39}{}
\begin{enumerate}
  \item[1.] Ο Πρόεδρος της Δημοκρατίας, κατά τας υπό του Συντάγματος ειδικώς προβλεπομένας περιπτώσεις, συγκαλεί παρ' αυτώ και υπό την προεδρίαν του το Συμβούλιον Δημοκρατίας. Ομοίως συγκαλεί τούτο και εις πάσαν άλλην σοβαράν κατά την κρίσιν του εθνική περίστασιν.
  \item[2.] Το Συμβούλιον δημοκρατίας απαρτίζεται εκ των κατά δημοκρατικόν τρόπον εκλεγέντων και διατελεσάντων Προέδρων της Δημοκρατίας, του Πρωθυπουργού, του Προέδρου της Βουλής, του Αρχηγού της Αξιωματικής εν τη Βουλή Αντιπολιτεύσεως, ως και των προερχομένων εκ της Βουλής Πρωθυπουργών ή διατελεσάντων Πρωθυπουργών εις Κυβέρνησιν τυχούσαν της εμπιστοσύνης της Βουλής.
\end{enumerate}

\Article{40}{}
\TOCArticle{40}{}
\begin{enumerate}
  \item[1.] Ο Πρόεδρος της Δημοκρατίας συγκαλεί τακτικώς την Βουλήν άπαξ του έτους κατά τα εν άρθρω 64 παράγραφος 1 οριζόμενα και εκτάκτως οσάκις κρίνη τούτον εύλογον, κηρύσσει δε αυτοπροσώπως ή δια του Πρωθυπουργού την έναρξιν και την λήξιν εκάστης βουλευτικής περιόδου.
  \item[2.] Ο Πρόεδρος της Δημοκρατίας άπαξ μόνον δύναται να αναστείλη τας εργασίας της βουλευτικής συνόδου είτε αναβάλλων την έναρξιν είτε διακόπτων την εξακολούθησιν αυτών.
  \item[3.] Η αναστολή των εργασιών δεν επιτρέπεται να διαρκέση πλέον των τριάκοντα ημερών, ουδέ να επαναληφθή, κατά την αυτήν βουλευτικήν σύνοδον, άνευ της συγκαταθέσεως της Βουλής.
\end{enumerate}

\Article{41}{}
\TOCArticle{41}{}
\begin{enumerate}
  \item[1.] Ο Πρόεδρος της Δημοκρατίας δύναται να διαλύση την βουλήν μετά γνώμην του Συμβουλίου της Δημοκρατίας, εάν αύτη ευρίσκεται εν προφανεί δυσαρμονία προς το λαϊκόν αίσθημα ή εάν η σύνθεσις αυτής δεν εξασφαλίζη κυβερνητικήν σταθερότητα.
  \item[2.] Ο Πρόεδρος της δημοκρατίας δύναται να διαλύση την Βουλήν, προτάσει της τυχούσης ψήφου εμπιστοσύνης Κυβερνήσεως προς ανανέωσιν της λαϊκής εντολής, προκειμένου να αντιμετωπισθή εξαιρετικής σημασίας εθνικόν θέμα.
  \item[3.] Το περί διαλύσεως διάταγμα, προσυπογεγραμμένον εν τη περιπτώσει της προηγουμένης παραγράφου υπό του Υπουργικού συμβουλίου, πρέπει να διαλαμβάνη συγχρόνως την προκήρυξιν εκλογών εντός τριάκοντα ημερών και την σύγκλησιν της νέας Βουλής εντός τριάκοντα ημερώναπό τούτων.
  \item[4.] Βουλή εκλεγείσα μετά διάλυσιν της προηγουμένης δεν δύναται να διαλυθή προ της παρελεύσεως έτους από της ενάρξεως των εργασιών αυτής, εξαιρέσει της περιπτώσεως της καταψηφίσεως παρ' αυτής δύο Κυβερνήσεων. Προ της υπογραφής του διατάγματος ο Πρόεδρος της Δημοκρατίας ζητεί την γνώμην του Συμβουλίου της Δημοκρατίας. Αποκλείεται η διάλυσις της Βουλής δις δια την αυτήν αιτίαν.
  \item[5.] Η Βουλή διαλύεται υποχρεωτικώς εις την περίπτωσιν του άρθρου 32 παράγραφος 4.
\end{enumerate}

\Article{42}{}
\TOCArticle{42}{}
\begin{enumerate}
  \item[1.] Ο Πρόεδρος της Δημοκρατίας κυροί, εκδίδει και δημοσιεύει τους υπό της Βουλής ψηφισθέντας νόμους εντός μηνός από της επιψηφίσεως αυτών.
  \item[2.] Ο Πρόεδρος της Δημοκρατίας δύναται, εντός της κατά την προηγουμένην παράγραφον προθεσμίας, να αναπέμψη εις την Βουλήν ψηφισθέν υπ' αυτής νομοσχέδιον, εκθέτων και τους λόγους τη μη κυρώσεως.
  \item[3.] Πρότασις νόμου ή νομοσχέδιο αναπεμφθέν υπό του Προέδρου της Δημοκρατίας εις την Βουλήν εισάγεται εις την Ολομέλειαν αυτής, εάν δε επιψηφισθή και αύθις δια της απολύτου πλειοψηφίας του όλου αριθμού των βουλευτών, κατά την διαδικασίαν του άρθρου 76 παράγραφος 2, ο Πρόεδρος της Δημοκρατίας κυροί, εκδίδει και δημοσιεύει τούτο υποχρεωτικώς εντός δέκα ημερών από της δευτέρας επιψηφίσεως αυτού.
\end{enumerate}

\Article{43}{}
\TOCArticle{43}{}
\begin{enumerate}
  \item[1.] O Πρόεδρος της Δημοκρατίας εκδίδει τα αναγκαία διατάγματα προς εκτέλεσιν των νόμων, ουδέποτε δε δύναται να αναστείλη την εφαρμογήν των ουδέ να εξαιρέση τινα της εκτελέσεως αυτών.
  \item[2.] Επί τη προτάσει του αρμοδίου Υπουργού επιτρέπεται η έκδοσις κανονιστικών διαταγμάτων βάσει ειδικής εξουσιοδοτήσεως νόμου εντός των ορίων ταύτης. Εξουσιοδότησις προς έκδοσιν κανονιστικών πράξεων υπό ετέρων οργάνων της διοικήσεως επιτρέπεται προκειμένου περί ρυθμίσεως ειδικωτέρων θεμάτων ή θεμάτων τοπικού ενδιαφέροντος ή τεχνικού ή λεπτομερειακού χαρακτήρος.
  \item[3.] Ο Πρόεδρος της Δημοκρατίας εκδίδει οργανωτικά διατάγματα προς ρύθμισιν των εις την εσωτερικήν αποκλειστικώς διάρθρωσιν και λειτουργίαν των υπηρεσιών του Κράτους και των δημοσίων οργανισμών αναγομένων θεμάτων, αποκλειομένης της δι' αυτών αυξήσεως του αριθμού του προσωπικού, ως και της μεταβολής της βαθμολογικής διαρθρώσεως τούτου. Τα οργανωτικά ταύτα διατάγματα εκδίδονται μετά γνώμην ανωτάτου συμβουλίου αποτελουμένου  κατά τα δύο τρίτα τουλάχιστον των μελών του εκ δικαστικών λειτουργών, ως ο νόμος ορίζει.
  \item[4.] Δια νόμων ψηφιζομένων υπό της Ολομελείας της Βουλής δύναται να παρέχεται εξουσιοδότησις προς έκδοσιν κανονιστικών διαταγμάτων δια την ρύθμισιν των εις αυτούς εν γενικώ πλαισίω καθοριζομένων θεμάτων. Δια των νόμων τούτων χαράσσονται αι γενικαί αρχαί και κατευθύνσεις της ακολουθητέας ρυθμίσεως και τίθενται χρονικά όρια δια την χρήσιν της εξουσιοδοτήσεως.
  \item[5.] Τα κατά το άρθρον 72 παράγραφος 1 θέματα της αρμοδιότητος της Ολομελείας της Βουλής δεν δύναται να αποτελέσουν αντικείμενον της κατά την προηγουμένην παράγραφον εξουσιοδοτήσεως.
\end{enumerate}

\Article{44}{}
\TOCArticle{44}{}
\begin{enumerate}
  \item[1.] Εις εκτάκτους περιπτώσεις εξαιρετικώς επειγούσης και απροβλέπτου ανάγκης, ο Πρόεδρος της Δημοκρατίας, προτάσει του Υπουργικού Συμβουλίου, δύναται να εκδίδη πράξεις νομοθετικού περιεχομένου. Αύται υποβάλλονται εις την βουλήν προς κύρωσιν κατά τας διατάξεις του άρθρου 72 παράγραφος 1, εντός τεσσαράκοντα ημερών από της εκδόσεώς των ή εντός τεσσαράκοντα ημερών από της συγκλήσεως της Βουλής εις σύνοδον. Εάν δεν υποβληθούν εις την Βουλήν εντός των ανωτέρω προθεσμιών ή δεν εγκριθούν υπ' αυτής εντός τριών μηνών από της υποβολής των, αποβάλλουν εφ' εξής την ισχύν των.
  \item[2.] Ο Πρόεδρος της Δημοκρατίας δύναται δια διατάγματος να προκηρύσση την διεξαγωγήν δημοψηφίσματος επί κρισίμων εθνικών θεμάτων.
  \item[3.] Ο Πρόεδρος της Δημοκρατίας εις όλως εξαιρετικάς περιστάσεις απευθύνει διαγγέλματα, δημοσιευόμενα δια της Εφημερίδος της Κυβερνήσεως.
\end{enumerate}

\Article{45}{}
\TOCArticle{45}{}
Ο Πρόεδρος της Δημοκρατίας άρχει των ενόπλων Δυνάμεων της Χώρας, την διοίκησιν των οποίων ασκεί η Κυβέρνησις, ως νόμος ορίζει. Απονέμει δε τους βαθμούς εις τους υπηρετούντας εις αυτάς, ως νόμος ορίζει.

\Article{46}{}
\TOCArticle{46}{}
\begin{enumerate}
  \item[1.] Ο Πρόεδρος της Δημοκρατίας διορίζει και παύει κατά νόμον τους δημοσίους υπαλλήλους, εκτός των υπό του νόμου οριζομένων εξαιρέσεων.
  \item[2.] Ο Πρόεδρος της Δημοκρατίας απονέμει τα κεκανονισμένα παράσημα, κατά τας διατάξεις του περί αυτών νόμου.
\end{enumerate}

\Article{47}{}
\TOCArticle{47}{}
\begin{enumerate}
  \item[1.] Ο Πρόεδρος της Δημοκρατίας έχει το δικαίωμα όπως, προτάσει του Υπουργού Δικαιοσύνης και μετά γνώμην συμβουλίου συγκροτουμένου κατά πλειοψηφίαν εκ δικαστών, χαρίζη, μετατρέπη ή μετριάζη τας παρά των δικαστηρίων καταγιγνωσκομένας ποινάς, ως και να αίρη τας πάσης φύσεως κατά νόμον συνεπείας καταγνωσθεισών και εκτιθεισών ποινών.
  \item[2.] Ο Πρόεδρος της Δημοκρατίας μόνον τη  συγκατεθέσει της Βουλής έχει το δικαίωμα να απονέμη χάρις εις Υπουργόν καταδικασθέντα κατά το άρθρον 86.
  \item[3.] Αμνηστία παρέχεται μόνον επί πολιτικών εγκλημάτων δια προεδρικού διατάγματος εκδιδομένου προτάσει του Υπουργικού Συμβουλίου.
  \item[4.] Αμνηστία επί κοινών εγκλημάτων ουδέ δια νόμου παρέχεται.
\end{enumerate}

\Article{48}{}
\TOCArticle{48}{}
\begin{enumerate}
  \item[1.] Ο Πρόεδρος της Δημοκρατίας εν περιπτώσει πολέμου ή επιστρατεύσεως ένεκεν εξωτερικών κινδύνων, δια προεδρικού διατάγματος προσυπογραφομένου υπό του Υπουργικού Συμβουλίου, εν περιπτώσει δε σοβαράς διαταραχής ή εκδήλου απειλής κατά της δημοσίας τάξεως και ασφαλείας του κράτους εξ εσωτερικών κινδύνων, δια προεδρικού διατάγματος προσυπογραφομένου υπό του Πρωθυπουργού, δύναται να αναστείλη  καθ' άπασαν την Επικράτειαν ή τμήμα αυτής, την ισχύν των διατάξεων των άρθρων 5 παράγραφος 4, 6, 8, 9, 11 παράγραφοι 1 έως 4, 12, 14, 19, 22, 23, 96 παράγραφος 4 και 97 ή τινών εξ αυτών, να θέση εις εφαρμογήν τον εκάστοτε ισχύοντα νόμον περί καταστάσεως πολιορκίας και να συστήση εξαιρετικά δικαστήρια. Ο νόμος ούτος δεν δύναται να τροποποιηθή κατά την διάρκειαν  της εφαρμογής του.
  \item[2.] Ο Πρόεδρος της Δημοκρατίας δύναται, από της εκδόσεως του προεδρικού διατάγματος, υπό τας αυτάς προϋποθέσεις να λαμβάνη προσέτι  και πάντα τα αναγκαία νομοθετικής ή διοικητικής φύσεως μέτρα, προς αντιμετώπισιν της καταστάσεως και την ταχυτέραν αποκατάστασιν της λειτουργίας των συνταγματικών θεσμών.
  \item[3.] Η ισχύς του κατά την παράγραφον 1 εκδιδομένου προεδρικού διατάγματος, αν τούτο δεν ανακληθή δι' ομοίου διατάγματος ενωρίτερον, αίρεται αυτοδικαίως, εν περιπτώσει μεν πολέμου από της λήξεως αυτού, εν πάση δε άλλη περιπτώσει μετά τριάκοντα ημέρας από της δημοσιεύσεώς του,εκτός εάν η ισχύς αυτού παραταθή και πέραν των τριάκοντα ημερών δια προεδρικού διατάγματος εκδιδομένου μετά προηγουμένην άδειαν της Βουλής. Η σχετική απόφασις ταύτης λαμβάνεται δια της απολύτου πλειοψηφίας των παρόντων μελών αυτής, συμφώνως προς τας διατάξεις του άρθρου 67.
  \item[4.] Εάν η έκδοσις του κατά την παράγραφον 1 προεδρικού διατάγματος γίνη εν απουσία της Βουλής, αύτη συγκαλείται και αν έτι έληξεν η περίοδος αυτής ή διελύθη, μέχρι πέρατος της κατά την προηγουμένην παράγραφον προθεσμίας, ίνα αποφασίση περί της παρατάσεως της ισχύος του ως άνω διατάγματος.
  \item[5.] Από της δημοσιεύσεως του κατά την παράγραφον 1 προεδρικού διατάγματος και καθ' όλην την διάρκειαν της εφαρμογής του ισχύει αυτοδικαίως ή κατά το άρθρον 62 βουλευτική ασυλία και αν έτι διελύθη η βουλή ή έληξεν η περίοδος αυτής.
\end{enumerate}

\Chapter{ΤΡΙΤΟΝ}{Ειδικαί ευθύναι του Προέδρου της Δημοκρατίας}
\Article{49}{}
\TOCArticle{49}{}
\begin{enumerate}
  \item[1.] Ο Πρόεδρος της Δημοκρατίας δεν ευθύνεται οπωσδήποτε δια πράξεις ενεργηθείσας κατά την ενάσκησιν των καθηκόντων του, ει μη μόνον δι' εσχάτην προδοσίαν ή εκ προθέσεως παραβίασιν του Συντάγματος. Δια πράξεις μη σχετιζομένας  προς την άσκησιν των καθηκόντων του η δίωξις αναστέλλεται μέχρι λήξεως της προεδρικής θητείας.
  \item[2.] Η πρότασις περί κατηγορίας και παραπομπής εις δίκην του Προέδρου της Δημοκρατίας υποβάλλεται εις την Βουλήν υπογεγραμμένη υπό του ενός τρίτου τουλάχιστον των μελών αυτής, γίνεται δε αποδεκτή δι' αποφάσεως λαμβανομένης κατά πλειοψηφίαν των δύο τρίτων του συνόλου των μελών αυτής.
  \item[3.] Εάν η πρότασις γίνη αποδεκτή, ο Πρόεδρος της Δημοκρατίας παραπέμπεται εις το κατά το άρθρον 86 δικαστήριον, των περί τούτου διατάξεων εφαρμοζομένων αναλόγως και εν προκειμένω.
  \item[4.] Από της παραπομπής του ο Πρόεδρος της Δημοκρατίας απέχει της ασκήσεως των καθηκόντων του αναπληρούμενος κατά τα εν άρθρω 34 οριζόμενα, αναλαμβάνει δε εκ νέου ταύτα εφ' όσον δεν εξηντλήθη η θητεία του από της εκδόσεως απαλλακτικής αποφάσεως του υπό του άρθρου 86 δικαστηρίου.
  \item[5.] Νόμος ψηφιζόμενος υπό της Ολομελείας της Βουλής ρυθμίζει τα της εφαρμογής των διατάξεων του παρόντος άρθρου.
\end{enumerate}

\Article{50}{}
\TOCArticle{50}{}
Ο Πρόεδρος της Δημοκρατίας δεν έχει άλλας αρμοδιότητας, ει μη όσας απονέμει εις αυτόν ρητώς το Σύνταγμα και οι αυτό συνάδοντες νόμοι.

\Section{Γ'}{ΒΟΥΛΗ}
\Chapter{ΠΡΩΤΟΝ}{Ανάδειξις και συγκρότησις της Βουλής}
\Article{51}{}
\TOCArticle{51}{}
\begin{enumerate}
  \item[1.] Ο αριθμός των βουλευτών καθορίζεται δια νόμου, δεν δύναται όμως να είναι κατώτερος των διακοσίων, ουδέ ανώτερος των τριακοσίων.
  \item[2.] Οι βουλευταί αντιπροσωπεύουν το Έθνος.
  \item[3.] Οι βουλευταί εκλέγονται δι' αμέσου, καθολικής και μυστικής ψηφοφορίας υπό των εχόντων το δικαίωμα του εκλέγειν πολιτών, ως νόμος ορίζει. Ο νόμος δεν δύναται να περιορίση το δικαίωμα του εκλέγειν, ει μη μόνον λόγω μη συμπληρώσεως κατωτάτου ορίου ηλικίας ή λόγω ανικανίτητος προς δικαιοπραξίαν ή συνεπεία αμετακλήτου ποινικής καταδίκης δι' ωρισμένα εγκλήματα.
  \item[4.] Αι βουλευτικαί εκλογαί διενεργούνται ταυτοχρόνως καθ' άπασαν την Επικράτειαν.
Νόμος δύναται να ορίζη τα της ασκήσεως του εκλογικού δικαιώματος υπό των εκτός της Επικρατείας ευρισκομένων εκλογέων.
  \item[5.] Η άσκησις του εκλογικού δικαιώματος, είναι υποχρεωτική. Νόμος ορίζει εκάστοτε τας εξαιρέσεις και τας ποινικάς κυρώσεις.
\end{enumerate}

\Article{52}{}
\TOCArticle{52}{}
Η ελευθέρα και ανόθευτος εκδήλωσις της λαϊκής θελήσεως, ως έκφρασις της λαϊκής κυριαρχίας, τελεί υπό την εγγύησιν πάντων των λειτουργών της Πολιτείας, οίτινες υποχρεούνται να διασφαλίζουν ταύτην εις πάσαν περίπτωσιν. Δια νόμου ορίζονται αι ποινικαί κυρώσεις κατά των παραβατών της διατάξεως ταύτης.

\Article{53}{}
\TOCArticle{53}{}
\begin{enumerate}
  \item[1.] Οι βουλευταί εκλέγονται δια τέσσαρα συναπτά έτη, αρχόμενα από της ημέρας των γενικών εκλογών. Άμα τη λήξει της βουλευτικής περιόδου διατάσσεται, δια προεδρικού διατάγματος, προσυπογραφομένου υπό του Υπουργικού Συμβουλίου, η διενέργεια γενικών βουλευτικών εκλογών εντός τριάκοντα ημερών και η σύγκλησις της νέας Βουλής εις τακτικήν σύνοδον, εντός ετέρων τριάκοντα ημερών από τούτων.
  \item[2.] Βουλευτική έδρα, κενωθείσα κατά το τελευταίον έτος της περιόδου, δεν πληρούται δι' αναπληρωματικής εκλογής, όταν τοιαύτη απαιτήται κατά νόμον, εφ' όσον ο αριθμός των κενών εδρών δεν υπερβαίνη το πέμπτον του όλου αριθμού των βουλευτών.
  \item[3.] Εις περίπτωσιν πολέμου παρατείνεται καθ' όλην την διάρκειαν αυτού η βουλευτική περίοδος. Εάν έχη διαλυθή η Βουλή, αναστέλλεται η διενέργεια εκλογών μέχρι λήξεως του πολέμου, ανακαλουμένης μέχρι ταύτης αυτοδικαίως της διαλυθείσης Βουλής.
\end{enumerate}

\Article{54}{}
\TOCArticle{54}{}
\begin{enumerate}
  \item[1.] Το εκλογικόν σύστημα και αι εκλογικαί περιφέρειαι ορίζονται δια νόμου.
  \item[2.] Ο αριθμός των βουλευτών εκάστης εκλογικής περιφέρειας ορίζεται δια προεδρικού διατάγματος επί τη βάσει του νομίμου πληθυσμού της περιφέρειας, ως ούτος προκύπτει εκ της τελευταίας απογραφής.
  \item[3.] Μέρος της Βουλής, ουχί μείζον του ενός εικοστού του όλου αριθμού των βουλευτών, δύναται να εκλέγεται ενιαίως καθ' άπασαν την Επικράτειαν, εν συναρτήσει προς την συνολικήν εν τη Επικρατεία εκλογικήν δύναμιν εκάστου κόμματος, ως νόμος ορίζει.
\end{enumerate}

\Chapter{ΔΕΥΤΕΡΟΝ}{Κωλύματα και ασυμβίβαστα των Βουλευτών}
\Article{55}{}
\TOCArticle{55}{}
\begin{enumerate}
  \item[1.] Όπως εκλεγή τις βουλευτής απαιτείται να είναι Έλλην πολίτης, να έχη την νόμιμον ικανότητα του εκλέγειν και συμπεπληρωμένον το εικοστόν πέμπτον έτος της ηλικίας του κατά την ημέραν της εκλογής.
  \item[2.] Βουλευτής στερηθείς τινός των ανωτέρω προσόντων, εκπίπτει αυτοδικαίως του βουλευτικού αξιώματος.
\end{enumerate}

\Article{56}{}
\TOCArticle{56}{}
\begin{enumerate}
  \item[1.] Έμμισθοι δημόσιοι λειτουργοί και υπάλληλοι, αξιωματικοί των ενόπλων δυνάμεων και των σωμάτων ασφαλείας, υπάλληλοι οργανισμών τοπικής αυτοδιοικήσεως ή άλλων νομικών προσώπων δημοσίου δικαίου, δήμαρχοι και πρόεδροι κοινοτήτων και διοικηταί ή πρόεδροι διοικητικών συμβουλίων νομικών προσώπων δημοσίου δικαίου ή δημοσίων ή δημοτικών επιχειρήσεων, συμβολαιογράφοι, φύλακες μεταγραφών και υποθηκών, δεν δύνανται να ανακηρυχθούν υποψήφιοι, ουδέ να εκλεγούν βουλευταί, εάν δεν παραιτηθούν προ της ανακηρύξεώς των ως υποψηφίων. Η παραίτησις συντελείται δια μόνης της εγγράφου υποβολής αυτής. Αποκλείεται η επάνοδος των παραιτουμένων στρατιωτικών εις την ενεργόν υπηρεσίαν, απαγορεύεται δε η επάνοδος των παραιτουμένων πολιτικών υπαλλήλων και λειτουργών προ της παρελεύσεως έτους από της παραιτήσεως.
  \item[2.] Των περιορισμών της προηγουμένης παραγράφου εξαιρούνται οι καθηγηταί των ανωτάτων εκπαιδευτικών ιδρυμάτων. Νόμος ορίζει τον τρόπον της αναπληρώσεως αυτών, αναστελλομένης της ασκήσεως των προς την ιδιότητα του καθηγητού αρμοδιοτήτων του εκλεγέντος, διαρκούσης της βουλευτικής περιόδου.
  \item[3.] Έμμισθοι δημόσιοι υπάλληλοι, στρατιωτικοί εν ενεργεία και αξιωματικοί των σωμάτων ασφαλείας, υπάλληλοι νομικών εν γένει προσώπων δημοσίου δικαίου, διοικηταί και υπάλληλοι δημοσίων και δημοτικών επιχειρήσεων ή κοινωφελών ιδρυμάτων, δεν δύνανται να ανακηρυχθούν υποψήφιοι, ουδέ να εκλεγούν βουλευταί εις οιανδήποτε εκλογικήν περιφέρειαν, εις την οποίαν υπηρέτησαν πλέον του τριμήνου, κατά την προ των εκλογών τριετίαν. Εις τους αυτούς περιορισμούς υπάγονται και οι διατελέσαντες γενικοί γραμματείς υπουργείων κατά το τελευταίον εξάμηνον της τετραετούς βουλευτικής περιόδου. Δεν υπάγονται εις τους αυτούς περιορισμούς οι υποψήφιοι βουλευταί της Επικρατείας και οι κατώτεροι υπάλληλοι των κεντρικών κρατικών υπηρεσιών.
  \item[4.] Πολιτικοί υπάλληλοι και στρατιωτικοί εν γένει, έχοντες ανειλημμένην, κατά νόμον, υποχρέωσιν παραμονής εν τη υπηρεσία επί ωρισμένον χρόνον, δεν δύνανται να ανακηρυχθούν υποψήφιοι, ουδέ να εκλεγούν βουλευταί, διαρκούντος του χρόνου της υποχρεώσεως αυτών.
\end{enumerate}

\Article{57}{}
\TOCArticle{57}{}
\begin{enumerate}
  \item[1.] Τα καθήκοντα του βουλευτού είναι ασυμβίβαστα προς τα έργα ή την ιδιότητα μέλους του διοικητικού συμβουλίου, διοικητού, γενικού διευθυντού ή των αναπληρωτών αυτών, ή υπαλλήλου εμπορικής εταιρείας, ή επιχειρήσεως απολαβούσης ειδικών προνομίων ή κρατικής επιχορηγήσεως ή τυχούσης παραχωρήσεως δημοσίας επιχειρήσεως.
  \item[2.] Βουλευταί, εμπίπτοντες εις τας διατάξεις της προηγουμένης παραγράφου, οφείλουν εντός οκτώ ημερών, αφ' ης καταστή οριστική η εκλογή των να δηλώσουν επιλογήν μεταξύ του βουλευτικού αξιώματος και των ως άνω έργων. Εν παραλείψει τοιαύτης εμπροθέσμου δηλώσεως, εκπίπτουν αυτοδικαίως του αξιώματος του βουλευτού.
  \item[3.] Βουλευταί αποδεχόμενοι οιονδήποτε των εν τω προηγουμένω ή τω παρόντι άρθρω αναφερομένων καθηκόντων ή έργων, χαρακτηριζομένων ως αποτελούντων κώλυμα δια την υποψηφιότητα βουλευτού, ή ως ασυμβιβάστων δια το βουλευτικόν αξίωμα, εκπίπτουν αυτοδικαίως τούτου.
  \item[4.] Οι βουλευταί δεν δύνανται να αναλαμβάνουν προμηθείας, μελέτας, ή την εκτέλεσιν έργων του Κράτους, των οργανισμών τοπικής αυτοδιοικήσεως, ή άλλων νομικών προσώπων δημοσίου δικαίου ή δημοσίων, δημοτικών επιχειρήσεων και ενοικιάσεις δημοσίων ή δημοτικών φόρων, ουδέ να μισθώνουν ακίνητα ανήκοντα εις τα άνω αναφερόμενα πρόσωπα, ή να δέχωνται  πάσης μορφής παραχωρήσεις επί των ακινήτων αυτών. Η παράβασις των διατάξεων της παρούσης παραγράφου συνεπάγεται έκπτωσιν από του βουλευτικού αξιώματος και ακυρότητα των πράξεων. Αι πράξεις αύται είναι άκυροι και όταν γίνωνται υπό εμπορικών εταιρειών ή επιχειρήσεων, εις τας οποίας έργα διευθυντού ή διοικητικού ή νομικού συμβούλου εκτελεί βουλευτής, ή μετέχει τούτων ως ομόρρυθμος η ετερόρρυθμος εταίρος.
  \item[5.] Ειδικός νόμος ορίζει τον τρόπον συνεχίσεως ή εκχωρήσεως ή διαλύσεως συμβάσεων εκτελέσεως των έργων και μελετών περί ων η παράγραφος 4, ανειλημμένων υπό βουλευτού προ της εκλογής αυτού.
\end{enumerate}

\Article{58}{}
\TOCArticle{58}{}
Η εξέλεγξις και εκδίκασις των βουλευτικών εκλογών, κατά του κύρους των οποίων εγείρονται ενστάσεις, αναφερόμεναι είτε εις εκλογικάς παραβάσεις περί την ενέργειαν τούτων, είτε εις έλλειψιν των νομίμων προσόντων, ανατίθεται εις το κατ' άρθρον 100 Ανώτατον Ειδικόν Δικαστήριον. 

\Chapter{Τρίτον}{Καθήκοντα και δικαιώματα των Βουλευτών}
\Article{59}{}
\TOCArticle{59}{}
\begin{enumerate}
  \item[1.] Οι βουλευταί προ της αναλήψεως των καθηκόντων αυτών ομνύουν εν τω Βουλευτηρίω και εις δημοσίαν συνεδρίασιν τον ακόλουθον όρκον:

“Ομνύω εις το όνομα της Αγίας και Ομοουσίου και Αδιαιρέτου Τριάδος να φυλάττω πίστιν εις την Πατρίδα και εις το δημοκρατικόν πολίτευμα, υπακοήν εις το Σύνταγμα και τους νόμους και να εκπληρώ ευσυνειδήτως τα καθήκοντά μου”.
  \item[2.] Αλλόθρησκοι ή ετερόδοξοι βουλευταί δίδουν τον αυτόν όρκον κατά τον τύπον της ιδίας αυτών θρησκείας, ή του ιδίου αυτών δόγματος.
  \item[3.] Ανακηρυσσόμενοι βουλευταί απούσης της Βουλής δίδουν τον όρκον ενώπιον του λειτουργούντος Τμήματος αυτής.
\end{enumerate}

\Article{60}{}
\TOCArticle{60}{}
\begin{enumerate}
  \item[1.] Οι βουλευταί  έχουν απεριόριστον  το δικαίωμα της κατά συνείδησιν γνώμης και ψήφου.
  \item[2.] Η από του βουλευτικού αξιώματος παραίτησις είναι δικαίωμα του βουλευτού, συντελείται δε άμα τη υποβολή εγγράφου δηλώσεως εις τον Πρόεδρον της Βουλής και δεν υπόκειται εις ανάκλησιν.
\end{enumerate}

\Article{61}{}
\TOCArticle{61}{}
\begin{enumerate}
  \item[1.] Ο βουλευτής δεν καταδιώκεται, ουδ' οπωσδήποτε εξετάζεται, ένεκα γνώμης ή ψήφου δοθείσης παρ' αυτού κατά την άσκησιν των βουλευτικών καθηκόντων.
  \item[2.] Ο βουλευτής διώκεται μόνον δια συκοφαντικήν δυσφήμησιν κατά νόμον, μετ' άδειαν της Βουλής. Αρμόδιον δια την εκδίκασιν είναι το Εφετείον. Η άδεια θεωρείται ως οριστικώς μη παρασχεθείσα, εάν η Βουλή δεν αποφανθή εντός τεσσαράκοντα πέντε ημερών από της περιελεύσεως της εγκλήσεως εις τον Πρόεδρον της Βουλής. Εν αρνήσει χορηγήσεως της αδείας ή παρελθούσης απράκτου της προθεσμίας, η πράξις θεωρείται ανέγκλητος.

Η παράγραφος αύτη έχει εφαρμογήν από της προσεχούς βουλευτικής περιόδου.
  \item[3.] Ο βουλευτής δεν υποχρεούται εις μαρτυρίαν περί πληροφοριών περιελθουσών εις αυτόν ή παρασχεθεισών υπ' αυτού εν τη ασκήσει των καθηκόντων του, ούτε περί των προσώπων τα οποία ενεπιστεύθηκαν εις αυτόν τας πληροφορίας ή εις τα οποία ούτος παρέσχε ταύτας.
\end{enumerate}

\Article{62}{}
\TOCArticle{62}{}
Διαρκούσης της βουλευτικής περιόδου, βουλευτής δεν διώκεται, ουδέ συλλαμβάνεται, φυλακίζεται, ουδ' άλλως πως περιορίζεται, άνευ αδείας του Σώματος. Ομοίως, δεν διώκεται βουλευτής της διαλυθείσης Βουλής δια πολιτικά εγκλήματα, από της διαλύσεως ταύτης μέχρι της ανακηρύξεως των βουλευτών της νέας Βουλής.

Η άδεια θεωρείται μη παρασχεθείσα, εάν η Βουλή δεν αποφανθή εντός τριμήνου από της  διαβιβάσεως της αιτήσεως διώξεως του εισαγγελέως εις τον Πρόεδρον αυτής.

Η τρίμηνος προθεσμία αναστέλλεται κατά την διάρκειαν των διακοπών της Βουλής.

Άδεια δεν απαιτείται δια τα επ' αυτοφώρω κακουργήματα.

\Article{63}{}
\TOCArticle{63}{}
\begin{enumerate}
  \item[1.] Οι βουλευταί δικαιούνται εκ του δημοσίου αποζημιώσεως και δαπανών, δια την άσκησιν  του λειτουργήματός των, το ύψος δε αμφοτέρων δι' αποφάσεως της Ολομελείας της Βουλής.
  \item[2.] Οι βουλευταί απολαμβάνουν συγκινωνιακής, ταχυδρομικής και τηλεφωνικής ατελείας, η έκτασις της οποίας καθορίζεται δι' αποφάσεως της Ολομελείας της Βουλής.
  \item[3.] Εν αδικαιολογήτω απουσία βουλευτού επί πλείονας των πέντε συνεδριάσεων κατά μήνα, κρατείται υποχρεωτικώς το τριακοστόν της μηνιαίας αποζημιώσεως, δι εκάστην απουσίαν.
\end{enumerate}

\Chapter{Τέταρτον}{Οργάνωσις και λειτουργία της Βουλής}
\Article{64}{}
\TOCArticle{64}{}
\begin{enumerate}
  \item[1.] Η Βουλή συνέρχεται αυτοδικαίως κατ' έτος την πρώτην Δευτέραν του μηνός Οκτωβρίου εις τακτικήν σύνοδον δια τα ετήσια έργα αυτής, εκτός εάν ο Πρόεδρος της Δημοκρατίας την συγκαταλέση ενωρίτερον, συμφώνως τω άρθρω 40.
  \item[2.] Η διάρκεια της τακτικής περιόδου δεν δύναται να είναι βραχυτέρα των πέντε μηνών, μη συνυπολογιζομένου του χρόνου της κατά το άρθρον 40 αναστολής.

Η τακτική σύνοδος παρατείνεται υποχρεωτικώς μέχρι της εγκρίσεως του προϋπολογισμού, κατά το άρθρον 79, ή της ψηφίσεως του, κατά το αυτό άρθρον, ειδικού νόμου.
\end{enumerate}

\Article{65}{}
\TOCArticle{65}{}
\begin{enumerate}
  \item[1.] Η Βουλή ορίζει τον τρόπον της ελευθέρας λαι δημοκρατικής λειτουργίας αυτής δια Κανονισμού, ψηφιζομένου εν ολομελεία κατά το άρθρον 76 και δημοσιευομένου δια της  Εφημερίδος της Κυβερνήσεως, παραγγελία του Προέδρου αυτής.
  \item[2.] Η Βουλή εκλέγει εκ των μελών αυτής τον Πρόεδρον και τα λοιπά μέλη του Προεδρείου,κατά τα δια του Κανονισμού οριζόμενα.
  \item[3.] Ο Πρόεδρος και οι Αντιπρόεδροι εκλέγονται κατά την αρχήν εκάστης βουλευτικής περιόδου.

	Η διάταξις αύτη δεν εφαρμόζεται επί των κατά την διανυομένην πρώτην σύνοδον της Ε' Αναθεωρητικής Βουλής εκλεγέντων Προέδρου και Αντιπροέδρων.

	Η Βουλή προτάσει πεντήκοντα βουλευτών δύναται να εκφράση μομφήν εις βάρος του Προέδρου της Βουλής ή μέλους του Προεδρείου, συνεπαγομένην την λήξιν της θητείας αυτού.
  \item[4.] Ο Πρόεδρος της Βουλής διευθύνει τας εργασίας του Σώματος, μεριμνά δια την διασφάλισιν της ακωλύτου διεξαγωγής των εργασιών του, δια την κατοχύρωσιν της ελευθέρας γνώμης και εκφράσεως των βουλευτών και την τήρησιν της τάξεως , δυνάμενος να λάβη και πειθαρχικά μέτρα κατά παντός παρεκτρεπομένου βουλευτού, κατά τα δια του Κανονισμού της Βουλής οριζόμενα.
  \item[5.] Δια του Κανονισμού δύναται να συσταθή παρά τη Βουλή επιστημονική υπηρεσία προς υποβοήθησιν του νομοθετικού αυτής έργου.
  \item[6.] Ο Κανονισμός καθορίζει την οργάνωσιν των υπηρεσιών της Βουλής υπό την εποπτείαντου Προέδρου και πάντα τα αφορώντα το προσωπικόν αυτής. Αι πράξεις του Προέδρου, αι αφορώσαι εις την πρόσληψιν και την υπηρεσιακήν κατάστασιν του προσωπικού της Βουλής,υπόκεινται εις προσφυγήν ή αίτησιν ακυρώσεως ενώπιον του Συμβουλίου της Επικρατείας.
\end{enumerate}

\Article{66}{}
\TOCArticle{66}{}
\begin{enumerate}
  \item[1.] Η Βουλή συνεδριάζει δημοσία εν τω Βουλευτηρίω, δύναται όμως να διασκεφθή κεκλεισμένων των θυρών, κατ' αίτησιν της Κυβερνήσεως ή δέκα πέντε βουλευτών, εάν τούτο αποφασισθή  εν μυστική συνεδριάσει, κατά πλειοψηφίαν. Μετά ταύτα αποφασίζει εάν πρέπει να επαναληφθή η συζήτησις επί του αυτού θέματος εν δημοσία συνεδριάσει.
  \item[2.] Οι Υπουργοί και οι Υφυπουργοί έχουν ελευθέραν είσοδον εις τας συνεδριάσεις της Βουλής και ακούονται οσάκις ζητήσουν τον λόγον.
  \item[3.] Η Βουλή και αι κοινοβουλευτικαί επιτροπαί δύνανται να αιτήσωνται την παρουσίαν του αρμοδίου επί συζητουμένων υπ' αυτών θεμάτων Υπουργού ή Υφυπουργού.

	Αι κοινοβουλευτικαί επιτροπαί δικαιούνται να καλούν, δια του αρμοδίου Υπουργού, οιονδήποτε δημόσιον λειτουργόν θεωρούν χρήσιμον δια το έργον αυτών.
\end{enumerate}

\Article{67}{}
\TOCArticle{67}{}
Η Βουλή δεν δύναται ν' αποφασίση άνευ της απολύτου πλειοψηφίας των παρόντων μελών, ήτις όμως ουδέποτε δύναται να είναι μικροτέρα του ενός τετάρτου του όλου αριθμού των βουλευτών.

	Εν περιπτώσει ισοψηφίας επαναλαμβάνεται η ψηφοφορία, μετά νέαν δε ισοψηφίαν, η πρότασις απορρίπτεται.

\Article{68}{}
\TOCArticle{68}{}
\begin{enumerate}
  \item[1.] Η Βουλή συνιστά εις την αρχήν εκάστης τακτικής εισόδου κοινοβουλευτικάς επιτροπάς εκ των μελών αυτής προς επεξεργασίαν και εξέτασιν των υποβαλλομένων νομοσχεδίων και προτάσεων νόμων, των υπαγομένων εις την Ολομέλειαν και τα Τμήματα αυτής.
  \item[2.] Η Βουλή συνιστά εκ των μελών αυτής εξεταστικάς επιτροπάς, δι' αποφάσεώς της λαμβανομένης δια πλειοψηφίας των δύο πέμπτων του συνόλου των βουλευτών, προτάσει του ενός πέμπτου του όλου αριθμού των βουλευτών.

	Προς σύστασιν εξεταστικών επιτροπών επί ζητημάτων αναγομένων εις την εξωτερικήν πολιτικήν και την εθνικήν άμυναν, απαιτείται απόφασις της Βουλής λαμβανομένη δια της απολύτου πλειοψηφίας του όλου αριθμού των βουλευτών.

Τα της συγκροτήσεως και λειτουργίας των επιτροπών τούτων καθορίζονται υπό του Κανονισμού της Βουλής.
  \item[3.] Αι κοινοβουλευτικαί και εξεταστικαί επιτροπαί ως και τα κατά τα άρθρα 70 και 71 Τμήματα της Βουλής συνιστώνται κατ' αναλογίαν της δυνάμεως των κομμάτων, των ομάδων και των ανεξαρτήτων, ς ο Κανονισμός ορίζει.
\end{enumerate}

\Article{69}{}
\TOCArticle{69}{}
Ουδείς εμφανίζεται αυτόκλητος ενώπιον της Βουλής δια να αναφέρη  τι προφορικώς ή εγγράφως. Αι αναφοραί παρουσιάζονται δια τινος βουλευτού, ή παραδίδονται εις τον Πρόεδρον. Η Βουλή έχει  δικαίωμα ν' αποστέλλη τας απευθυνομένας προς αυτήν αναφοράς εις τους Υπουργός και Υφυπουργούς, οι οποίοι υποχρεούνται να δίδουν διευκρινίσεις, οσάκις ζητηθούν.

\Article{70}{}
\TOCArticle{70}{}
\begin{enumerate}
  \item[1.] Η Βουλή ασκεί το νομοθετικόν έργον αυτής εν Ολομελεία.
  \item[2.] Ο Κανονισμός της Βουλής προβλέπει την άσκησιν του δι' αυτού καθοριζομένου νομοθετικού έργου και εις Τμήματα, ουχί πλείονα των δύο, υπό τους εν άρθρω 72 περιορισμούς. Η σύστασις και η λειτουργία των Τμημάτων αποφασίζεται εκάστοτε υπό της Βουλής εις την αρχήν εκάστης συνόδου, Κανονισμού της Βουλής ορίζεται, επίσης, η μεταξύ των Τμημάτων κατανομή της αρμοδιότητος αυτών κατά Υπουργεία.
  \item[3.] Εφ' όσον δεν ορίζεται άλλως, αι περί Βουλής διατάξεις του Συντάγματος ισχύουν και δια την εν Ολομελεία και δια την κατά Τμήματα λειτουργίαν αυτής.
  \item[4.] Προς λήψιν αποφάσεως των Τμημάτων η απαιτουμένη πλειοψηφία  δεν δύναται να είναι κατωτέρα των δύο πέμπτων του αριθμού των βουλευτών των Τμημάτων.
  \item[5.] Ο κοινοβουλευτικός έλεγχος ασκείται υπό της Βουλής εν ολομελεία δις τουλάχιστον της Εβδομάδος, ως ο Κανονισμός της Βουλής ορίζει.
\end{enumerate}

\Article{71}{}
\TOCArticle{71}{}
Κατά την διάρκειαν της διακοπής των εργασιών της Βουλής, το νομοθετικόν έργον αυτής, εξαιρέσει των νομοθετημάτων αρμοδιότητος της Ολομελείας κατά τα εν άρθρω 72 οριζόμενα, επιτελείται υπό Τμήματος αυτής, συγκροτουμένου και λειτουργούντος κατά τα εν άρθροις68 παράγραφος 3 και 70 οριζόμενα.

	Δια του Κανονισμού δύναται να προβλεφθή η επεξεργασία των νομοσχεδίων ή προτάσεων νόμων υπό κοινοβουλευτικής επιτροπής εκ των μελών του αυτού Τμήματος.

\Article{72}{}
\TOCArticle{72}{}
\begin{enumerate}
  \item[1.] Εν Ολομελεία της Βουλής συζητούνται και ψηφίζονται ο Κανονισμός αυτής, νομοσχέδια και προτάσεις νόμων περί εκλογής βουλευτών,περί των εν άρθροις 3, 13, 27, 28 και 36 παράγραφος1 θεμάτων, περί της ασκήσεως και προστασίας των ατομικών δικαιωμάτων, περί της λειτουργίας των πολιτικών κομμάτων, περί παροχής νομοθετικής εξουσιοδοτήσεως κατά το άρθρον 43 παράγραφος 4, περί ευθύνης των υπουργών, περί καταστάσεως πολιορκίας, περί της χορηγίας του Προέδρου της Δημοκρατίας και περί της αυθεντικής ερμηνείας των νόμων κατά τα  εν άρθρω 77, ως και παντός ετέρου θέματος, αναγομένου εις την Ολομέλειαν της Βουλής, κατά ειδικήν πρόβλεψιν του Συντάγματος, ή δια την ρύθμισιν του οποίου απαιτείται ειδική πλειοψηφία.

	Εν Ολομελεία της Βουλής ψηφίζεται επίσης ο προϋπολογισμός και ο απολογισμός του Κράτους και της Βουλής.
  \item[2.] Η κατ' αρχήν, κατ' άρθρον και εις το σύνολον συζήτησις και ψήφισις πάντων των λοιπών νομοσχεδίων ή προτάσεων νόμων, δύναται να ανατεθή εις Τμήμα της Βουλής, κατά τα εν άρθρω 70 οριζόμενα.
  \item[3.] Το επιλαμβανόμενον της ψηφίσεως νομοσχεδίου ή προτάσεως νόμου Τμήμα αποφαίνεται οριστικώς περί της αρμοδιότητος αυτού, δικαιούμενον να παραπέμψη οιανδήποτε περί αυτής αμφισβήτησιν εις την Ολομέλειαν της Βουλής, δι' αποφάσεως λαμβανομένης δια της απολύτου πλειοψηφίας του όλου αριθμού των μελών αυτού. Η απόφασις της Ολομελείας δεσμεύει τα Τμήματα.
  \item[4.] Η Κυβέρνησις δύναται να εισάγη προς συζήτησιν και ψήφισιν νομοσχέδιον μείζονος σημασίας αντί των Τμημάτων εις την Ολομέλειαν.
  \item[5.] Η Ολομέλεια της Βουλής δύναται να ζητήσει δι' αποφάσεως  αυτής λαμβανομένης δι' απολύτου πλειοψηφίας του όλου αριθμού των βουλευτών, της παρ' αυτής συζήτησιν και κατ' αρχήν, κατ' άρθρον και εις το σύνολον ψήφισιν εκκρεμούς ενώπιον Τμήματος νομοσχεδίου ή προτάσεως νόμου.
\end{enumerate}

\Chapter{Πέμπτον}{Νομοθετική λειτουργία της Βουλής}
\Article{73}{}
\TOCArticle{73}{}
\begin{enumerate}
  \item[1.] Το δικαίωμα προτάσεως νόμων ανήκει εις την Βουλήν και την Κυβέρνησιν.
  \item[2.] Νομοσχέδια, αναφερόμενα οπωσδήποτε εις την απονομήν συντάξεως και τας προϋποθέσεις ταύτης, υποβάλλονται μόνον υπό του Υπουργού Οικονομικών, μετά γνωμοδότησιν του Ελεγκτικού Συνεδρίου, προκειμένου δε περί συντάξεων βαρυνουσών τον προϋπολογισμόν οργανισμών τοπικής αυτοδιοικήσεως ή άλλων νομικών προσώπων δημοσίου δικαίου, υπό του αρμοδίου Υπουργού και του Υπουργού των οικονομικών. Τα νομοσχέδια περί συντάξεων πρέπει να είναι ειδικά, μη επιτρεπομένης επί ποινή ακυρότητος της αναγραφής διατάξεων περί συντάξεων εις νόμους σκοπούντας την ρύθμισιν άλλων θεμάτων.
  \item[3.] Ουδεμία πρότασις νόμου ή τροπολογία ή προσθήκη εισάγεται προς συζήτησιν, εάν προέρχεται εκ της Βουλής, εφ' όσον συνεπάγεται εις βάρος του Δημοσίου, των οργανισμών τοπικής αυτοδιοικήσεως ή άλλων νομικών προσώπων δημοσίου δικαίου, δαπάνας ή ελάττωσιν εσόδων, ή της περιουσίας αυτών, προς μισθοδοσίαν ή σύνταξιν, ή εν γένει όφελος προσώπου.
  \item[4.] Είναι όμως παραδεκτή τροπολογία ή προσθήκη υποβαλλομένη υπό αρχηγού κόμματος ή εκπροσώπου ομάδος κατά τα εν παραγράφω 3 του άρθρου 74 οριζόμενα, προκειμένου περί νομοσχεδίων αναφερομένων εις την οργάνωσιν των δημοσίων υπηρεσιών και των οργανισμών δημοσίου ενδιαφέροντος, εις την υπηρεσιακήν εν γένει κατάστασιν των δημοσίων υπαλλήλων, των στρατιωτικών και των οργάνων των σωμάτων ασφαλείας, των υπαλλήλων οργανισμών τοπικής αυτοδιοικήσεως ή άλλων νομικών προσώπων δημοσίου δικαίου, ως και δημοσίων εν γένει επιχειρήσεων.
  \item[5.] Νομοσχέδιον δι' ου επιβάλλονται τοπικοί ή ειδικοί φόροι, η οιασδήποτε φύσεως βάρη υπέρ οργανισμών, ή νομικών προσώπων δημοσίου ή ιδιωτικού δικαίου, δέον να προσυπογράφεται και υπό των Υπουργών Συντονισμού και Οικονομικών.
\end{enumerate}

\Article{74}{}
\TOCArticle{74}{}
\begin{enumerate}
  \item[1.] Παν νομοσχέδιον και πάσα πρότασις νόμου συνοδεύεται υποχρεωτικώς υπό αιτιολογικής εκθέσεως, προ δε της εισαγωγής αυτού εις την Βουλήν, Ολομέλειαν ή Τμήματα, δύναται να παραπεμφθή προς νομοθετικήν επεξεργασίαν εις την κατά την παράγραφον 5 του άρθρου 65 επιστημονικήν υπηρεσίαν από της συστάσεως αυτής, ως ο Κανονισμός ορίζει.
  \item[2.] Τα εις την Βουλήν κατατιθεμένα νομοσχέδια και προτάσεις νόμων παραπέμπονται εις την οικείαν κοινοβουλευτικήν επιτροπήν. Υποβληθείσης της εκθέσεως ή παρελθούσης απράκτου της ταχθείσης προσθεσμίας προς υποβολήν ταύτης, εισάγονται εις την Βουλήν προς συζήτησιν μετά παρέλευσιν τριών έκτοτε ημερών, πλην αν έχουν χαρακτηρισθή υπό του αρμοδίου Υπουργού ως επείγοντος χαρακτήρος, Η συζήτησις άρχεται μετά προφορικήν εισήγησιν του αρμοδίου Υπουργού και των εισηγητών της επιτροπής.
  \item[3.] Τροπολογίαι βουλευτών επί νομοσχεδίων και προτάσεων νόμων της αρμοδιότητος της Ολομελείας ή των Τμημάτων της Βουλής δεν εισάγονται προς συζήτησιν,εάν δεν υποβληθούν μέχρι και της προτεραίας της ημέρας της ενάρξεως της συζητήσεως, εκτός εάν εις την συζήτησιν αυτών συγκατατίθεται και η Κυβέρνησις.
  \item[4.] Δεν εισάγεται προς συζήτησιν νομοσχέδιον ή πρότασις νόμου αποσκοπούσα εις την τροποποίησιν διατάξεως νόμου, εάν δεν έχη καταχωρισθή εις μεν την αιτιολογικήν έκθεσιν ολόκληρον το κείμενον της τροποποιουμένης διατάξεως, εις δε το κείμενον του νομοσχεδίου ή της προτάσεως ολόκληρος η νέα διάταξις, ως διαμορφούται δια της τροποποιήσεως.
  \item[5.] Νομοσχέδιον ή πρότασις νόμου, περιέχουσα διατάξεις ασχέτους προς το κύριον αντικείμενον αυτών, δεν εισάγονται προς συζήτησιν.

	Ουδεμία προσθήκη ή τροπολογία εισάγεται προς συζήτησιν, αν δεν σχετίζεται προς το κύριον αντικείμενον του νομοσχεδίου ή της προτάσεως.

	Εν αμφισβητήσει αποφαίνεται η Βουλή.
  \item[6.] Άπαξ του μηνός, εις προσδιοριστέαν υπό του Κανονισμού ημέραν,εγγράφονται εις την ημερησίαν διάταξιν κατά προτεραιότητα και συζητούνται εκκρεμείς προτάσεις νόμων.
\end{enumerate}

\Article{75}{}
\TOCArticle{75}{}
\begin{enumerate}
  \item[1.] Παν νομοσχέδιον και πάσα πρότασις νόμου,συνεπαγόμενα επιβάρυνσιν του προϋπολογισμού, εφ' όσον μεν υποβάλλεται υπό Υπουργών, δεν εισάγεται προς συζήτησιν, εάν δεν συνοδεύεται υπό εκθέσεως του Γενικού Λογιστηρίου του Κράτους, καθοριζούσης την δαπάνην, εφ'όσον δεν υποβάλλεται υπό βουλευτών, διαβιβάζεται προ πάσης συζητήσεως εις το Γενικόν Λογιστήριον του Κράτους, υποχρεούμενον να υποβάλη σχετικήν έκθεσιν εντός δέκα πέντε ημερών. Παρεχομένης απράκτου της προθεσμίας ταύτης, η πρότασις νόμου εισάγεται προς συζήτησιν και άνευ εκθέσεως.
  \item[2.] Το αυτό ισχύει και δια τας τροπολογίας, εφ' όσον ζητηθή τούτο υπό των αρμοδίων Υπουργών. Εις την περίπτωσιν ταύτην  το Γενικόν Λογιστήριον υποχρεούται να υποβάλη εις την Βουλήν την έκθεσίν του εντός τριών ημερών. Μόνον παρερχομένης απράκτου της προθεσμίας ταύτης, η συζήτησις χωρεί και άνευ εκθέσεως.
  \item[3.] Νομοσχέδιον, συνεπαγόμενον δαπάνην ή ελάττωσιν εσόδων, δεν εισάγεται προς συζήτησιν εάν δεν συνοδεύεται υπό ειδικής εκθέσεως περί του τρόπου καλύψεώς των, υπογεγραμμένης υπό του αρμοδίου Υπουργού και του Υπουργού Οικονομικών.
\end{enumerate}

\Article{76}{}
\TOCArticle{76}{}
\begin{enumerate}
  \item[1.] Παν νομοσχέδιον και πάσα πρότασις νόμου εισαγομένη ενώπιον της Ολομελείας μή των Τμημάτων συζητείται και ψηφίζεται εφ' άπαξ, κατ' αρχήν, κατ' άρθρον και εις το σύνολον.
  \item[2.] Κατ' εξαίρεσιν νομοσχέδια και προτάσεις νόμων συζητούνται και ψηφίζονται υπό της Ολομελείας της Βουλής δις και εις δυο διαφόρους συνεδριάσεις, απεχούσας αλλήλων δύο τουλάχιστον ημέρας, κατ'αρχήν μεν και κατ'άρθρον κατά την πρώτην συζήτησιν, κατ' άρθρον δε και εις το σύνολον κατά την δευτέραν, εάν ζητηθή τούτο μέχρι της ενάρξεως της κατ'αρχήν συζητήσεως, υπό του ενός τρίτου του όλου αριθμού των βουλευτών.
  \item[3.] Εάν κατά την συζήτησιν εγένοντο δεκταί τροπολογίαι, η ψήφισις του συνόλου αναβάλλεται επί εικοσιτετράωρον από της διανομής του τροποποιηθέντος νομοσχεδίου ή προτάσεως νόμου.
  \item[4.] Νομοσχέδιον ή πρότασις νόμου, χαρακτηριζόμενον υπό της Κυβερνήσεως ως κατεπείγον, εισάγεται προς ψήφισιν μετά περιωρισμένην συζήτησιν, εις ην μετέχουν, πλην των οικείων εισηγητών, ο Πρωθυπουργός ή ο αρμόδιος Υπουργός, οι Αρχηγοί των εν τη Βουλή κομμάτων και ανά εις εκπρόσωπος τούτων. Δια του Κανονισμού της Βουλής δύναται να περιορισθή η διάρκεια των ομιλιών και ο χρόνος συζητήσεως.
  \item[5.] Η Κυβέρνησις δύναται να ζητήση  όπως νομοσχέδιον ή πρότασις νόμου ιδιαιτέρας σημασίας ή επείγοντος χαρακτήρος, συζητηθεί εις ωρισμένον αριθμόν συνεδριάσεων, ουχί μείζονα των τριών. Η Βουλή δύναται να παρατείνη την συζήτησιν επί δύο εισέτι συνεδριάσεις, προτάσει του ενός δεκάτου του όλου αριθμού των βουλευτών. Δια του Κανονισμού της Βουλής καθορίζεται η διάρκεια εκάστης ομιλίας.
  \item[6.] Η επιψήφισις δικαστικών ή διοικητικών κωδίκων, συνταχθέντων υπό ειδικών επιτροπών,αι οποίαι συνεστήθηκαν δι' ειδικών νόμων, δύναται να γίνη εν Ολομελεία της Βουλής δι' ιδιαιτέρου νόμου, κυρούντος εν όλω τούτους.
  \item[7.] Κατά τον αυτόν τρόπον δύναται να γίνη κωδικοποίησις υφισταμένων διατάξεων δι' απλής ταξινομήσεως αυτών ή εν όλω επαναφορά καταργηθέντων νόμων, πλην των φορολογικών.
  \item[8.] Νομοσχέδιον ή πρότασις νόμου, αποκρουσθείσα υπό της Ολομελείας της Βουλής ή Τμήματος αυτής, δεν εισάγεται εκ νέου εις την αυτήν σύνοδον, ουδέ εις το μετά την λήξιν αυτής λειτουργούν Τμήμα.
\end{enumerate}

\Article{77}{}
\TOCArticle{77}{}
\begin{enumerate}
  \item[1.] Η αυθεντική ερμηνεία των νόμων ανήκει  εις την νομοθετικήν λειτουργίαν.
  \item[2.] Νόμος μη πράγματι ερμηνευτικός έχει ισχύν μόνον από της δημοσιεύσεως αυτού.
\end{enumerate}

\Chapter{Έκτον}{Φορολογία και δημοσιονομική διαχείρισις}
\Article{78}{}
\TOCArticle{78}{}
\begin{enumerate}
  \item[1.] Ουδείς φόρος επιβάλλεται ουδ' εισπράττεται άνευ τυπικού νόμου, καθορίζοντος το υποκείμενον της φορολογίας και το εισόδημα, το είδος της περιουσίας, τας δαπάνας και τας συναλλαγάς ή τας κατηγορίας τούτων, εις τας οποίας αναφέρεται ο φόρος.
  \item[2.] Φόρος ή άλλο οιονδήποτε οικονομικόν βάρος δεν δύναται να επιβληθή δια νόμου αναδρομικής ισχύος, εκτεινομένης πέραν του προηγουμένου της επιβολής του φόρου οικονομικού έτους.
  \item[3.] Εξαιρετικώς επί επιβολής ή αυξήσεως εισαγωγικού ή εξαγωγικού δασμού ή φόρου καταναλώσεως, επιτρέπεται η είσπραξις αυτών από της ημέρας της καταθέσεως εις την Βουλήν του περί αυτών νομοσχεδίου υπό τον όρον της δημοσιεύσεως του νόμου εντός της κατά το άρθρον 42 παράγραφος 1 προθεσμίας, πάντως δε το βραδύτερον εντός δέκα ημερών από της λήξεως της συνόδου.
  \item[4.] Το αντικείμενον της φορολογίας, ο φορολογικός συντελεστής, αι από της φορολογίας απαλλαγαί ή εξαιρέσεις και η απονομή συντάξεων δεν δύναται να αποτελέσουν αντικείμενο νομοθετικής εξουσιοδοτήσεως.

	Δεν αντίκειται εις την απαγόρευσιν ο καθορισμός δια νόμου  του τρόπου βεβαιώσεως της συμμετοχής του Κράτους και των δημοσίων εν γένει οργανισμών εις την αποκλειστικώς εκ των εκτελέσεως δημοσίων έργων προκαλουμένην αυτόματος υπερτίμησιν της παρακειμένης ιδιωτικής ακινήτου ιδιοκτησιας.
  \item[5.] Κατ'εξαίρεσιν επιτρέπεται η κατ' εξουσιοδότησιν νόμων πλαισίων επιβολή  εξισωτικών ή αντισταθμιστικών εισφορών ή δασμών, ως και η λήψις οικονομικών μέτρων, εν τω πλαισίω των διεθνών σχέσεων της Χώρας, προς οικονομικούς οργανισμούς ή μέτρων αποβλεπόντων εις την διασφάλισιν της συναλλαγματικής θέσεως της Χώρας.
\end{enumerate}

\Article{79}{}
\TOCArticle{79}{}
\begin{enumerate}
  \item[1.] Η βουλή ψηφίζει κατά την τακτικήν ετήσιαν σύνοδον αυτής τον προϋπολογισμόν των εσόδων και των εξόδων του Κράτους δια το επόμενον έτος.
  \item[2.] Πάντα τα έσοδα και έξοδα του Κράτους  πρέπει να αναγράφωνται εις τον ετήσιον προϋπολογισμόν και τον απολογισμόν.
  \item[3.] Ο προϋπολογισμός εισάγεται εις την Βουλήν δια του επί των Οικονομικών Υπουργού ένα τουλάχιστον μήνα προ της ενάρξεως του οικονομικού έτους, ψηφίζεται δε κατά τα υπό του Κανονισμού οριζόμενα, δια του οποίου και διασφαλίζεται το δικαίωμα εκφράσεως των αντιλήψεων όλων των εν τη Βουλή πολιτικών μερίδων.
  \item[4.] Εάν καθίσταται ανέφικτος δι' οιονδήποτε  λόγον η διοίκησις των εσόδων και εξόδων επί τη βάσει του προϋπολογισμού, ενεργείται αύτη βάσει ειδικού εκάστοτε νόμου.
  \item[5.] Εάν δεν καθίσταται δυνατή, λόγω λήξεως της περιόδου της Βουλής, η ψήφισις του προϋπολογισμού ή του κατά την προηγουμένην παράγραφον ειδικού νόμου, παρατείνεται επί τέσσαρας μήνας η ισχύς του προϋπολογισμού του λήξαντος  ή λήγοντος οικονομικού έτους, δια διατάγματος εκδιδομένου προτάσει του Υπουργικού Συμβουλίου.
  \item[6.] Δια νόμου δύναται να καθιερωθή η σύνταξις προϋπολογισμού διετούς χρήσεως.
  \item[7.] Εντός έτους το βραδύτερον  από της λήξεως του οικονομικού έτους κατατίθεται εις την Βουλήν ο απολογισμός, ως και ο γενικός ισολογισμός του Κράτους, οι οποίοι εξετάζονται υπό ειδικής επιτροπής βουλευτών και κυρούται υπό της Βουλής κατά τα υπό του Κανονισμού αυτής οριζόμενα.
  \item[7.] Τα προγράμματα οικονομικής και κοινωνικής αναπτύξεως εγκρίνεται υπό της Ολομελείας της Βουλής, ως ο νόμος ορίζει.  
\end{enumerate}

\Article{80}{}
\TOCArticle{80}{}
\begin{enumerate}
  \item[1.] Μισθός, σύνταξις, χορηγία ή αμοιβή ούτε εγγράφεται εις τον προϋπολογισμόν του Κράτους, ούτε παρέχεται άνευ οργανικού ή άλλου ειδικού νόμου.
  \item[2.] Νόμος ορίζει τα της κοπής ή εκδόσεως νομίσματος.
\end{enumerate}

\Section{Τέταρτον}{ΚΥΒΕΡΝΗΣΙΣ}
\Chapter{Πρώτον}{Συγκρότησις και αποστολή της Κυβερνήσεως}
\Article{81}{}
\TOCArticle{81}{}
\begin{enumerate}
  \item[1.] Την Κυβέρνησιν αποτελεί το Υπουργικόν Συμβούλιον, συγκείμενον εκ του Πρωθυπουργού και των Υπουργών. Δια νόμου ορίζονται τα της συνθέσεως και λειτουργίας  του Υπουργικού Συμβουλίου. Δια διατάγματος, προκαλουμένου υπό του Προέδρου της Κυβερνήσεως, δύναται να διορισθούν εις ή πλείονες εκ των Υπουργών Αντιπρόεδροι του Υπουργικού Συμβουλίου.

	Νόμος ρυθμίζει την θέσιν των αναπληρωτών και των άνευ χαρτοφυλακίου Υπουργών, των Υφυπουργών, οι οποίοι δύνανται να αποτελούν μέλη της Κυβερνήσεως, ως και των μονίμων υπηρεσιακών Υφυπουργών. 
  \item[2.] Ουδείς δύναται να διορισθή μέλος της Κυβερνήσεως, ή φυπουργός, εάν δεν συγκετρώνη  τα κατά το άρθρον 55 οριζόμενα δια τον Βουλευτήν προσόντα.
  \item[3.] Οιαδήποτε επαγγελματική δραστηριότης των μελών της Κυβερνήσεως, των Υφυπουργών και του Προέδρου της Βουλής, αναστέλλεται κατά την διάρκειαν της ασκήσεως των καθηκόντων των.
  \item[4.] Νόμος δύναται να καθιερώνη το ασυμβίβαστον του αξιώματος του Υπουργού και Υφυπουργού και προς έτερα έργα.
  \item[5.] Εν ελλείψει Αντιπροέδρου ο Πρωθυπουργός, οσάκις παρίσταται ανάγκη, ορίζει εκ των Υπουργών προσωρινόν αναπληρωτή του.
\end{enumerate}

\Article{82}{}
\TOCArticle{82}{}
\begin{enumerate}
  \item[1.] Η Κυβέρνησις καθορίζει και κατευθύνει την γενικήν πολιτικήν της Χώρας, συμφώνως προς τους ορισμούς του Συντάγματος και των νόμων.
  \item[2.] Ο Πρωθυπουργός εξασφαλίζει την ενότητα της Κυβερνήσεως και κατευθύνει τας ενεργείας αυτής και των δημοσίων εν γένει υπηρεσιών προς εφαρμογήν της κυβερνητικής πολιτικής, εντός του πλαισίου των νόμων.
\end{enumerate}

\Article{83}{}
\TOCArticle{83}{}
\begin{enumerate}
  \item[1.] Έκαστος των Υπουργών ασκεί τας υπό του νόμου οριζόμενας αρμοδιότητας. Οι άνευ χαρτοφυλακίου Υπουργοί ασκούν όσας αρμοδιότητας αναθέτει εις αυτούς ο Πρωθυπουργός δι' αποφάσεώς του.
  \item[2.] Οι Υφυπουργοί ασκούν τας δια κοινής αποφάσεως του Πρωθυπουργού και του οικείου Υπουργού ανατιθεμένας εις αυτούς αρμοδιότητας.
\end{enumerate}

\Chapter{Δεύτερον}{Σχέσεις Βουλής  και  Κυβερνήσεως}
\Article{84}{}
\TOCArticle{84}{}
\begin{enumerate}
  \item[1.] Η Κυβέρνησις οφείλει να απολαύη της εμπιστοσύνης της Βουλής. Εντός δεκαπενθημέρου από της ορκωμοσίας του Πρωθυπουργού η Κυβέρνησις υποχρεούται να ζητήσει ψήφον εμπιστοσύνης της Βουλής και δύναται να πράττη τούτο οποτεδήποτε άλλοτε.  Εάν κατά τον σχηματισμόν της Κυβερνήσεως έχουν διακοπή αι εργασίαι της Βουλής, καλείται αύτη εντός δέκα πέντε ημερών όπως αποφανθή επί της προτάσεως εμπιστοσύνης.
  \item[2.] Η Βουλή δύναται δι' αποφάσεώς της να αποσύρη την εμπιστοσύνην της από την Κυβέρνησιν ή μέλος αυτής. Πρότασις περί δυσπιστίας δεν δύναται να υποβληθή, ει μή  μετά πάροδον εξαμήνου από της παρά της Βουλής απορρίψεως προτάσεως δυσπιστίας.

	Η πρότασις δυσπιστίας πρέπει να είναι υπογεγραμμένη υπό του ενός έκτου τουλάχιστον βουλευτών και να περιλαμβάνη σαφώς τα θέματα επί των οποίων να διεξαχθή η συζήτησις.
  \item[3.] Κατ' εξαίρεσιν δύναται να υποβληθή πρότασις περί δυσπιστίας και προ της παρόδου εξαμήνου, εάν είναι υπογεγραμμένη υπό της πλειοψηφίας του όλου αριθμού των βουλευτών.
  \item[4.] Η συζήτησις επί προτάσεως εμπιστοσύνης ή δυσπιστίας άρχεται μετά δύο ημέρας από της υποβολής της σχετικής προτάσεως, πλην αν η Κυβέρνησις, επί προτάσεως δυσπιστίας, ζητήση την άμεσον έναρξιν, δεν δύναται να παραταθή πέραν των τριών ημερών από της ενάρξεως αυτής.
  \item[5.] Η επί προτάσεως εμπιστοσύνης ή δυσπιστίας ψηφοφορία διεξάγεται ευθύς μετά το πέρας της συζητήσεως, δύναται όμως να αναβληθή επί τεσσαράκοντα οκτώ ώρας, εάν ζητήσει τούτο η Κυβέρνησις.
  \item[6.] Πρότασις περί εμπιστοσύνης δεν δύναται να γίνη δεκτή, αν δεν γκριθή παρά της απολύτου πλειοψηφίας των παρόντων βουλευτών, η οποία όμως δεν επιτρέπεται να είναι κατωτέρα των δύο πέμπτων του όλου αριθμού αυτών. Πρότασις περί δυσπιστίας γίνεται δεκτή μόνον αν εγκριθή  παρά της απολύτου πλειοψηφίας του όλου αριθμού των βουλευτών.
  \item[7.] Κατά την ψηφοφορίαν επί των ανωτέρω προτάσεων ψηφίζουν οι Υπουργοί και οι Υφυπουργοί μέλη της Βουλής.
\end{enumerate}

\Article{85}{}
\TOCArticle{85}{}
Τα μέλη του Υπουργικού Συμβουλίου, ως και οι Υφυπουργοί είναι συλλογικώς υπεύθυνοι δια την γενικήν πολιτικήν της Κυβερνήσεως, έκαστος δε εξ αυτών δια τας πράξεις ή παραλείψεις της αρμοδιότητός του, κατά τας διατάξεις των περί ευθύνης Υπουργών νόμων. Ουδέποτε έγγραφος ή προφορική εντολή του Προέδρου της Δημοκρατίας απαλλάσσει τους Υπουργούς και Υφυπουργούς της ευθύνης των.

\Article{86}{}
\TOCArticle{86}{}
\begin{enumerate}
  \item[1.] Η Βουλή έχει το δικαίωμα να κατηγορή τους διατελούντας ή διατελέσαντας μέλη Κυβερνήσεως και τους Υφυπουργούς, κατά τους περί ευθύνης Υπουργών νόμους, ενώπιον του επί τούτω Δικαστηρίου, το οποίον, προεδρευόμενον υπό του Προέδρου του Αρείου Πάγου, συγκροτείται εκ δώδεκα δικαστών, κληρουμένων υπό του Προέδρου της Βουλής εν δημοσία συνεδριάσει εξ απάντων των προ της κατηγορίας διοριζομένων Αρεοπαγιτών και Προέδρων Εφετών, κατά τα υπό του νόμου οριζόμενα.
  \item[2.] Δίωξις, ανάκρισις ή προανάκρισις κατά των εν παραγράφω 1 προσώπων δια πράξεις ή παραλείψεις τελεσθείσας εν τη ασκήσει των  καθηκόντων των, δεν επιτρέπεται άνευ προηγουμένης περί τούτου αποφάσεως της Βουλής.

	Εάν κατά την διεξαγωγήν διοικητικής εξετάσεως προκύψουν στοιχεία δυνάμενα να θεμελιώσουν ευθύνην μέλους Κυβερνήσεως ή Υφυπουργού, κατά τας διατάξεις του νόμου περί ευθύνης Υπουργών, οι ενεργήσαντες αυτήν διαβιβάζουν ταύτα μετά το πέρας της διοικητικής εξετάσεως δια του αρμοδίου εισαγγελέως εις την Βουλήν.

	Μόνη η Βουλή έχει το δικαίωμα να αναστέλλη την ποινικήν δίωξιν.
  \item[3.] Μη περατωθείσης της διαδικασίας επί προτάσεως κατά του Υπουργού ή Υφυπουργού δι' οιονδήποτε λόγον, περιλαμβανομένου και του της παραγραφής, η Βουλή δύναται, αιτήσει του κατηγορηθέντος, δι' αποφάσεώς της να συστήση ειδικήν επιτροπήν εκ βουλευτών και ανωτάτων δικαστικών λειτουργών, προς έλεγχον της κατηγορίας, ως ο Κανονισμός ορίζει.
\end{enumerate}

\Section{Ε'}{ΔΙΚΑΣΤΙΚΗ  ΕΞΟΥΣΙΑ}
\Chapter{Πρώτον}{Δικαστικοί λειτουργοί και υπάλληλοι}
\Article{87}{}
\TOCArticle{87}{}
\begin{enumerate}
  \item[1.] Η δικαιοσύνη απονέμεται υπό δικαστηρίων συγκροτουμένων εκ τακτικών δικαστών απολαυόντων λειτουργικής και προσωπικής ανεξαρτησίας.
  \item[2.] Οι δικασταί κατά την άσκησιν των καθηκόντων των υπόκεινται μόνον εις το Σύνταγμα και τούς νόμους, εν ουδεμιά δε περιπτώσει υποχρεούνται να συμμορφούνται προς διατάξεις τιθεμένας κατά κατάλυσιν του Συντάγματος.
  \item[3.] Η επιθεώρησις των τακτικών δικαστών ενεργείται υπό ανωτέρων κατά βαθμόν δικαστών, ως και υπό του εισαγγελέως και των Αντιεισαγγελέων του Αρείου Πάγου, των δε εισαγγελέων υπό αρεοπαγιτών και ανωτέρων κατά βαθμόν εισαγγελέων, κατά τα υπό του νόμου οριζόμενα.
\end{enumerate}

\Article{88}{}
\TOCArticle{88}{}
\begin{enumerate}
  \item[1.] Οι δικαστικοί λειτουργοί διορίζονται δια προεδρικού διατάγματος, επί τη βάσει νόμου ορίζοντος τα προσόντα και την διαδικασίαν επιλογής των, είναι δε ισόβιοι.
  \item[2.] Αι αποδοχαί των δικαστικών λειτουργών είναι ανάλογοι προς το λειτούργημα αυτών. Τα της βαθμολογικής και μισθολογικής εξελίξεως, ως και τα της εν γένει καταστάσεως αυτών, καθορίζονται δι' ειδικών νόμων.
  \item[3.] Δια νόμου δύναται να προβλεφθή εκπαιδευτική και δοκιμαστική περίοδος των δικαστικών λειτουργών προ του, ως τακτικών, διορισμού των, διαρκείας μέχρι τριών ετών. Κατά την περίοδον ταύτην δύνανται ούτοι να ασκούν και καθήκοντα τακτικού δικαστού, ως νόμος ορίζει.
  \item[4.] Οι δικαστικοί λειτουργοί δύνανται να παυθούν μόνον κατόπιν δικαστικής αποφάσεως, ένεκα ποινικής καταδίκης ή ένεκα βαρέος πειθαρχικού παραπτώματος ή νόσου ή αναπηρίας ή υπηρεσιακής ανεπαρκείας, βεβαιουμένων καθ' όν τρόπον νόμος ορίζει και τηρουμένων των διατάξεων των παραγράφων 2 και 3 του άρθρου 93.
  \item[5.] Οι δικαστικοί λειτουργοί μέχρι και του βαθμού του εφέτου ή αντεισαγγελέως εφετών και τούτοις αντιστοίχων αποχωρούν υποχρεωτικώς της υπηρεσίας άμα τη συμπληρώσει του εξηκοστού πέμπτου έτους της ηλικίας των, πάντες δε οι επί ανωτέρω των ως άνω βαθμών, ή των τούτοις αντιστοίχων αποχωρούν υποχρεωτικώς της υπηρεσίας άμα τη συμπληρώσει του εξηκοστού εβδόμου έτους της ηλικίας των. Δια την εφαρμογήν της διατάξεως ταύτης εις πάσαν περίπτωσιν ως ημέρα συμπληρώσεως του ως άνω ορίου θεωρείται η 30η Ιουνίου του έτους της αποχωρήσεως του δικαστικού λειτουργού.
  \item[6.] Μετάταξις δικαστικών λειτουργών απαγορεύεται. Κατ' εξαίρεσιν επιτρέπεται μετάταξις τακτικών δικαστών προς πλήρωσιν θέσεων αντιεισαγγελέων του Αρείου Πάγου και μέχρι του ημίσεως αυτών, ως και μεταξύ παρέδρων παρά πρωτοδίκαις και παρέδρων παρ' εισαγγελίαις, τη αιτήσει των μετατασσομένων, ως νόμος ορίζει.
  \item[7.] Εις τα υπό του Συντάγματος ειδικώς προβλεπόμενα δικαστήρια ή συμβούλια, εις α μετέχουν μέλη του Συμβουλίου της Επικρατείας και του Αρείου Πάγου, προεδρεύει ο εκ των μελών τούτων αρχαιότερος εις τον βαθμόν.
\end{enumerate}

\textbf{Ερμηνευτική δήλωσις:}
Κατά την αληθή έννοιαν του άρθρου 88 επιτρέπεται ο διορισμός εις θέσεις παρέδρων και συμβούλων του Ελεγκτικού Συνεδρίου κατά τα δια νόμου οριζόμενα.


\Article{89}{}
\TOCArticle{89}{}
\begin{enumerate}
  \item[1.] Απαγορεύεται εις τους δικαστικούς λειτουργούς η παροχή πάσης άλλης εμμίσθου υπηρεσίας, ως και η άσκησις οιουδήποτε  επαγγέλματος.
  \item[2.] Κατ' εξαίρεσιν επιτρέπεται η εκλογή δικαστικών λειτουργών ως μελών της Ακαδημίας ή ως καθηγητών ή υφηγητών ανωτάτων σχολών, ως και η συμμετοχή αυτών εις ειδικά διοικητικά δικαστήρια και εις συμβούλια ή επιτροπάς, πλην των διοικητικών συμβουλίων επιχειρήσεων και εμπορικών εταιρειών.
  \item[3.] Επίσης επιτρέπεται η ανάθεσις εις δικαστικούς λειτουργούς διοικητικών καθηκόντων, είτε παραλλήλως προς την άσκησιν των κυρίων αυτών καθηκόντων είτε αποκλειστικώς επί ωρισμένον
 χρονικόν διάστημα, ως νόμος ορίζει.
  \item[4.] Απαγορεύεται εις τους δικαστικούς λειτουργούς η συμμετοχή εις την Κυβέρνησιν.
  \item[5.] Επιτρέπεται η συγκρότησις ενώσεως δικαστικών λειτουργών, ως νόμος ορίζει.
\end{enumerate}

\Article{90}{}
\TOCArticle{90}{}
\begin{enumerate}
  \item[1.] Αι προαγωγαί, τοποθετήσεις, μεταθέσεις, αποσπάσεις και μετατάξεις των δικαστικών λειτουργών ενεργούνται δια προεδρικού διατάγματος, εκδιδομένου μετά προηγουμένην απόφασιν ανωτάτου δικαστικού Συμβουλίου. Τούτο συγκροτείται εκ του προέδρου του οικείου ανωτάτου δικαστηρίου και μελών του αυτού δικαστηρίου οριζομένων δια κληρώσεως μεταξύ των εχόντων διετή τουλάχιστον υπηρεσίαν παρ' αυτώ, ως νόμος ορίζει. Του ανωτάτου δικαστικού συμβουλίου της πολιτικής και ποινικής δικαιοσύνης μετέχει και ο Εισαγγελέας του Αρείου Πάγου του δε Ελεγκτικού Συνεδρίου ο παρ' αυτώ Γενικός Επίτροπος της Επικρατείας.
  \item[2.] Δια τας κρίσεις προς προαγωγήν εις τας θέσεις των συμβούλων της Επικρατείας, αρεοπαγιτών, αντεισαγγελέων του Αρείου Πάγου, προέδρων εφετών, εισαγγελέων εφετών και συμβούλων του Ελεγκτικού Συνεδρίου, το κατά την παράγραφον 1 συμβούλιον συγκροτείται υπό ηυξημένην σύνθεσιν, ως νόμος ορίζει. Η διάταξις του τελευταίου εδαφίου της παραγράφου 1 έχει εφαρμογήν και εν προκειμένω.
  \item[3.] Αν ο Υπουργός διαφωνή προς την κρίσιν ανωτάτου δικαστικού συμβουλίου, δύναται να παραπέμψη το κριθέν ζήτημα εις την ολομέλειαν του οικείου ανωτάτου δικαστηρίου, ως νόμος ορίζει. Δικαίωμα προσφυγής εις την ολομέλειαν έχει και ο παραλειφθής δικαστικός λειτουργός, υπότας εν τω νόμω οριζομένας προϋποθέσεις.
  \item[4.] Αι αποφάσεις της ολομελείας επί του παραπεμφθέντος αυτή ζητήματος, ως και αι αποφάσεις ανωτάτου δικαστικού συμβουλίου, προς τας οποίας δεν διεφώνησεν ο Υπουργός, είναι υποχρεωτικά δι' αυτόν.
  \item[5.] Αι προαγωγαί εις τας θέσεις του προέδρου και των αντιπροέδρων του Συμβουλίου της Επικρατείας, του Αρείου Πάγου και του Ελεγκτικού Συμβουλίου ενεργούνται δια προεδρικού διατάγματος, εκδιδομένου τη προτάσει του Υπουργικού συμβουλίου, κατ' επιλογήν μεταξύ των μελών του αντιστοίχου ανωτάτου δικαστηρίου, ως νόμος ορίζει.

	Η προαγωγή εις την θέσιν του εισαγγελέως του Αρείου Πάγου ενεργείται δι' ομοίου διατάγματος, κατ' επιλογήν μεταξύ των μελών του Αρείου Πάγου και των παρ' αυτώ αντεισαγγελέων.
  \item[6.] Αι κατά τας διατάξεις του παρόντος άρθρου αποφάσεις ή πράξεις δεν υπόκειται εις προσβολήν ενώπιον του Συμβουλίου της Επικρατείας.
\end{enumerate}

\Article{91}{}
\TOCArticle{91}{}
\begin{enumerate}
  \item[1.] Η πειθαρχική εξουσία επί των δικαστικών λειτουργών, από του βαθμού του αρεοπαγίτου ή αντεισαγγελέως του Αρείου Πάγου και ανωτέρου, η των τούτοις αντιστοίχων ασκείται υπό ανωτάτου πειθαρχικού συμβουλίου, ως νόμος ορίζει.

	Την πειθαρχικήν  αγωγήν εγείρει ο Υπουργός της Δικαιοσύνης.
  \item[2.] Το Ανώτατον Πειθαρχικόν Συμβούλιον  συγκροτείται υπό του Προέδρου του Συμβουλίου της Επικρατείας, ως Προέδρου αυτού, εκ δύο αντιπροέδρων ή συμβούλων της Επικρατείας, σύο αντιπροέδρων του Αρείου Πάγου 'η αρεοπαγιτών, δύο εκ των αντιπροέδρων ή συμβούλων του Ελεγκτικού Συνεδρίου και δύο τακτικών καθηγητών νομικών μαθημάτων των νομικών σχολών των πανεπιστημίων της Χώρας, ως μελών. Τα μέλη του Συμβουλίου ορίζονται δια κληρώσεως  μεταξύ των εχόντων τριετή τουλάχιστον υπηρεσίαν εις το οικείον ανώτατον δικαστήριον ή εις νομικήν σχολήν, αποκλείονται δε εκάστοτε της συνθέσεως αυτού τα μέλη τα ανήκοντα εις το δικαστήριον, επί ενεργείας μέλους, εισαγγελέως ή επιτρόπου του οποίου καλείται να αποφανθή το Συμβούλιον. Εφ' όσον πρόκειται περί πειθαρχικής διώξεως κατά μελών του Συμβουλίου της Επικρατείας, του Ανωτάτου Πειθαρχικού Συμβουλίου προεδρεύει ο Πρόεδρος του Αρείου Πάγου.
  \item[3.] Η πειθαρχική εξουσία επί των λοιπών δικαστικών λειτουργών ασκείται εις πρώτον και εις δέυτερον βαθμόν υπό συμβουλίων συγκροτουμένων εκ τακτικών δικαστών δια κληρώσεως,κατά τα υπό του νόμου οριζόμενα. Την πειθαρχικήν αγωγήν εγείρει και ο Υπουργός της Δικαιοσύνης.
  \item[4.] Αι κατά τας διατάξεις του παρόντος άρθρου πειθαρχικαί αποφάσεις δεν υπόκεινται εις προσβολήν ενώπιον του Συμβουλίου της Επικρατείας.
\end{enumerate}

\Article{92}{}
\TOCArticle{92}{}
\begin{enumerate}
  \item[1.] Οι υπάλληλοι της γραμματείας πάντων των δικαστηρίων και των εισαγγελιών είναι μόνιμοι. Ούτοι δύναται να παυθούν μόνον δυνάμει δικαστικής αποφάσεως ένεκα ποινικής καταδίκης, ή δι' αποφάσεως δικαστικού συμβουλίου ένεκα βαρέος πειθαρχικού παραπτώματος, νόσου ή αναπηρίας ή υπηρεσιακής ανεπαρκείας, βεβαιουμένων καθ' ον τρόπον νόμος ορίζει.
  \item[2.] Τα προσόντα των υπαλλήλων της γραμματείας πάντων των δικαστηρίων και των εισαγγελιών, ως και τα της εν γένει καταστάσεως αυτών, ορίζονται δια νόμου.
  \item[3.] Αι προαγωγαί, τοποθετήσεις, μεταθέσεις, αποσπάσεις και μετατάξεις των δικαστικών υπαλλήλων ενεργούνται μετά σύμφωνον γνώμην δικαστικών συμβουλίων, η πειθαρχική δε επ' αυτών εξουσία ασκείται υπό των ιεραρχικώς  προϊσταμένων αυτών δικαστών ή εισαγγελέων ή επιτρόπων, ως και υπό δικαστικών συμβούλων κατά τα υπό του νόμου οριζόμενα,

	Κατά των περί προαγωγής, ως και των πειθαρχικών αποφάσεων των δικαστικών  συμβουλίων, επιτρέπεται προσφυγή, ως νόμος ορίζει.
  \item[4.] Οι συμβολαιογράφοι, οι φύλακες υποθηκών και μεταγραφών και οι διευθυνταί των κτηματολογικών γραφείων είναι μόνιμοι, εφ' όσον υφίστανται αι σχετικαί υπηρεσίαι και θέσεις.

	Αι διατάξεις  των προηγουμένων παραγράφων έχουν ανάλογον εφαρμογήν και επ' αυτών.
  \item[5.] Οι συμβολαιογράφοι και οι άμισθοι φύλακες υποθηκών και μεταγραφών αποχωρούν της υπηρεσίας υποχρεωτικώς άμα τη συμπληρώσει του εβδομηκοστού έτους της ηλικίας των, οι δε λοιποί άμα τη συμπληρώσει του υπό του νόμου προβλεπομένου ορίου.
\end{enumerate}

\Chapter{Δεύτερον}{Οργάνωσις και δικαιοδοσία των δικαστηρίων}
\Article{93}{}
\TOCArticle{93}{}
\begin{enumerate}
  \item[1.] Τα δικαστήρια διακρίνονται ως διοικητικά, πολιτικά και ποινικά, οργανούνται δε  δι ειδικών νόμων.
  \item[2.] Αι συνεδριάσεις παντός δικαστηρίου είναι δημόσιαι, εκτός εάν δι' αποφάσεως τούτου κριθή  ότι η δημοσιότης πρόκειται να είναι επιβλαβής εις τα χρηστά ήθη ή ότι συντρέχουν ειδικοί λόγοι προς προστασίαν του ιδιωτικού ή οικογενειακού βίου των διαδίκων.
  \item[3.] Πάσα δικαστική απόφασις πρέπει να είναι ειδικώς και εμπεριστατωμένως ητιολογημένη, απαγγελεται δε εν δημοσία συνεδριάσει. Η γνώμη της μειοψηφίας δημοσιεύεται υποχρεωτικώς. Νόμος ορίζει τα της εις τα πρακτικά καταχωρήσεως ενδεχομένης μειοψηφίας και τους όρους και προϋποθέσεις δημοσιότητος ταύτης.
  \item[4.] Τα δικαστήρια υποχρεούνται όπως μη εφαρμόζουν νόμον, το περιεχόμενον του οποίου αντίκειται προς το Σύνταγμα.
\end{enumerate}

\Article{94}{}
\TOCArticle{94}{}
\begin{enumerate}
  \item[1.] Η εκδίκασις των διοικητικών διαφορών ουσίας ανήκει εις τα υφιστάμενα τακτικά διοικητικά δικαστήρια . Εκ των ως άνω διαφορών αι μη υπαχθείσαι εισέτι εις τα δικαστήρια ταύτα, δέον να υπαχθούν υποχρεωτικώς εις την δικαιοδοσίαν αυτών, εντός πέντε ετών από της ισχύος του παρόντος, της προθεσμίας ταύτης δυναμένης να παρατείνεται δια νόμου.
  \item[2.] Μέχρι της υπαγωγής εις τα τακτικά διοικητικά δικαστήρια και των λοιπών ουσιαστικών διοικητικών διαφορών, είτε εν τω συνόλω είτε κατά κατηγορίας, αύται εξακολουθούν να υπάγωνται εις τα πολιτικά δικαστήρια, πλην εκείνων δια τας οποίας ειδικοί νόμοι συνέστησαν ειδικά διοικητικά δικαστήρια εις τα οποία τηρούνται αι διατάξεις των παραγράφων 2 έως 4 του άρθρου 93.
  \item[3.] Εις τα πολιτικά δικαστήρια υπάγονται πάσαι αι ιδιωτικαί διαφοραί, ως και αι δια νόμου ανατιθέμεναι εις ταύτα υποθέσεις εκουσίας δικαιοδοσία.
  \item[4.] Εις τα πολιτικά ή διοικητικά δικαστήρια δύναται να ανατεθή και πάσα άλλη υπό του νόμου οριζομένη διοικητικής φύσεως αρμοδιότης.
\end{enumerate}
\textbf{Ερμηνευτική δήλωσις:}

Ως τακτικά διοικητικά δικαστήρια νοούνται μόνον τα συσταθέντα δια του νομοθετικού διατάγματος 3845/1958 τακτικά φορολογικά δικαστήρια.

\Article{95}{}
\TOCArticle{95}{}
\begin{enumerate}
  \item[1.] Εις την αρμοδιότητα του Συμβουλίου της Επικρατείας ανήκουν ιδίως:
  \begin{enumerate}
  	\item[α)] Η κατ' αίτησιν ακύρωσις των εκτελεστών πράξεων των διοικητικών αρχών, δι' υπέρβασιν εξουσίας ή παράβασιν νόμου.
  	\item[β)] Η κατ' αίτησιν αναίρεσις των τελεσιδίκων αποφάσεων των διοικητικών δικαστηρίων, δι' υπέρβασιν εξουσίας ή παράβασιν νόμου.
  	\item[γ)] Η εκδίκασις των κατά το Σύνταγμα και τους νόμους υποβαλλομένων εις αυτό διοικητικών διαφορών ουσίας,
  	\item[δ)] Η επεξεργασία πάντων των κανονιστικού χαρακτήρος διαταγμάτων.
  \end{enumerate}

  \item[2.] Κατά την άσκησιν των υπό στοιχ. δ' της προηγουμένης παραγράφου αρμοδιοτήτων δεν εφαρμόζονται αι διατάξεις του άρθρου 93 παράγραφοι 2 και 3.
  \item[3.] Δια νόμου δύναται να υπαχθή η εκδίκασις κατηγοριών υποθέσεων της ακυρωτικής αρμοδιότητος του Συμβουλίου της Επικρατείας εις άλλου βαθμού διοικητικά
 τακτικά δικαστήρια, επιφυλασσομένης πάντως της εις τελευταίον βαθμόν αρμοδιότητος του Συμβουλίου της Επικρατείας.
  \item[4.] Αι αρμοδιότητες του Συμβουλίου της Επικρατείας ρυθμίζονται και ασκούνται ως ειδικώτερον νόμος ορίζει.
  \item[5.] Η διοίκησις έχει υποχρέωσιν συμμορφώσεως προς τας ακυρωτικάς αποφάσεις του Συμβουλίου της Επικρατείας. Η παράβασις της υποχρεώσεως ταύτης δημιουργεί ευθύνην δια παν υπαίτιον όργανον, ως νόμος ορίζει.
\end{enumerate}

\Article{96}{}
\TOCArticle{96}{}
\begin{enumerate}
  \item[1.] Εις τα τακτικά ποινικά δικαστήρια ανήκει ο κολασμός των εγκλημάτων και η λήψις πάντων των κατά τους ποινικούς νόμους μέτρων.
  \item[2.] Δύναται  δια νόμου: α) να ανατεθή και εις τας αρχάς ασκούσας αστυνομικά καθήκοντα ή έκδίκασις αστυνομικών παραβάσεων τιμωρουμένων δια προστίμου, β) να ανατεθή εις αρχάς αγροτικής ασφαλείας η εκδίκασις των περί τους αγρούς πταισμάτων και των εξ αυτών απορρεουσών ιδιωτικών διαφορών.

	Εις αμφοτέρας  τας περιπτώσεις αι εκδιδόμεναι αποφάσεις υπόκεινται εις έφεσιν ενώπιον του αρμοδίου τακτικού δικαστηρίου, έχουσαν ανασταλτικήν δύναμιν.
  \item[3.] Ειδικοί νόμοι ορίζουν τα περί δικαστηρίων ανηλίκων, εφ ων επιτρέπεται να μη έχουν εφαρμογήν αι διατάξεις των άρθρων 93 παράγραφος 2 και 97. Αι αποφάσεις των δικαστηρίων τούτων δύναται να απαγγέλλωνται κεκλεισμένων των θυρών.
  \item[4.] Ειδικοί νόμοι ορίζουν:
  \begin{enumerate}
    \item[α)] τα περί στρατοδικείων, ναυτοδικείων και αεροδικείων εις την αρμοδιότητα των οποίων δεν δύναται να υπαχθούν ιδιώται.
	\item[β)] Τα περί δικαστηρίων λειών.
  \end{enumerate}
  \item[5.] Τα υπό στοιχείον α' της προηγουμένης παραγράφου δικαστήρια συγκροτούνται κατά πλειοψηφίαν εκ μελών του δικαστικού σώματος των ενόπλων δυνάμεων, περιβαλλομένων υπό των κατά το άρθρον 87 παρ.1 του παρόντος εγγυήσεων λειτουργικής και προσωπικής ανεξαρτησίας. Δια τας συνεδριάσεις και αποφάσεις των δικαστηρίων τούτων εφαρμόζονται αι διατάξεις των παραγράφων 2 έως 4 του άρθρου 93. Τα της εφαρμογής των διατάξεων της παρούσης παραγράφου, ως και ο χρόνος ενάρξεως της ισχύος αυτών, ορίζονται δια νόμου.
\end{enumerate}

\Article{97}{}
\TOCArticle{97}{}
\begin{enumerate}
  \item[1.] Τα κακουργήματα και τα πολιτικά εγκλήματα δικάζονται υπό μικτών ορκωτών δικαστηρίων, συγκροτουμένων εκ τακτικών δικαστών και ενόρκων, ως νόμος ορίζει. Αι αποφάσεις των δικαστηρίων τούτων υπόκεινται εις τα υπό του νόμου οριζόμενα ένδικα μέσα.
  \item[2.] Κακουργήματα και πολιτικά εγκλήματα, υπαχθέντα μέχρι της ισχύος του παρόντος δια συντακτικών πράξεων, ψηφισμάτων και ειδικών νόμων, εις την δικαιοδοσίαν των εφετείων, εξακολουθούν να δικάζωνται υπ' αυτών, εφ' όσον νόμος δεν υπαγάγη ταύτα εις την αρμοδιότητα των μικτών ορκωτών δικαστηρίων.
  
	Δια νόμου δύναται να υπαχθούν εις την δικαιοδοσίαν των αυτών εφετείων και έτερα κακουργήματα.
  \item[3.] Τα δια του τύπου διαπραττόμενα εγκλήματα παντός βαθμού υπάγονται εις τα τακτικά ποινικά δικαστήρια, ως νόμος ορίζει.
\end{enumerate}

\Article{98}{}
\TOCArticle{98}{}
\begin{enumerate}
  \item[1.] Εις την αρμοδιότητα του Ελεγκτικού Συνεδρίου ανήκουν ιδίως:
  \begin{enumerate}
  	\item[α)] Ο έλεγχος των δαπανών του κράτους, ως και των δι' ειδικών νόμων εις τον έλεγχον αυτού υπαγομένων εκάστοτε οργανισμών τοπικής αυτοδιοικήσεως ή άλλων νομικών προσώπων δημοσίου δικαίου.
  	\item[β)] Η έκθεσις προς την Βουλήν επί του απολογισμού και ισολογισμού του Κράτους.
  	\item[γ)] Η γνωμοδότησις  επί των νόμων περί συντάξεων ή αναγνωρίσεως υπηρεσίας δια την παροχήν δικαιώματος συντάξεως κατά το άρθρον 73 παράγραφος 2, ως  και επί παντός ετέρου θέματος οριζομένου υπό του νόμου.
  	\item[δ)] Ο έλεγχος των λογαριασμών των δημοσίων υπολόγων και των εν εδαφίω α' οργανισμών τοπικής αυτοδιοικήσεως και των νομικών προσώπων δημοσίου δικαίου.
  	\item[ε)] Η εκδίκασις ενδίκων μέσων επί διαφορών εξ απονομής συντάξεων, ως και εκ του ελέγχου των λογαριασμών εν γένει.
  	\item[στ)] Η εκδίκασις υποθέσεων αναφερομένων εις την ευθύνην των δημοσίων πολιτικών ή στρατιωτικών υπαλλήλων, ως και των υπαλλήλων των οργανισμών τοπικής αυτοδιοικήσεως, δια πάσαν εκ δόλου ή αμελείας επελθούσαν εις το Κράτος ή εις τους ανωτέρω οργανισμούς και νομικά πρόσωπα ζημίαν.
  \end{enumerate}
  \item[2.] Αι αρμοδιότητες του Ελεγκτικού Συνεδρίου ρυθμίζονται και ασκούνται, ως νόμος ορίζει.
Κατά τας υπό στοιχεία α' έως δ' περιπτώσεις της προηγουμένης παραγράφου δεν εφαρμόζονται ααι διατάξεις του άρθρου 93 παράγραφοι 2 και 3.
  \item[3.] Αι αποφάσεις του Ελεγκτικού Συνεδρίου επί των εν τη παραγράφω 1 υποθέσεων, δεν υπόκεινται εις τον έλεγχον του Συμβουλίου της Επικρατείας.
\end{enumerate}

\Article{99}{}
\TOCArticle{99}{}
\begin{enumerate}
  \item[1.] Αγωγαί κακοδικίας κατά δικαστικών λειτουργών δικάζονται, ως νόμος ορίζει, υπό ειδικού δικαστηρίου, συγκροτουμένου υπό του Προέδρου του Συμβουλίου της Επικρατείας, ως Προέδρου αυτού και εξ ενός συμβούλου της Επικρατείας, ενός αρεοπαγίτου, ενός συμβούλου του Ελεγκτικού Συνεδρίου, δύο τακτικών καθηγητών νομικών μαθημάτων των νομικών σχολών των πανεπιστημίων της Χώρας και δύο δικηγόρων εκ των μελών του Ανωτάτου Πειθαρχικού συμβουλίου των δικηγόρων, ως μελών, οριζομένων δια κληρώσεως.
  \item[2.] Εκ των μελών του ειδικού δικαστηρίου εξαιρείται εκάστοτε το ανήκον εις το σώμα ή τον κλάδον της δικαιοσύνης, επί ενεργείας ή παραλείψεως λειτουργών του οποίου καλείται να αποφανθή το δικαστήριον. Εφ' όσον πρόκειται  περί αγωγής κακοδικίας κατά μέλους του Συμβουλίου της Επικρατείας ή λειτουργών των τακτικών διοικητικών δικαστηρίων, του ως άνω ειδικού δικαστηρίου προεδρεύει ο πρόεδρος του Αρείου Πάγου.
  \item[3.] Προς έγερσιν αγωγής κακοδικίας ουδεμία απαιτείται άδεια.
\end{enumerate}

\Article{100}{}
\TOCArticle{100}{}
\begin{enumerate}
  \item[1.] Συνιστάται ανώτατον Ειδικόν Δικαστήριον εις το οποίον υπάγονται:
\begin{enumerate}
  	\item[α)] Η εκδίκασις ενστάσεων κατά το άρθρον 58.
  	\item[β)] Ο έλεγχος του κύρους και των αποτελεσμάτων δημοψηφίσματος, ενεργουμένου κατά το άρθρον 44 παράγραφος 2.
  	\item[γ)] Η κρίσις περί των ασυμβιβάστον ή της εκπτώσεως βουλευτού κατά τα άρθρα 55 παράγραφος 2 και 57.
  	\item[δ)] Η άρσις των συγκρούσεων μεταξύ των δικαστηρίων και των διοικητικών αρχών ή μεταξύ του Συμβουλίου της Επικρατείας και των τακτικών διοικητικών δικαστηρίων αφ' ενός και των αστικών και ποινικών δικαστηρίων αφ' ετέρου ή τέλος μεταξύ του Ελεγκτικού Συνεδρίου και των λοιπών δικαστηρίων.
  	\item[ε)] Η άρσις της αμφισβητήσεως περί της ουσιαστικής αντισυνταγματικότητος ή της εννοίας διατάξεων τυπικού νόμου, εάν εξεδόθησαν περί αυτών αντίθετοι αποφάσεις του Συμβουλίου της Επικρατείας, του Αρείου Πάγου ή του Ελεγκτικού Συνεδρίου.
  	\item[στ)] Η άρσις της αμφισβητήσεως περί τον χαρακτηρισμόν κανόνων διεθνούς δικαίου ως γενικώς παραδεδεγμένων, κατά την παράγραφον 1 του άρθρου 28.
	\end{enumerate}
  \item[2.] Το κατά την προηγουμένην παράγραφον δικαστήριον συγκροτείται εκ των Προέδρων του Συμβουλίου της Επικρατείας, του Αρείου Πάγου και του Ελεγκτικού Συνεδρίου, εκ τεσσάρων συμβούλων της Επικρατείας και εκ τεσσάρων αρεοπαγιτών, οριζομένων ανά διετίαν δια κληρώσεως ως μελών. Του δικαστηρίου τούτου προεδρεύει ο αρχαιότερος των Προέδρων του Συμβουλίου της Επικρατείας ή του Αρείου Πάγου.

	Εις τας περιπτώσεις δ' και ε' της προηγουμένης παραγράφου μετέχουν της συνθέσεως του δικαστηρίου και δύο τακτικοί καθηγηταί νομικών μαθημάτων των νομικών σχολών των πανεπιστημίων της Χώρας, οριζόμενοι δια κληρώσεως.
  \item[3.] Η οργάνωσις και λειτουργία του δικαστηρίου, τα του ορισμού, αναπληρώσεως και επικουρίας των μελών αυτού, ως και τα της ενώπιον αυτού διαδικασίας ορίζονται δι' ειδικού νόμου.
  \item[4.] Αι αποφάσεις του δικαστηρίου είναι αμετάκλητοι.
	Διάταξις νόμου κηρυσσομένη ως αντισυνταγματική είναι ανίσχυρος από της δημοσιεύσεως της περί τούτου αποφάσεως ή από του υπό της αποφάσεως οριζομένου χρόνου.
\end{enumerate}

\Section{ΣΤ'}{ΔΙΟΙΚΗΣΙΣ}
\Chapter{Πρώτον}{Οργάνωσις της  Διοικήσεως}
\Article{101}{}
\TOCArticle{101}{}
\begin{enumerate}
  \item[1.] Η διοίκησις του Κράτους οργανούται κατά το αποκεντρωτικόν σύστημα.
  \item[2.] Η διοικητική διαίρεσις της Χώρας διαμορφούται βάσει των γεωοικονομικών, κοινωνικών και συγκοινωνιακών συνθηκών.
  \item[3.] Τα περιφερειακά κρατικά όργανα έχουν γενικήν αποφασιστικήν αρμοδιότητα επί των υποθέσεων της περιφερείας των, αι δε κεντρικαί υπηρεσίαι, πλην ειδικών αρμοδιοτήτων, την γενικήν κατεύθυνσιν, τον συντονισμόν και τον έλεγχον των περιφερειακών οργάνων, ως νόμος ορίζει.
\end{enumerate}

\Article{102}{}
\TOCArticle{102}{}
\begin{enumerate}
  \item[1.] Η διοίκησις των τοπικών υποθέσεων ανήκει εις τους οργανισμούς τοπικής αυτοδιοικήσεως, των οποίων την πρώτην βαθμίδα αποτελούν οι δήμοι και αι κοινότητες. Αι λοιπαί βαθμίδες ορίζονται δια νόμου.
  \item[2.] Οι οργανισμοί τοπικής αυτοδιοικήσεως απολαύουν διοικητικής αυτοτελείας. Αι αρχαί αυτών εκλέγονται δια καθολικής και μυστικής ψηφοφορίας.
  \item[3.] Δια νόμου δύναται να προβλέπωνται  αναγκαστικοί ή εκούσιοι σύνδεσμοι οργανισμών τοπικής αυτοδιοικήσεως προς εκτέλεσιν έργων ή παροχήν υπηρεσιών, διοικούμενοι υπό συμβουλίου εξ αιρετών αντιπροσώπων εκάστου δήμου ή κοινότητος, λαμβανομένων κατ' αναλογίαν του πληθυσμού τούτων.
  \item[4.] Δια νόμου δύναται να προβλεφθή η εις την διοίκησιν των οργανισμών τοπικής αυτοδιοικήσεως δευτέρας βαθμίδος συμμετοχή αιρετών αντιπροσώπων τοπικών επαγγελματικών, επιστημονικών και πνευματικών οργανώσεων και της κρατικής διοικήσεως μέχρι του ενός τρίτου του όλου αριθμού των μελών.
  \item[5.] Το Κράτος ασκεί εποπτείαν επί των οργανισμών της τοπικής αυτοδιοικήσεως, μη εμποδίζουσαν την πρωτοβουλίαν και την ελευθέραν δράσιν αυτών. Αι πειθαρχικαί ποιναί αργίας και απολύσεως εκ του αξιώματος των αιρετών οργάνων της τοπικής αυτοδιοικήσεως, εξαιρέσει των περιπτώσεων των συνεπαγομένων αυτοδικαίαν έκπτωσιν, απαγγέλλονται μόνον μετά σύμφωνον γνώμην συμβουλίου αποτελουμένου κατά πλειοψηφίαν εκ τακτικών δικαστών.
  \item[6.] Το Κράτος μεριμνά δια την εξασφάλισιν των αναγκαίων πόρων, προς εκπλήρωσιν της αποστολής των οργανισμών τοπικής αυτοδιοικήσεως. Νόμος ορίζει τα της αποδόσεως και κατανομής μεταξύ των ως άνω οργανισμών των υπέρ αυτών καθοριζομένων και υπό του Κράτους εισπραττομένων φόρων ή τελών.
\end{enumerate}

\Chapter{Δεύτερον}{Υπηρεσιακή  κατάστασις  των  οργάνων της  διοικήσεως}
\Article{103}{}
\TOCArticle{103}{}
\begin{enumerate}
  \item[1.] Οι δημόσιοι υπάλληλοι είναι εκτελεσταί της θελήσεως του Κράτους και υπηρετούν  τον λαόν, οφείλοντες πίστιν εις το Σύνταγμα  και αφοσίωσιν εις την Πατρίδα. Τα προσόντα και ο τρόπος διορισμού  τούτων καθορίζονται υπό του νόμου.
  \item[2.] Ουδείς δύναται να διορισθή υπάλληλος εις μη νενομοθετημένην οργανικήν θέσιν. Εξαιρέσεις δύνανται να προβλέπωνται υπό ειδικού  νόμου προς κάλυψιν απροβλέπτων και επειγουσών αναγκών δια προσωπικού προσλαμβανομένου δι' ωρισμένην χρονικήν περίοδον, επί σχέσει ιδιωτικού δικαίου.
  \item[3.] Οργανικαί θέσεις ειδικού επιστημονικού, ως και τεχνικού ή βοηθητικού προσωπικού, δύναται να πληρούνται δια προσωπικού προσλαμβανομένου επί σχέσει ιδιωτικού δικαίου. Νόμος ορίζει τους όρους της προσλήψεως, ως και ειδικωτέρας εγγυήσεις, υφ' ας τελεί το προσλαμβανόμενον προσωπικόν.
  \item[4.] Οι κατέχοντες οργανικάς θέσεις δημόσιοι υπάλληλοι είναι μόνιμοι, εφ' όσον υφίστανται αι θέσεις αύται. Ούτοι εξελίσσονται μισθολογικώς κατά τους όρους του νόμου, πλην δε των περιπτώσεων της αποχωρήσεως λόγω ορίου ηλικίας και της παύσεως συνεπεία δικαστικής αποφάσεως, δεν δύναται να μετατεθούν άνευ γνωμοδοτήσεως, ουδέ να υποβιβασθούν ή παυθούν άνευ αποφάσεως υπηρεσιακού συμβουλίου, αποτελουμένου κατά τα δύο τρίτα αυτού τουλάχιστον εκ μονίμων δημοσίων υπαλλήλων.
  
	Κατά των αποφάσεων των συμβουλίων τούτων χωρεί προσφυγή ενώπιον του Συμβουλίου της Επικρατείας, ως νόμος ορίζει.
  \item[5.] Της μονιμότητος δύναται να εξαιρούνται,  δια νόμου, ανώτατοι διοικητικοί υπάλληλοι των εκτός της υπαλληλικής ιεραρχίας θέσεων, οι απ' ευθείας διοριζόμενοι επί πρεσβευτικώ βαθμώ, οι υπάλληλοι της Προεδρίας της Δημοκρατίας και των γραφείων του Πρωθυπουργού, των Υπουργών και Υφυπουργών.
  \item[6.] Αι διατάξεις των προηγουμένων παραγράφων έχουν εφαρμογήν και επί των υπαλλήλων της Βουλής, διεπομένων εξ ολοκλήρου κατά τα λοιπά υπό του Κανονισμού αυτής, ως και επί των υπαλλήλων των οργανισμών τοπικής αυτοδιοικήσεως και λοιπών νομικών προσώπων δημοσίου δικαίου.
\end{enumerate}

\Article{104}{}
\TOCArticle{104}{}
\begin{enumerate}
  \item[1.] Ουδείς δύναται εκ των εν τω προηγουμένω άρθρω υπαλλήλων δύναται να διορισθή εις ετέραν θέσιν δημοσίας υπηρεσίας ή οργανισμού τοπικής αυτοδιοικήσεως ή ετέρου νομικού προσώπου δημοσίου δικαίου ή δημοσίας επιχειρήσεως ή οργανισμού κοινής ωφελείας. Κατ' εξαίρεσιν δύναται να επιτραπή ο διορισμός και εις δευτέραν θέσιν επί τη βάσει ειδικού νόμου, τηρουμένων των διατάξεων της επομένης παραγράφου.
  \item[2.] Αι πάσης φύσεως πρόσθετοι αποδοχαί ή απολαυαί των κατά το προηγούμενον άρθρον υπαλλήλων δεν δύναται να είναι κατά μήνα ανώτεραι του συνόλου των αποδοχών της οργανικής αυτών θέσεως.
  \item[3.] Ουδεμία προηγουμένη άδεια απαιτείται προς εισαγωγήν εις δίκην δημοσίων υπαλλήλων, ως και υπαλλήλων οργανισμών τοπικής αυτοδιοικήσεως, ή άλλων νομικών προσώπων δημοσίου δικαίου.
\end{enumerate}

\Chapter{Τρίτον}{Καθεστώς  του Αγίου Όρους}
\Article{105}{}
\TOCArticle{105}{}
\begin{enumerate}
  \item[1.] Η χερσόνησος του Άθω από της Μεγάλης Βίγλας και εξής, αποτελούσα την περιοχήν του Αγίου Όρους, είναι κατά το αρχαίον τούτου προνομιακόν καθεστώς αυτοδιοίκητον τμήμα του Ελληνικού Κράτους, του οποίου η κυριαρχία παραμένει άθικτος επ' αυτού. Εξ απόψεως  πνευματικής το Άγιον Όρος διατελεί υπό την άμεσον δικαιοδοσίαν του Οικουμενικού Πατριαρχείου. Όλοι οι μονάζοντες εις αυτό αποκτούν, άνευ άλλης διατυπώσεως, την ελληνικήν ιθαγένειαν, άμα τη προσλήψει αυτών ως δοκίμων ή μοναχών.
  \item[2.] Το Άγιον Όρος διοικείται, κατά το καθεστώς αυτού, υπό των είκοσιν Ιερών Μονών του, μεταξύ των οποίων είναι κατανεμημένη ολόκληρος η χερσόνησος του Άθω, το έδαφος της οποίας είναι αναπαλλοτρίωτον.
  
	Η διοίκησις αυτού ασκείται δι' αντιπροσώπων των Ιερών Μονών, αποτελούντων την Ιεράν Κοινότητα. Ουδεμία απολύτως επιτρέπεται μεταβολή του διοικητικού συστήματος ή του αριθμού των Μονών του Αγίου Όρους, ουδέ της ιεραρχικής τάξεως και της θέσεως αυτών προς τα υποτελή των εξαρτήματα. Απαγορεύεται η εν αυτώ  εγκαταβίωσις ετεροδόξων ή σχισματικών.
  \item[3.] Ο λεπτομερής καθορισμός των αγιορειτικών καθεστώτων και του τρόπου της λειτουργίας αυτών γίνεται δια του Καταστατικού Χάρτου του Αγίου Όρους, το οποίον, συμπράττοντος του αντιπροσώπου του Κράτους, συντάσσουν μεν και ψηφίζουν αι είκοσιν Ιεραί Μοναί, επικυρώνουν δε το Οικουμενικόν Πατριαρχείον και η Βουλή των Ελλήνων.
  \item[4.] Η ακριβής τήρησις των αγιορειτικών καθεστώτων τελεί, ως προς μεν το πνευματικόν μέρος, υπό την ανωτάτην εποπτείαν του Οικουμενικού Πατριαρχείου, ως προς δε το διοικητικόν, υπό την εποπτείαν του Κράτους, εις το οποίον ανήκει αποκλειστικώς και η διαφύλαξις της δημοσίας τάξεως και ασφαλείας.
  \item[5.] Αι ανωτέρω εξουσίαι του Κράτους ασκούνται δια διοικητού, του οποίου τα δικαιώματα και καθήκοντα καθορίζονται δια νόμου.

	Δια νόμου επίσης καθορίζονται η υπό των μοναστηριακών αρχών και της Ιεράς Κοινότητος ασκουμένη δικαστική εξουσία, ως και τα τελωνειακά και φορολογικά πλεονεκτήματα του Αγίου Όρους.
\end{enumerate}

\Part{ΜΕΡΟΣ ΤΕΤΑΡΤΟΝ}{ΕΙΔΙΚΑΙ  ΤΕΛΙΚΑΙ  ΚΑΙ  ΜΕΤΑΒΑΤΙΚΑΙ  ΔΙΑΤΑΞΕΙΣ}
\Section{Α'}{ΕΙΔΙΚΑΙ  ΔΙΑΤΑΞΕΙΣ}
\Chapter{Πρώτον}{Οργάνωσις της  Διοικήσεως}
\Article{106}{}
\TOCArticle{106}{}
\begin{enumerate}
  \item[1.] Προς εδραίωσιν της κοινωνικής ειρήνης και προστασίαν του γενικού συμφέροντος, το Κράτος προγραμματίζει και συντονίζει την οικονομικήν δραστηριότητα εν τη Χώρα, επιδιώκον την εξασφάλισιν της οικονομικής αναπτύξεως όλων των τομέων της εθνικής οικονομίας. Λαμβάνει τα επιβαλλόμενα μέτρα προς αξιοποίησιν των πηγών του εθνικού πλούτου εκ της ατμοσφαίρας και των υπογείων ή υποθαλασσίων κοιτασμάτων και προς προώθησιν της περιφερειακής αναπτύξεως και προαγωγήν ιδία της οικονομίας των ορεινών, νησιωτικών και παραμεθορίων περιοχών.
  \item[2.] Η ιδιωτική οικονομική πρωτοβουλία δεν επιτρέπεται να αναπτύσσεται εις βάρος της ελευθερίας και της ανθρωπίνης αξιοπρεπείας, ή επί βλάβη της εθνικής οικονομίας.
  \item[3.] Επιφυλασσομένης της υπό του άρθρου 107 παρεχομένης προστασίας, ως προς την επανεξαγωγήν κεφαλαίων εξωτερικού, δύναται δια νόμου να ρυθμίζωνται τα της εξαγοράς επιχειρήσεων ή αναγκαστικής εις ταύτας συμμετοχής του Κράτους ή άλλων δημοσίων φορέων, εφ' όσον αύται κέκτηνται χαρακτήρα μονοπωλίου ή έχουν ζωτικήν σημασίαν δια την αξιοποίησιν των πηγών του εθνικού πλούτου ή έχουν ως κύριον σκοπόν την παροχήν υπηρεσιών προς το κοινωνικόν σύνολον.
  \item[4.] Το τίμημα της εξαγοράς ή το αντάλλαγμα της αναγκαστικής συμμετοχής του Κράτους ή άλλων δημοσίων φορέων, καθορίζεται απαραιτήτως δικαστικώς, πρέπει δε να είναι πλήρες, ανταποκρινόμενον προς την αξίαν της εξαγοραζομένης επιχειρήσεως, ή της εις ταύτην συμμετοχής.
  \item[5.] Μέτοχος, εταίρος ή κύριος επιχειρήσεως, της οποίας ο έλεγχος περιέρχεται εις το Κράτος ή εις υπ' αυτού ελεγχόμενον φορέα συνεπεία αναγκαστικής συμμετοχής κατά την παράγραφον 3, δικαιούται να ζητήση την εξαγοράν της συμμετοχής του εις την επιχείρησιν, ως νόμος ορίζει.
  \item[6.] Νόμος δύναται να ορίση τα της συμμετοχής εις την δαπάνην του Δημοσίου των οφειλουμένων εκ της εκτελέσεως έργων κοινής ωφελείας ή γενικωτέρας σημασίας δια την οικονομικήν ανάπτυξιν της Χώρας.
\end{enumerate}
\textbf{Ερμηνευτική δήλωσις:}

Εις την κατά την παράγραφον 4 αξίαν δεν περιλαμβάνεται  η οφειλομένη εις το μονοπωλιακόν, τυχόν, χαρακτήρα της επιχειρήσεως.

\Article{107}{}
\TOCArticle{107}{}
\begin{enumerate}
  \item[1.] Η προς της 21 Απριλίου 1967 ηυξημένης  τυπικής ισχύος νομοθεσίας προστασίας κεφαλαίων εξωτερικού, διατηρεί την ην εκέκτητο ηυξημένην τυπικήν ισχύν, εφαρμοζομένη και επί των εφεξής εισαγομένων κεφαλαίων.
  
	Την αυτήν ισχύν κέκτηνται και οι διατάξεις των κεφαλαίων Α' έως Δ' του Τμήματος Α' του υπ' αριθμόν 27/75 νόμου “περί φορολογίας πλοίων, επιβολής εισφοράς προς ανάπτυξιν της εμπορικής ναυτιλίας, εγκαταστάσεως αλλοδαπών ναυτιλιακών επιχειρήσεων και ρυθμίσεως συναφών θεμάτων”.
  \item[2.] Νόμος, εφ' άπαξ εκδιδόμενος, εντός τριμήνου από της ισχύος του παρόντος, ορίζει τους όρους και την διαδικασίαν αναθεωρήσεως ή λύσεως του κατ' εφαρμογήν του νομοθετικού διατάγματος 2687/1953, από 21 Απριλίου 1967 μέχρις 23 Ιουλίου 1974 εκδοθεισών υφ' οιονδήποτε τύπον εγκριτικών διοικητικών πράξεων ή συναφθεισών συμβάσεων περί επενδύσεων κεφαλαίων εξωτερικού, εξαιρέσει των αφορωσών εις την νηολόγησιν πλοίων υπό ελληνικήν σημαίαν.
\end{enumerate}

\Article{108}{}
\TOCArticle{108}{}
Το Κράτος μεριμνά δια την ζωήν του αποδήμου ελληνισμού και την διατήρησιν των δεσμών του με την μητέρα Πατρίδα. Επίσης μεριμνά δια την παιδείαν και την κοινωνικήν και επαγγελματικήν προαγωγήν των εκτός της επικρατείας εργαζομένων Ελλήνων.

\Article{109}{}
\TOCArticle{109}{}
\begin{enumerate}
  \item[1.] Δεν επιτρέπεται η μεταβολή του περιεχομένου η των όρων διαθήκης, κωδικέλλου ή δωρεάς κατά τας διατάξεις αυτής υπέρ του Δημοσίου ή υπέρ κοινωφελούς σκοπού.
  \item[2.] Εξαιρετικώς επιτρέπεται η επωφελεστέρα αξιοποίησις ή διάθεσις του καταλειφθέντος ή δωρηθέντος υπέρ του αυτού ή άλλου κοινωφελούς σκοπού εις την υπό του δωρητού ή του διαθέτου καθοριζομένην περιοχήν ή εις την ευρυτέραν ταύτης περιφέρειαν, όταν δια δικαστικής αποφάσεως βεβαιούται ότι η θέλησις του διαθέτου ή του δωρητού δεν δύναται να πραγματοποιηθή εξ οιουδήποτε λόγου, καθ' ολοκληρίαν ή κατά το μείζον του περιεχομένου ταύτης, ως και αν δύναται να ικανοποιηθή πληρέστερον δια της μεταβολής της εκμεταλλεύσεως, ως νόμος ορίζει.
\end{enumerate}

\Section{Β'}{ΑΝΑΘΕΩΡΗΣΙΣ  ΤΟΥ  ΣΥΝΤΑΓΜΑΤΟΣ}
\Article{110}{}
\TOCArticle{110}{}
\begin{enumerate}
  \item[1.] Αι διατάξεις του Συντάγματος υπόκεινται εις αναθεώρησιν, εξαιρέσει των καθοριζουσών την βάσιν και την μορφήν του πολιτεύματος ως Προεδρευομένης Κοινοβουλευτικής Δημοκρατίας, ως και των τοιούτων των άρθρων 2 παράγραφος 1, 4 παράγραφοι 1, 4, και 7, 5 παράγραφοι 1 και 3, 13 παράγραφος 1 και 26.
  \item[2.] Η ανάγκη της αναθεωρήσεως του Συντάγματος διαπιστούται δι' αποφάσεως της Βουλής, λαμβανομένης κατόπιν προτάσεως πεντήκοντα  τουλάχιστον βουλευτών, δια πλειοψηφίας των τριών πέμπτων του όλου αριθμού των μελών αυτής, εις δυο ψηφοφορίας αφισταμένας αλλήλων κατά ένα τουλάχιστον μήνα. Δια της αποφάσεως ταύτης καθορίζονται ειδικώς αι αναθεωρητέαι διατάξεις.
  \item[3.] Αποφασισθείσης της αναθεωρήσεως  υπό της Βουλής, η επομένη Βουλή κατά την πρώτην συνεδρίασιν αυτής αποφασίζει επί των αναθεωρητέων διατάξεων, δι' απολύτου πλειοψηφίας του όλου αριθμού των μελών αυτής.
  \item[4.] Εάν πρότασις περί αναθεωρήσεως του Συντάγματος έτυχε της πλειοψηφίας του όλου αριθμού των βουλευτών, ουχί όμως και της κατά την παράγραφον 2 πλειοψηφίας των τριών πέμπτων τούτων, η επομένη Βουλή κατά την πρώτην συνεδρίασιν αυτής δύναται να αποφασίση επί των αναθεωρητέων διατάξεων δια της πλειοψηφίας των τριών πέμπτων του όλου αριθμού των μελών αυτής.
  \item[5.] Πάσα ψηφιζομένη αναθεώρησις διατάξεων του Συντάγματος δημοσιεύεται δια της Εφημερίδος της Κυβερνήσεως εντός δέκα ημερών από της επιψηφίσεως αυτής υπό της Βουλής, τίθεται δε εν ισχύϊ δι' ειδικού ταύτης ψηφίσματος.
  \item[6.] Δεν επιτρέπεται  αναθεώρησις του Συντάγματος προ της παρόδου πενταετίας από της περατώσεως της προηγουμένης.
\end{enumerate}

\Section{Γ'}{ΜΕΤΑΒΑΤΙΚΑΙ  ΕΞΕΡΕΣΕΙΣ}
\Article{111}{}
\TOCArticle{111}{}
\begin{enumerate}
  \item[1.] Πάσα διάταξις νόμου ή διοικητικής πράξεως κανονιστικού χαρακτήρος, αντικειμένη εις το Σύνταγμα, καταργείται από της ενάρξεως της ισχύος αυτού.
  \item[2.] Συνταγματικαί πράξεις, εκδοθείσαι από της 24ης Ιουλίου 1974 μέχρι της συγκλήσεως της Ε' Αναθεωρητικής Βουλής, ως και Ψηφίσματα ταύτης, εξακολουθούν να ισχύουν και κατά τας αντιτιθεμένας προς το Σύνταγμα διατάξεις αυτών επιτρεπομένης της δια νόμου τροποποιήσεως ή καταργήσεως αυτών. Από της ενάρξεως ισχύος  του Συντάγματος καταργείται η διάταξις του άρθρου 8 της 3ης συντακτικής πράξεως από 3.9.1974, ως προς το όριον ηλικίας αποχωρήσεως των  καθηγητών ανωτάτων εκπαιδευτικών ιδρυμάτων.
  \item[3.] Διατηρούνται εν ισχύι α) το άρθρον 2 του υπ' αριθμ. 700 της 9/9 Οκτωβρίου 1974 προεδρικού διατάγματος “περί μερικής επαναφοράς εν ισχύΐ  των άρθρων 5, 6,  8, 10, 12, 14, 95 και 97 του Συντάγματος και άρσεως του νόμου περί καταστάσεως πολιορκίας” και β) το ν.δ. υπ' αριθ. 167 της 16/16/ Νοεμβρίου 1974  “περί χορηγήσεως του ενδίκου μέσου της εφέσεως κατά των αποφάσεων του στρατιωτικού δικαστηρίου”, επιτρεπομένης της δια νόμου τροποποιήσεως ή καταργήσεως αυτών.
  \item[4.] Το ψήφισμα της 16/29 Απριλίου 1952 διατηρείται εν ισχύϊ  επί εξ μήνας από της ενάρξεως του παρόντος. Εντός της προθεσμίας ταύτης επιτρέπεται η δια νόμου τροποποίησις συμπλήρωσις ή κατάργησις των εν παραγράφω 1του άρθρου 3 του ανωτέρω ψηφίσματος αναφερομένων συντακτικών πράξεων και ψηφισμάτων ή η διατήρησίς των εκ τούτων, εν όλω ή εν μέρει  και μετά την παρέλευσιν της προθεσμίας ταύτης, υπό τον περιορισμόν ότι αι τροποιούμεναι, συμπληρούμεναι ή διατηρούμεναι εν ισχύϊ διατάξεις δεν δύναται να είναι αντίθετοι προς το παρόν Σύνταγμα.
  \item[5.] Έλληνες στερηθέντες, μέχρι της ενάρξεως ισχύος του παρόντος, καθ' οιονδήποτε τρόπον της ιθαγενείας των, ανακτούν ταύτην κατόπιν κρίσεως υπό ειδικών επιτροπών εκ δικαστικών λειτουργών, ως νόμος ορίζει.
  \item[6.] Διατηρείται εν ισχύϊ η διάταξις του άρθρου 19 του ν.δ. 3370/1955 “περί κυρώσεως του Κώδικος Ελληνικής Ιθαγενείας” μέχρι ης δια νόμου καταργήσεώς της.
\end{enumerate}

\Article{112}{}
\TOCArticle{112}{}
\begin{enumerate}
  \item[1.] Επί θεμάτων, προς ρύθμισιν των οποίων προβλέπεται ρητώς υπό διατάξεων του παρόντος Συντάγματος η έκδοσις νόμου, οι κατά την έναρξιν της ισχύος αυτού υφιστάμενοι κατά περίπτωσιν νόμοι ή διοικητικαί πράξεις κοινοτικού χαρακτήρος, εξαιρέσει των αντικειμένων εις τας διατάξεις του Συντάγματος, εξακολουθούν να ισχύουν μέχρι της εκδόσεως του κατά περίπτωσιν νόμου.
  \item[2.] Αι διατάξεις των άρθρων 109 παράγραφος 2 και 79 παράγραφος 8 τίθενται εις εφαρμογήν από της ενάρξεως της ισχύος του υπό εκάστης τούτων ειδικώς προβλεπομένου νόμου, εκδιδομένου το βραδύτερον μέχρι τέλους του έτους 1976. Μέχρις ενάρξεως της ισχύος του υπό της παραγράφου 2 του άρθρου 109 προβλεπομένου νόμου εξακολουθεί εφαρμοζομένη η κατά την έναρξιν ισχύος του 

Συντάγματος υφισταμένη συνταγματική και νομοθετική ρύθμισις.
  \item[3.] Κατά την έννοιαν της από 5 Οκτωβρίου 1974 συντακτικής πράξεως, διατηρουμένης εν ισχύϊ, η αναστολή εκτελέσεως των καθηκόντων από της εκλογής αυτών ως βουλευτών, καθ' όλην την παρούσαν βουλευτικήν περίοδον δεν εκτείνεται εις την διδασκαλίαν, έρευναν, συγγραφικήν εργασίαν και επιστημονικήν απασχόλησιν εις τα εργαστήρια και τα σπουδαστήρια των οικείων σχολών, αποκλειομένης της συμμετοχής τούτων εις την διοίκησιν των σχολών και εις την εκλογήν του διδακτικού εν γένει προσωπικού ή την εξέτασιν των σπουδαστών. 
  \item[4.] Η εφαρμογή της παραγράφου 3 του άρθρου16 περί ετών υποχρεωτικής φοιτήσεως θα ολοκληρωθή επί τη βάσει νόμου εντός πενταετίας από της ενάρξεως της ισχύος του παρόντος.
\end{enumerate}

\Article{113}{}
\TOCArticle{113}{}
Ο Κανονισμός της Βουλής, ως και τα εις αυτόν αναφερόμενα ψηφίσματα και οι νόμοι περί της λειτουργίας της Βουλής, εξακολουθούν ισχύοντες μέχρι της ενάρξεως του νέου Κανονισμού της Βουλής, εξαιρέσει των αντικειμένων εις ορισμούς του Συντάγματος.

	Προκειμένου περί της λειτουργίας των κατά τα άρθρα 70 και 71 του Συντάγματος Τμημάτων της Βουλής έχουν συμπληρωματικήν εφαρμογήν αι διατάξεις του τελευταίου Κανονισμού των εργασιών της Ειδικής Νομοθετικής Επιτροπής του άρθρου 35 του Συντάγματος της 1ης Ιανουαρίου 1952,κατά τα εν άρθρω 3 του υπό στοιχ. Α' ψηφίσματος της 24.12.1974 ειδικώτερον οριζόμενα. Μέχρι της ενάρξεως της ισχύος του νέου Κανονισμού της Βουλής, η Επιτροπή του άρθρου71 του Συντάγματος συγκροτείται εξ εξήκοντα τακτικών μελών και τριάκοντα αναπληρωματικών, επιλεγομένων υπό του Προέδρου της Βουλής, εξ όλων των κομμάτων και ομάδων και κατ' αναλογίαν της δυνάμεως αυτών. Εγειρομένης αμφισβητήσεως μέχρι της δημοσιεύσεως του νέου Κανονισμού περί των εφαρμοστέων εκάστοτε διατάξεων αποφαίνεται η Ολομέλεια ή το Τμήμα της Βουλής, εις την λειτουργίαν του οποίου προεκλήθη το ζήτημα.
	
\Article{114}{}
\TOCArticle{114}{}
\begin{enumerate}
  \item[1.] Η εκλογή του πρώτου Προέδρου της Δημοκρατίας δέον να πραγματοποιηθή το βραδύτερον εντός διμήνου από της δημοσιεύσεως του Συντάγματος εις ειδικήν συνεδρίασιν της Βουλής, προσκαλουμένης προ πέντε τουλάχιστον ημερών υπό του Προέδρου αυτής, τηρουμένων αναλόγως των ορισμών του Κανονισμού της Βουλής περί εκλογής του Προέδρου αυτής.

	Ο εκλεγόμενος Πρόεδρος της Δημοκρατίας αναλαμβάνει την άσκησιν των καθηκόντων αυτού από της δόσεως του όρκου του, εντός πέντε το βραδύτερον ημερών από της εκλογής αυτού.

	Ο κατά το άρθρον 49 παράγραφος 5 νόμος περί ρυθμίσεως των αφορώντων εις την ευθύνην του Προέδρου της Δημοκρατίας θεμάτων εκδίδεται υποχρεωτικώς μέχρι της 31 Δεκεμβρίου 1975.

	Μέχρι της ενάρξεως της ισχύος του κατά την παράγραφον 3 του άρθρου 33 νόμου τα εν αυτή θέματα διέπονται υπό των διατάξεων των αφορωσών εις τον προσωρινόν Πρόεδρον της Δημοκρατίας.
  \item[2.] Από της ενάρξεως της ισχύος του Συντάγματος και μέχρι της υπό του οριστικού Προέδρου της Δημοκρατίας αναλήψεως της ασκήσεως των καθηκόντων του, ο προσωρινός Πρόεδρος της Δημοκρατίας ασκεί τας δια του Συντάγματος αναγνωριζομένας εις τον Πρόεδρον της Δημοκρατίας αρμοδιότητας, υπό τους εν άρθρω 2 του υπό στοιχ. Β' από 24.12.1974 ψηφίσματος της Ε' Αναθεωρητικής Βουλής περιορισμούς.
\end{enumerate}

\Article{115}{}
\TOCArticle{115}{}
\begin{enumerate}
  \item[1.] Μέχρι της εκδόσεως του εν άρθρω 86 παράγραφος 1 προβλεπομένου νόμου, έχουν εφαρμογήν αι κείμεναι διατάξεις περί διώξεως, ανακρίσεως και εκδικάσεως των κατά τα άρθρα 49 παράγραφος 1 και 85 πράξεων και παραλείψεων.
  \item[2.] Ο εν άρθρω 100 προβλεπόμενος νόμος δέον να εκδοθή το βραδύτερον εντός έτους από της ισχύος του Συντάγματος. Μέχρι της εκδόσεως  τούτου και της ενάρξεως της λειτουργίας του συνιστωμένου Ανωτάτου Ειδικού Δικαστηρίου:
	\begin{enumerate}
	  \item[α)] Αι αμφισβητήσεις, περί ων η παράγραφος 2 του άρθρου 55 και το άρθρον 57 επιλύονται  δι' αποφάσεως της Βουλής κατά τας επί προσωπικών θεμάτων διατάξεις του Κανονισμού αυτής.
	  \item[β)] Ο έλεγχος του κύρους και των αποτελεσμάτων δημοψηφίσματος ενεργουμένου κατ' άρθρον 44 παράγραφος 2 ως και η εκδίκασις ενστάσεων κατά του κύρους και των αποτελεσμάτων των βουλευτικών εκλογών κατά το άρθρον 58 ασκείται υπό του κατ' άρθρον 73 του από 1 Ιανουαρίου 1952 Συντάγματος Ειδικού Δικαστηρίου, εφαρμοζομένης της διαδικασίας των άρθρων 116 επομ. Του Π.Δ. 650/1974. 
	  \item[γ)] Η άρσις των εν άρθρω 100 παράγραφος 1 εδαφ. Δ' συγκρούσεων υπάγεται εις την δικαιοδοσίαν  του κατά το άρθρον 65 του από 1ης Ιανουαρίου 1952 Συντάγματος, Δικαστηρίου Συγκρούσεως  Καθηκόντων, διατηρουμένων προσωρινώς εν ισχύϊ  και των νόμων περί της οργανώσεως, λειτουργίας και διαδικασίας ενώπιον του Δικαστηρίου τούτου.
	\end{enumerate}
  \item[3.] Μέχρι της ενάρξεως ισχύος του υπό του άρθρου 99 προβλεπομένου νόμου, αι αγωγαί κακοδικίας εκδικάζονται κατά τα εν άρθρω 110 του Συντάγματος της 1ης Ιανουαρίου 1952 οριζόμενα υπό του προβλεπομένου δι' αυτού δικαστηρίου και κατά την εν ισχυϊ διαδικασίαν κατά τον χρόνον δημοσιεύσεως του παρόντος Συντάγματος.
  \item[4.] Μέχρι της ενάρξεως ισχύος του νόμου του προβλεπομένου υπό της παραγράφου 3 ου άρθρου 87 ως και μέχρι της συγκροτήσεως των υπό των άρθρων 90 παράγραφοι 1 και 2 και 91 προβλεπομένων δικαστικών και πειθαρχικών συμβουλίων, εξακολουθούν ισχύουσαι αι κατά την έναρξιν της ισχύος του Συντάγματος υφιστάμεναι σχετικαί διατάξεις. Οι περί των ως άνω θεμάτων νόμοι δέον όπως εκδοθούν το βραδύτερον εντός έτους από της ισχύος του παρόντος Συντάγματος.
  \item[5.] Μέχρι της ενάρξεως ισχύος των εν άρθρω 92 αναφερομένων νόμων, εξακολουθούν ισχύουσαι αι κατά την έναρξιν της ισχύος του παρόντος Συντάγματος υφιστάμεναι διατάξεις. Οι νόμοι ούτοι δέον όπως εκδοθούν το βραδύτερον εντός έτους από της ισχύος του παρόντος.
  \item[6.] Ο εν άρθρω 57 παράγραφος 5 ειδικός νόμος δέον να εκδοθή εντός εξαμήνου από της ενάρξεως ισχύος του Συντάγματος.
\end{enumerate}

\Article{116}{}
\TOCArticle{116}{}
\begin{enumerate}
  \item[1.] Υφιστάμεναι διατάξεις αντιβαίνουσαι εις το άρθρον 4 παράγραφος 2 παραμένουν εν ισχύϊ μέχρι της δια νόμου καταργήσεώς των, το βραδύτερον δε μέχρι της 31 Δεκεμβρίου 1982.
  \item[2.] Αποκλίσεις εκ των ορισμών της παραγράφου 2 του άρθρου 4 επιτρέπονται μόνον δι' αποχρώντας λόγους εις τας ειδικώς υπό του νόμου οριζομένας περιπτώσεις.
  \item[3.] Κανονιστικαί υπουργικαί αποφάσεις, ως και διατάξεις συλλογικών συμβάσεων ή διαιτητικών αποφάσεων περί ρυθμίσεως αμοιβής της εργασίας, αντιβαίνουσαι εις τας διατάξεις του άρθρου 22 παράγραφος 1 εξακολουθούν ισχύουσαι μέχρι της αντικαταστάσεως αυτών, συντελουμένης το βραδύτερον εντός τριετίας από της ενάρξεως της ισχύος του παρόντος.
\end{enumerate}

\Article{117}{}
\TOCArticle{117}{}
\begin{enumerate}
  \item[1.] Οι κατ' εφαρμογήν του άρθρου 104 του Συντάγματος της 1ης Ιανουαρίου 1952 εκδοθέντες μέχρι της 21.4.1967 νόμοι θεωρούνται ως μη αντικείμενοι εις το παρόν Σύνταγμα και διατηρούνται εν ισχυϊ.
  \item[2.] Επιτρέπεται η κατά παρέκκλισιν του άρθρου 17 νομοθετική ρύθμισις και διάλυσις υφισταμένων εισέτι αγροληψιών και ετέρων εδαφικών βαρών, η εξαγορά της ψιλής κυριότητος υπό εμφυτευτών εμφυτευτικών κτημάτων, ως και η κατάργησις και ρύθμισις ιδιορρύθμων εμπραγμάτων σχέσεων.
  \item[3.] Δημόσια ή ιδιωτικά δάση ή δασικαί εκτάσεις καταστραφείσαι ή καταστρεφόμεναι  εκ πυρκαϊάς ή άλλως πως αποψιλωθείσαι ή αποψιλούμεναι,δεν αποβάλλουν εκ του λόγου τούτου τον ον εκέκτητο προ της καταστροφής των χαρακτήρα και κηρύσσονται υποχρεωτικώς αναδασωτέαι, αποκλειομένης της διαθέσεως τούτων δι' έτερον προορισμόν.
  \item[4.] Η αναγκαστική απαλλοτρίωσις δασών ή δασικών εκτάσεων ανηκουσών εις φυσικά ή νομικά πρόσωπα ιδιωτικού ή δημοσίου δικαίου, επιτρέπεται μόνον υπέρ του Δημοσίου, κατά τα εν άρθρω 17 οριζόμενα, δια λόγους δημοσίας ωφελείας, διατηρουμένης πάντως αμεταβλήτου της μορφής αυτών ως δασικής.
  \item[5.] Αι πέρι της προσαρμογής των κειμένων περί αναγκαστικών απαλλοτριώσεων νόμων προς τας διατάξεις του παρόντος κηρυχθείσαι ή κηρυχθησόμεναι αναγκαστικαί απαλλοτριώσεις διέπονται υπό των κατά τον χρόνον της κηρύξεώς των ισχυουσών διατάξεων.
  \item[6.] Αι παράγραφοι 3 και 5 του άρθρου 24 εφαρμόζονται επί των από της ισχύος των εν αυταίς προβλεπομένων νόμων αναγνωριζομένων ή αναμορφουμένων οικιστικών περιοχών.
\end{enumerate}

\Article{118}{}
\TOCArticle{118}{}
\begin{enumerate}
  \item[1.] Από της ενάρξεως της ισχύος του Συντάγματος οι δικαστικοί λειτουργοί από του βαθμού του προέδρου ή εισαγγελέως εφετών και ανωτέρων ή τούτοις αντιστοίχου, αποχωρούν της υπηρεσίας, ως μέχρι τούδε, άμα τη συμπληρώσει του εβδομηκοστού έτους της ηλικίας των, του ορίου τούτου μειουμένου από του έτους 1977 κατά εν έτος ετησίως μέχρι του εξηκοστού εβδόμου έτους.
  \item[2.] Ανώτατοι δικαστικοί λειτουργοί, μη τελούντες εν υπηρεσία κατά την έναρξιν της ισχύος της από 4/5 Σεπτεμβρίου 1974 συντακτικής πράξεως “περί αντικαταστάσεως της τάξεως και ευρυθμίας εν τη Δικαιοσύνη” υποβιβασθέντες δε βάσει ταύτης, λόγω του χρόνου πραγματοποιήσεως της προαγωγής αυτών, και καθ' ων δεν ησκήθη η κατά το άρθρον 6 της αυτής συντακτικής  πράξεως πειθαρχική δίωξις, παραπέμπονται υποχρεωτικώς υπό του αρμοδίου Υπουργού εντός τριμήνου από της ισχύος του Συντάγματος εις το Ανώτατον Πειθαρχικόν Συμβούλιον.

	Το Ανώτατον Πειθαρχικόν Συμβούλιον αποφαίνεται αν αι συνθήκαι της προαγωγής  εμείωσαν το κύρος και την ιδιάζουσαν υπηρεσιακήν θέσιν του προαχθέντος, αποφαίνεται δε οριστικώς περί της ανακτήσεως ή μη του αυτομάτως απολεσθέντος βαθμού και των προς τον βαθμόν συναπτομένων δικαιωμάτων, αποκλειομένης της απολήψεως αναδρομικώς διαφοράς αποδοχών ή συντάξεως.

	Η απόφασις εκδίδεται υποχρεωτικώς εντός τριμήνου από της παραπομπής.

	Οι εν ζωή κατά βαθμόν στενώτεροι συγγενείς του υποβιβασθέντος και αποβιώσαντος δικαστικού δύναται να ασκήσουν όλα τα εις τους δικαζόμενους αναγνωριζόμενα δικαιώματα ενώπιον του Ανωτάτου Πειθαρχικού Συμβουλίου.
  \item[3.] Μέχρι εκδόσεως του κατ' άρθρον 101 παρ. 3 νόμου, εξακολουθούν εφαρμοζόμεναι αι περί κατανομής αρμοδιοτήτων μεταξύ κεντρικών και περιφερειακών υπηρεσιών ισχύουσαι διατάξεις. Αι διατάξεις αύται δύναται να τροποποιούνται δια μεταφοράς ειδικών αρμοδιοτήτων εκ των κεντρικών εις τας περιφερειακάς υπηρεσίας.
\end{enumerate}

\Article{119}{}
\TOCArticle{119}{}
\begin{enumerate}
  \item[1.] Δια νόμου δύναται να αρθή το καθ' οιονδήποτε τρόπον ισχύσαν απαράδεκτον της ασκήσεως αιτήσεως ακυρώσεως κατά πράξεων εκδοθεισών από 21ης Απριλίου 1967 μέχρι της 23ης Ιουλίου 1974 είτε είχεν ασκηθή τοιαύτη αίτησις είτε μη, αποκλειομένης πάντως της αναδρομικής χορηγήσεως αποδοχών εις τυχόν δικαιωθησομένους εκ του ενδίκου τούτου μέσου.
  \item[2.] Οι δυνάμει νόμου αυτοδικαίως  αποκαθιστάμενοι εις τας δημοσίας θέσεις εκέκτηντο στρατιωτικοί ή δημόσιοι υπάλληλοι, εφ' όσον ήδη απέκτησαν την ιδιότητα του βουλευτού, δύνανται εντός οκταημέρου προθεσμίας να δηλώσουν επιλογήν μεταξύ του βουλευτικού αξιώματος και της δημοσίας αυτών θέσεως. 
\end{enumerate}

\Section{Δ'}{ΑΚΡΟΤΕΛΕΥΤΙΟΣ  ΔΙΑΤΑΞΙΣ}
\Article{120}{}
\TOCArticle{120}{}
\begin{enumerate}
  \item[1.] Το παρόν Σύνταγμα, ψηφισθέν υπό της Ε' Αναθεωρητικής Βουλής των Ελλήνων, υπογράφεται υπό του Προέδρου αυτής και δημοσιεύεται υπό του προσωρινού Προέδρου της Δημοκρατίας δια της Εφημερίδος της Κυβερνήσεως δια διατάγματος προσυπογραφομένου υπό του Υπουργικού Συμβουλίου τίθεται δε εν ισχύϊ από της ενδεκάτης Ιουνίου 1975.
  \item[2.] Ο σεβασμός προς το Σύνταγμα και τους συνάδοντας προς αυτό νόμους και η αφοσίωσις προ την Πατρίδα και την Δημοκρατίαν συνιστούν θεμελιώδη υποχρέωσιν πάντων των Ελλήνων.
  \item[3.] Ο καθ' οιονδήποτε τρόπον σφετερισμός της λαϊκής κυριαρχίας και των εκ ταύτης απορρεουσών εξουσιών διώκεται άμα τη αποκαταστάσει της νομίμου εξουσίας, αφ' ης άρχεται και η παραγραφή του εγκλήματος.
  \item[4.] Η τήρησις του Συντάγματος επαφίεται εις τον πατριωτισμόν των Ελλήνων, δικαιουμένων και υποχρεουμένων εις την δια παντός μέσου αντίστασιν κατά οιουδήποτε επιχειρούντος την βιαίαν κατάλυσιν αυτού.
\end{enumerate}

\noindent\DateSignature{Εν Αθήναις}{9}{6}{1975}
\PersonSignature{Ο ΠΡΟΕΔΡΟΣ ΤΗΣ ΒΟΥΛΗΣ}{ΚΩΝΣΤ. Ε. ΠΑΠΑΚΩΝΣΤΑΝΤΙΝΟΥ}

\end{BigQuote}

  \item[Β.] Το Ημέτερον Υπουργικόν Συμβούλιον θέλει προσυπογράψει και δημοσιεύσει το παρόν, επιτιθεμένης της Μεγάλης του Κράτους Σφραγίδος.
\end{enumerate}

\end{multicols}

\noindent\DateSignature{Εν Αθήναις}{9}{6}{1975}
\PersonSignature{Ο ΠΡΟΕΔΡΟΣ ΤΗΣ ΔΗΜΟΚΡΑΤΙΑΣ}{ΜΙΧΑΗΛ ΣΤΑΣΙΝΟΠΟΥΛΟΣ}
\PersonSignature{ΤΟ ΥΠΟΥΡΓΙΚΟΝ ΣΥΜΒΟΥΛΙΟΝ}{}
\PersonSignature{Ο ΠΡΩΘΥΠΟΥΡΓΟΣ}{ΚΩΝΣΤΑΝΤΙΝΟΣ  ΚΑΡΑΜΑΝΛΗΣ}
\PersonSignature{ΤΑ ΜΕΛΗ}{ΠΑΝΑΓ. ΠΑΠΑΛΗΓΟΥΡΑΣ, ΙΩΑΝ. ΜΠΟΥΤΟΣ, ΓΕΩΡΓ. ΡΑΛΛΗΣ, ΔΗΜ. ΜΠΙΤΣΙΟΣ, ΚΩΝΣΤ. ΣΤΕΦΑΝΟΠΟΥΛΟΣ, ΚΩΝΣΤ. ΣΤΕΦΑΝΑΚΗΣ, ΣΟΛΩΝ ΓΚΙΚΑΣ, ΚΩΝΣΤ.ΤΡΥΠΑΝΗΣ, ΕΥΑΓΓ. ΔΕΒΛΕΤΟΓΛΟΥ, ΙΠΠΟΚΡ. ΙΟΔΑΝΟΓΛΟΥ, ΚΩΝΣΤ. ΚΟΝΟΦΑΓΟΣ, ΙΩΑΝ. ΒΑΡΒΙΤΣΙΩΤΗΣ, ΚΩΝΣΤ. ΧΡΥΣΑΝΘΟΠΟΥΛΟΣ, ΧΡΙΣΤΟΦ. ΣΤΡΑΤΟΣ, ΓΕΩΡΓ. ΒΟΓΙΑΤΖΗΣ, ΑΛΕΞ. ΠΑΠΑΔΟΓΓΟΝΑΣ, ΝΙΚΟΛ. ΜΑΡΤΗΣ.}

\validated{Εν Αθήναις}{9}{6}{1975}
\PersonSignature{Ο ΕΠΙ ΤΗΣ ΔΙΚΑΙΟΣΥΝΗΣ ΥΠΟΥΡΓΟΣ}{Κωσταντίνος Στεφανάκης}

\end{document}
